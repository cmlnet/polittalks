\appendix

\chapter{Anhang}

\section{Zusätzliche Tabellen}

\subsection{Alle Talkshowfolgen zum Thema „Eurokrise“}

\begin{table}[!htp]
	\caption{Alle Talkshowfolgen zum Thema „Eurokrise“}
	\centering
	\resizebox{0.63\textheight}{!}{%
		\begin{tabular}{@{}lll@{}}
			\toprule
			\textbf{Datum} & \textbf{Sendung}          & \textbf{Titel}                                                                           \\ \midrule
			31.08.11       & log in                    & Reichensteuer gegen Eurokrise – sollen Millionäre mehr zahlen?                           \\
			06.09.11       & Menschen bei Maischberger & Das Euro-Debakel: Ein Schrecken ohne Ende?                                               \\
			08.09.11       & maybrit illner            & Union der Verschwender: Europa ist, wenn Deutschland zahlt                               \\
			25.09.11       & Günther Jauch             & Der Kampf um den Euro – Bundeskanzlerin Angela Merkel zu Gast bei Günther Jauch          \\
			26.09.11       & hart aber fair            & Der Euro-Aufstand – Deutschland vor der Gewissensfrage?                                  \\
			28.09.11       & Anne Will                 & Die Euro-Abstimmung - Riskieren wir morgen alles?                                        \\
			28.09.11       & log in                    & Euro-top, Euro-hop: sind wir noch zu retten?                                             \\
			29.09.11       & maybrit illner            & Europa - einfach unbezahlbar? - Deutschland unterm Rettungsschirm                        \\
			13.10.11       & maybrit illner            & Griechen pleite, Banken in Not - wer rettet den Steuerzahler?                            \\
			18.10.11       & Menschen bei Maischberger & Die Angst wächst: Eurokalypse now?                                                       \\
			20.10.11       & maybrit illner            & Zocker an den Pranger - ran an die Banken, raus aus der Krise?                           \\
			23.10.11       & Günther Jauch             & Klartext in der Krise – Helmut Schmidt und Peer Steinbrück zu Gast bei Günther Jauch     \\
			24.10.11       & hart aber fair            & Bürger gegen Banken: Wut und Angst im Euroland                                           \\
			26.10.11       & log in                    & Der Euro-Poker - Retten wir die Richtigen?                                               \\
			27.10.11       & Beckmann                  & Europa vor dem Abgrund – wie sicher ist unser Geld?                                      \\
			27.10.11       & maybrit illner            & Europa zankt, der Bürger zahlt - Euro gerettet?                                          \\
			30.10.11       & Günther Jauch             & Banken an die Leine! Wie bekommen wir die Finanzmärkte in den Griff?                     \\
			03.11.11       & maybrit illner            & Europa in der Falle - gefährdet Demokratie unseren Wohlstand?                            \\
			06.11.11       & Günther Jauch             & Chaos-Tage in Athen – wer will die Griechen jetzt noch retten?                           \\
			08.12.11       & Beckmann                  & Auswege aus der Krise – sind die Grenzen des Wachstums erreicht?                         \\
			08.12.11       & maybrit illner            & Merkel, Macht und Märkte - Deutschland bald auf Ramschniveau?                            \\
			11.12.11       & Günther Jauch             & Nach dem Krisen-Gipfel – Geht's jetzt auch mit Deutschland bergab?                       \\
			15.12.11       & maybrit illner            & Tschüss, Krisenjahr! Kriegt Europa noch die Kurve?                                       \\
			25.01.12       & log in                    & Erst wachsen, dann platzen - mit dem Kapitalismus in den Bankrott?                       \\
			02.02.12       & maybrit illner            & Milliardengrab Griechenland - Rettung unmöglich?                                         \\
			08.02.12       & log in                    & Zurück in die Zukunft - erst Hellas bankrott, dann wir?                                  \\
			15.02.12       & Anne Will                 & Griechenland brennt, Deutschland zahlt - Euro-Rettung um jeden Preis?                    \\
			28.02.12       & Menschen bei Maischberger & Der letzte Sirtaki: Griechen bankrott, Deutsche zahlen trotzdem?                         \\
			01.03.12       & maybrit illner            & Dauerauftrag für Athen - wann verliert Deutschland die Geduld?                           \\
			26.04.12       & maybrit illner            & Rückkehr der Euro-Krise - Aus für Merkels Spardiktat?                                    \\
			08.05.12       & Menschen bei Maischberger & Alle gegen Merkel – Europa in Gefahr?                                                    \\
			09.05.12       & Anne Will                 & Griechen und Franzosen wählen den Sparkurs ab – zahlt Deutschland die Euro-Zeche allein? \\
			10.05.12       & maybrit illner            & Wer sparen will, wird abgestraft. Wählt Europa lieber neue Schulden?                     \\
			20.05.12       & Günther Jauch             & Brauchen wir den Euro wirklich? - Thilo Sarrazin gegen Peer Steinbrück                   \\
			23.05.12       & Anne Will                 & Spar-Angie gegen Spendier-François - das letzte Euro-Gefecht?                            \\
			24.05.12       & maybrit illner            & Alle pfeifen auf die Schulden - Wer hört noch auf die Kanzlerin?                         \\
			18.06.12       & hart aber fair            & Die Griechenwahl – statt Ende mit Schrecken jetzt Schecks ohne Ende?                     \\
			20.06.12       & Anne Will                 & Nach der Krise ist vor der Krise - ist unser Erspartes wirklich sicher?                  \\
			20.06.12       & log in                    & Haben wir den Euro schon verspielt?                                                      \\
			04.07.12       & Anne Will                 & Europas Schulden, unsere Schulden - ist Kanzlerin Merkel umgefallen?                     \\
			05.07.12       & maybrit illner            & Alle Macht den Schulden - Wird Deutschland in Brüssel über den Tisch gezogen?            \\
			23.08.12       & maybrit illner            & Arm gegen Reich, Nord gegen Süd - Wer zahlt den Preis für die Krise?                     \\
			06.09.12       & Beckmann                  & Regieren Banken die Politik?                                                             \\
			09.09.12       & Günther Jauch             & Im Namen des Volkes! Müssen wir die Euro-Rettung stoppen?                                \\
			12.09.12       & Anne Will                 & Das Euro-Urteil - ein guter Tag für Deutschland?                                         \\
			12.09.12       & log in                    & Zahlen, bis es kracht: Zum Euro verurteilt?                                              \\
			13.09.12       & maybrit illner            & Zur Rettung verurteilt -  Was ist uns Europa wert                                        \\ \bottomrule
		\end{tabular}%
	}
	\label{tab:anhang_folgen_eurokrise}
\end{table}

\pagebreak

\subsection{Themenbereiche nach Sendungen}

\begin{table}[!ht]
	\caption{Themenbereiche nach Sendungen}
	\centering
	\resizebox{\textwidth}{!}{%
		\begin{tabular}{@{}lccccccccc@{}}
			\toprule
			&
			\textbf{Politik} &
			\textbf{Wirtschaft} &
			\textbf{Gesundheit} &
			\textbf{Sport} &
			\textbf{Prominenz / Personality} &
			\textbf{Service} &
			\textbf{Religion} &
			\textbf{Kriminalität / Katastrophen} &
			\textbf{Sonstiges} \\ \midrule
			\textbf{Anne Will (n=37)}                 & 54,10 \% & 27,00 \% & 2,70 \%  & 5,40 \% & 0,00 \%  & 2,70 \% & 5,40 \% & 0,00 \% & 2,70 \%  \\
			\textbf{Beckmann (n=43)}                  & 37,20 \% & 11,60 \% & 11,60 \% & 4,70 \% & 23,30 \% & 0,00 \% & 2,30 \% & 7,00 \% & 2,30 \%  \\
			\textbf{Günther Jauch (n=40)}             & 65,00 \% & 20,00 \% & 7,50 \%  & 0,00 \% & 2,50 \%  & 0,00 \% & 0,00 \% & 0,00 \% & 5,00 \%  \\
			\textbf{hart aber fair (n=36)}            & 33,30 \% & 13,80 \% & 16,70 \% & 2,80 \% & 0,00 \%  & 2,80 \% & 2,80 \% & 0,00 \% & 27,80 \% \\
			\textbf{Menschen bei Maischberger (n=41)} & 24,40 \% & 9,80 \%  & 19,50 \% & 2,40 \% & 12,20 \% & 4,90 \% & 7,30 \% & 4,90 \% & 14,60 \% \\ \midrule
			\textbf{ARD-Talks (n=197)}                & 42,60 \% & 16,20 \% & 11,70 \% & 3,00 \% & 8,10 \%  & 2,00 \% & 3,60 \% & 2,50 \% & 10,20 \% \\ \midrule
			\textbf{log in (n=45)}                    & 46,70 \% & 22,30 \% & 8,90 \%  & 4,40 \% & 0,00 \%  & 0,00 \% & 4,40 \% & 0,00 \% & 13,30 \% \\
			\textbf{maybrit illner (n=41)}            & 48,80 \% & 41,50 \% & 4,90 \%  & 2,40 \% & 0,00 \%  & 0,00 \% & 2,40 \% & 0,00 \% & 0,00 \%  \\ \midrule
			\textbf{Alle Sendungen (n=283)}           & 44,20 \% & 20,80 \% & 10,20 \% & 3,20 \% & 5,70 \%  & 1,40 \% & 3,50 \% & 1,80 \% & 9,20 \%  \\ \bottomrule
		\end{tabular}%
	}
	\label{tab:anhang_themenbereiche_sendungen}
\end{table}

\pagebreak

\subsection{Talkshows im deutschen Fernsehen}

\begin{table}[!htp]
	\centering
	\caption{Talkshows im deutschen Fernsehen (Stand: August 2012)}
	\resizebox{0.35\textheight}{!}{%
		\begin{tabular}{@{}ll@{}}
			\toprule
			\textbf{Sender} & \textbf{Titel}                                   \\ \midrule
			3sat            & Peter Voß fragt$\ldots$                               \\
			3sat            & Debatte                                          \\ \midrule
			ARD             & Anne Will                                        \\
			ARD             & Beckmann                                         \\
			ARD             & Günther Jauch                                    \\
			ARD             & hart aber fair                                   \\
			ARD             & Menschen bei Maischberger                        \\
			ARD             & Presseclub                                       \\
			ARD             & Waldis Club / Sportschau Club                    \\ \midrule
			ARTE            & Philosophie                                      \\ \midrule
			BR              & BürgerForum live                                 \\
			BR              & Der Sonntags-Stammtisch                          \\
			BR              & Jetzt red i                                      \\
			BR              & Münchner Runde                                   \\ \midrule
			HR              & Meinungsmacher                                   \\
			HR              & Schlossplatz                                  \\ \midrule
			MDR             & Riverboat                                        \\ \midrule
			n-tv            & Bei Brender!                                     \\
			n-tv            & busch@n-tv                                       \\
			n-tv            & Das Duell bei n-tv                               \\ \midrule
			N24             & Deutschland akut                                 \\
			N24             & Studio Friedman                                  \\ \midrule
			NDR             & NDR Talk Show                                    \\
			NDR             & Tietjen und Hirschhausen                         \\ \midrule
			Phoenix         & Forum Politik                                    \\
			Phoenix         & Forum Wirtschaft                                 \\
			Phoenix         & Internationaler Frühschoppen                     \\
			Phoenix         & Phoenix Runde                                    \\
			Phoenix         & Tacheles - Die Talkshow der evangelischen Kirche \\
			Phoenix         & Unter den Linden                                 \\ \midrule
			Radio Bremen    & 3 nach 9                                         \\ \midrule
			RBB             & Dickes B.                                        \\
			RBB             & Thadeusz                                         \\ \midrule
			Sat 1           & Britt – Der Talk um eins                         \\
			Sat 1           & Eins gegen Eins                                  \\ \midrule
			Sport 1         & Doppelpass                                       \\ \midrule
			SR              & saartalk                                         \\ \midrule
			SWR             & 2+Leif                                           \\
			SWR             & Menschen der Woche                               \\
			SWR             & Nachtcafé                                        \\ \midrule
			WDR             & Kölner Treff                                     \\
			WDR             & Plasberg persönlich                              \\
			WDR             & Zimmer frei                                      \\ \midrule
			ZDF             & Markus Lanz                                      \\ 
			ZDF             & maybrit illner                                   \\
			ZDF             & Peter Hahne                                      \\ \midrule
			ZDFinfo         & log in                                           \\
			ZDFkultur       & Roche \& Böhmermann                              \\ \bottomrule
		\end{tabular}
	}
	\subcaption*{\tiny Berücksichtigt wurden Sendungen in den privaten deutschen Fernsehsendern (RTL, Sat.1, Kabel 1, Vox, n-tv, N24), das ZDF inklusive seiner digitalen Kanäle, die ARD, die Dritten Programme, 3sat, Arte, Phoenix sowie Sport 1.}
	\label{tab:anhang_talkshows}
\end{table}

\pagebreak

\subsection{Politikbereiche nach Sendungen}

\begin{table}[ht]
	\centering
	\caption{Politikbereiche nach Sendungen}
	\resizebox{\textwidth}{!}{%
		\begin{tabular}{@{}lcccccc@{}}
			\toprule
			& \textbf{Innenpolitik} & \textbf{Außenpolitik} & \textbf{Umwelt- / Energiepolitik} & \textbf{Sozialpolitik} & \textbf{Bildungspolitik} & \textbf{Sonstige Politik} \\ \midrule
			\textbf{\begin{tabular}[c]{@{}l@{}}Anne Will\\ (n=20)\end{tabular}}       & 50,00 \% & 5,00 \%  & 5,00 \%  & 35,00 \% & 0,00 \% & 5,00 \%  \\
			\textbf{\begin{tabular}[c]{@{}l@{}}Beckmann\\ (n=16)\end{tabular}}        & 81,30 \% & 6,30 \%  & 0,00 \%  & 12,50 \% & 0,00 \% & 0,00 \%  \\
			\textbf{\begin{tabular}[c]{@{}l@{}}Günther Jauch\\ (n=26)\end{tabular}}   & 61,50 \% & 7,70 \%  & 7,70 \%  & 15,50 \% & 3,80 \% & 3,80 \%  \\
			\textbf{\begin{tabular}[c]{@{}l@{}}hart aber fair\\ (n=12)\end{tabular}}  & 41,70 \% & 8,30 \%  & 8,30 \%  & 33,30 \% & 0,00 \% & 8,30 \%  \\
			\textbf{\begin{tabular}[c]{@{}l@{}}Maischberger\\ (n=10)\end{tabular}}    & 40,00 \% & 0,00 \%  & 10,00 \% & 40,00 \% & 0,00 \% & 10,00 \% \\ \midrule
			\textbf{\begin{tabular}[c]{@{}l@{}}ARD-Talks\\ (n=84)\end{tabular}}       & 57,10 \% & 6,00 \%  & 6,00 \%  & 25,00 \% & 1,20 \% & 4,80 \%  \\ \midrule
			\textbf{\begin{tabular}[c]{@{}l@{}}log in\\ (n=21)\end{tabular}}          & 61,90 \% & 4,80 \%  & 14,30 \% & 14,30 \% & 0,00 \% & 4,80 \%  \\
			\textbf{\begin{tabular}[c]{@{}l@{}}maybrit illner\\ (n=20)\end{tabular}}  & 60,00 \% & 10,00 \% & 10,00 \% & 20,00 \% & 0,00 \% & 0,00 \%  \\ \midrule
			\textbf{\begin{tabular}[c]{@{}l@{}}Alle Sendungen\\ (n=125)\end{tabular}} & 58,40 \% & 6,40 \%  & 8,00 \%  & 22,40 \% & 0,80 \% & 4,00 \%  \\ \bottomrule
		\end{tabular}%
	}
	\label{tab:anhang_politikbereiche_sendungen}
\end{table}

\pagebreak

\subsection{Gästegruppen nach Sendungen}

\begin{table}[ht]
	\centering
	\caption{Gästegruppen nach Sendungen}
	\resizebox{\textwidth}{!}{%
		\begin{tabular}{@{}lccccccccc@{}}
			\toprule
			&
			\textbf{Politik} &
			\textbf{Wirtschaft} &
			\textbf{Journalismus} &
			\textbf{Experten} &
			\textbf{Kultur / Sport} &
			\textbf{Zivilgesellschaft} &
			\textbf{Religion} &
			\textbf{"Normalbürger"} &
			\textbf{Sonstige} \\ \midrule
			\textbf{Anne Will}      & 29,40 \% & 12,10 \% & 18,70 \% & 16,40 \% & 15,00 \% & 3,70 \%  & 1,40 \% & 3,30 \%  & 0,00 \% \\
			\textbf{Beckmann}       & 20,90 \% & 2,60 \%  & 16,30 \% & 16,30 \% & 25,00 \% & 1,50 \%  & 0,00 \% & 17,30 \% & 0,00 \% \\
			\textbf{Günther Jauch}  & 45,30 \% & 6,90 \%  & 15,80 \% & 14,30 \% & 7,40 \%  & 1,50 \%  & 1,50 \% & 7,40 \%  & 0,00 \% \\
			\textbf{hart aber fair} & 21,80 \% & 11,90 \% & 20,70 \% & 13,50 \% & 19,20 \% & 6,20 \%  & 1,60 \% & 5,20 \%  & 0,00 \% \\
			\textbf{Maischberger}   & 14,70 \% & 11,80 \% & 10,10 \% & 21,00 \% & 25,20 \% & 3,80 \%  & 2,90 \% & 10,10 \% & 0,40 \% \\ \midrule
			\textbf{ARD-Talks}      & 26,10 \% & 9,20 \%  & 16,10 \% & 16,50 \% & 18,50 \% & 3,40 \%  & 1,50 \% & 8,60 \%  & 0,10 \% \\ \midrule
			\textbf{log in}         & 38,10 \% & 5,60 \%  & 10,00 \% & 12,50 \% & 11,90 \% & 15,00 \% & 2,50 \% & 4,40 \%  & 0,00 \% \\
			\textbf{maybrit illner} & 38,00 \% & 15,60 \% & 11,80 \% & 13,90 \% & 8,40 \%  & 7,20 \%  & 0,80 \% & 4,20 \%  & 0,00 \% \\ \midrule
			\textbf{\begin{tabular}[c]{@{}l@{}}Alle Sendungen\\ (n=1441)\end{tabular}} &
			29,40 \% &
			9,90 \% &
			14,70 \% &
			15,60 \% &
			16,10 \% &
			5,30 \% &
			1,50 \% &
			7,40 \% &
			0,10 \% \\ \bottomrule
		\end{tabular}%
	}
	\label{tab:anhang_gaestegruppen_sendungen}
\end{table}

\pagebreak

\subsection{Parteizugehörigkeit in allen Sendungen}

\begin{table}[ht]
	\centering
	\caption{Parteizugehörigkeit in allen Sendungen}
	\resizebox{\textwidth}{!}{%
		\begin{tabular}{@{}llllllll@{}}
			\toprule
			&
			\textbf{CDU / CSU} &
			\textbf{FDP} &
			\textbf{SPD} &
			\textbf{Grüne} &
			\textbf{Linkspartei} &
			\textbf{Piratenpartei} &
			\textbf{Sonstige} \\ \midrule
			\textbf{\begin{tabular}[c]{@{}l@{}}Anne Will\\ (n=62)\end{tabular}} &
			37,20 \% &
			17,70 \% &
			17,70 \% &
			4,90 \% &
			17,70 \% &
			1,60 \% &
			3,20 \% \\
			\textbf{\begin{tabular}[c]{@{}l@{}}Beckmann\\ (n=38)\end{tabular}} &
			34,20 \% &
			15,80 \% &
			31,60 \% &
			10,50 \% &
			2,60 \% &
			5,30 \% &
			0,00 \% \\
			\textbf{\begin{tabular}[c]{@{}l@{}}Günther Jauch\\ (n=88)\end{tabular}} &
			36,40 \% &
			17,00 \% &
			22,70 \% &
			9,10 \% &
			12,50 \% &
			2,30 \% &
			0,00 \% \\
			\textbf{\begin{tabular}[c]{@{}l@{}}hart aber fair\\ (n=42)\end{tabular}} &
			40,40 \% &
			16,70 \% &
			21,40 \% &
			16,70 \% &
			2,40 \% &
			0,00 \% &
			2,40 \% \\
			\textbf{\begin{tabular}[c]{@{}l@{}}Maischberger\\ (n=35)\end{tabular}} &
			37,10 \% &
			8,60 \% &
			28,60 \% &
			5,70 \% &
			11,40 \% &
			2,90 \% &
			5,70 \% \\ \midrule
			\textbf{\begin{tabular}[c]{@{}l@{}}ARD-Talks\\ (n=265)\end{tabular}} &
			37,00 \% &
			15,80 \% &
			23,30 \% &
			9,10 \% &
			0,00 \% &
			2,30 \% &
			1,90 \% \\ \midrule
			\textbf{\begin{tabular}[c]{@{}l@{}}log in\\ (n=60)\end{tabular}} &
			31,60 \% &
			20,00 \% &
			10,00 \% &
			16,70 \% &
			11,70 \% &
			10,00 \% &
			0,00 \% \\
			\textbf{\begin{tabular}[c]{@{}l@{}}maybrit illner\\ (n=87)\end{tabular}} &
			34,60 \% &
			19,50 \% &
			19,50 \% &
			14,90 \% &
			8,00 \% &
			3,50 \% &
			0,00 \% \\ \midrule
			\textbf{\begin{tabular}[c]{@{}l@{}}Alle Sendungen\\ (n=326)\end{tabular}} & 35,80 \% & 17,20 \% & 20,60 \% & 11,40 \% & 10,20 \% & 3,60 \% & 1,20 \% \\ \midrule
			\textbf{Wahlergebnis Bundestagswahl 2009} &
			33,80 \% &
			14,60 \% &
			23,00 \% &
			10,70 \% &
			11,90 \% &
			2,00 \% &
			4,00 \% \\ \bottomrule
		\end{tabular}%
	}
	\label{tab:anhang_parteizugehoerigkeit_sendungen}
\end{table}

\pagebreak

\section{Transkripte}

Die Transkripte wurden mit der Software {\href{https://www.audiotranskription.de/}{F4} erstellt. Zur besseren Lesbarkeit wurde die Klein- und Großschreibung beibehalten. Die Interpunktion wurde zugunsten der Lesbarkeit vereinfacht.

\subsection{Transkriptionsregeln}
\label{chap:transkriptionsregeln}

Es wurde wörtlich transkribiert. Die Satzstellung wurde beibehalten, auch wenn sie Fehler enthielt.

\begin{itemize}
	\item \textbf{Name}: Sprecher / Sprecherin
	\item \textbf{\#00:52:47-3\# $\ldots$ \#00:53:18-0\#}: Timecodes für Beginn und Ende einer Äußer"-ung
	\item \textbf{// \ldots //}: Überlappendes Sprechen
	\item \textbf{\textbackslash}:Wort- und Satzabbruch
	\item \textbf{((lacht))}: Nonverbale Äußerungen
	\item \textbf{(unv.)}: Unverständliche Äußerung
	\item \textbf{(.), (..), (\ldots)}: Pausenlänge von 1, 2 oder mehr Sekunden
	\item \textbf{kursiv}: Kursiv gesetzte Wörter sind für die Analyse von besonderer Relevanz
	\item \textbf{=}: Frageintonation
	\item \textbf{!}: Betonung am Ende des Satzes
\end{itemize}

\subsection{Weitere Trankriptausschnitte}

Die vollständigen Transkripte finden sich online im zugehörigen \href{https://github.com/cmlnet/polittalks}{GitHub-Repository} \textit{(github.com/cmlnet/polittalks)}.

\begin{description}
	\begin{linenumbers}
		\item \#01:11:18-9\# \textbf{Will}: Oder sagen sie sich, das wär dann meine Schlussfrage, wenn ich erst Kanzlerin bin hab ich noch weniger Zeit?\#01:11:23-4\# 
		
		\item \#01:11:23-4\# \textbf{Leyen}: Ha! Gute Frage, nächste Frage. \#01:11:25-6\# 
		
		\item \#01:11:25-6\# \textbf{Will}: Nee! Gute Frage, gute Antwort. \#01:11:28-5\# 
		
		\item \#01:11:28-5\# \textbf{Leyen}: Nice try, kann ich nur sagen. \#01:11:30-7\# 
		
		\item \#01:11:30-7\# \textbf{Will}: Guter Versuch war das? \#01:11:31-8\# 
		
		\item \#01:11:31-8\# \textbf{Leyen}: Guter Versuch. \#01:11:32-6\# 
		
		\item \#01:11:32-6\# \textbf{Will}: Aber ist was dran? \#01:11:34-3\# 
		
		\item \#01:11:34-3\# \textbf{Leyen}: Dummes Zeug. ((Gelächter)) Ich bin der festen überzeugen, dass und das gerade jetzt in dieser Europadiskussion, mein Gott, ein Glück das wir Angela Merkel als Kanzlerin haben die uns so durch diese Krise führt. ((Applaus)) \#01:11:47-5\# 
		
		\item \#01:11:47-5\# Will: \textbf{Hey}! Damit hören wir auf meine Damen und Herren. Danke an die Runde. Danke an sie. Bis zum nächsten Mittwoch. Tschüss und auf wiedersehen. Und nach uns auch super wach das Nachtmagazin. \#01:11:57-1\#
	\end{linenumbers}
	\captionof{transkript}{Anne Will (12.09.2012)}
	\label{lis:34}
\end{description}

\begin{description}
	\begin{linenumbers}
		\item \#01:13:47-7\# \textbf{Maischberger}: Kompliziert genug, dass wir müssen zum Schluss kommen $\ldots$ \#01:13:50-0\#
		
		\item \#01:13:48-7\# \textbf{Biedenkopf}: // Gut. Das dritte, das dritte ist // äh das dritte ist der wie wir's nennen Marschallplan, weil wir diesen Marschallplan vor Augen haben. Das heißt wir müssen uns in Europa mit der gleichen Frage befassen, und das mein ich jetzt äh mit aller Vorsicht als Metapher, mit der wir uns befassen mussten nach der Wiedervereinigung. Das heißt wie kann man einen Teil Europas so wettbewerbsfähig machen, das er mithalten kann. \#01:14:13-5\# 
		
		\item \#01:14:13-5\# \textbf{Maischberger}: Das ist eine Vision mit der ich jetzt diese Sendung beenden muss, weil wir am Ende sind. Vielen Dank für ihre Beiträge. Wir sind sehr gespannt ob das äh so leicht umzusetzen wird, ob Europa dabei$\ldots$ \#01:14:21-7\#
		
		\item \#01:14:21-0\# \textbf{Biedenkopf}: // Es ist nicht leicht // umzusetzen.  \#01:14:22-2\# 
		
		\item \#01:14:22-2\# \textbf{Maischberger}: Ganz sicher nicht. Ob Europa dabei gewinnt oder ob die Stimmen stärker werden die euro eurokritisch sind, wie die von Herr Henkel. Vielen herzlichen Dank Frau Wagenknecht, Frau Ostermann das sie bei uns waren, Frau von der Leyen, Herr Henkel und Herr Biedenkopf. Ähm hier kommen die Kollegen des Nachtmagazins. Ich wünsch' ihnen noch einen schönen Abend. \#01:14:37-6\#
		
		\item \#01:14:38-7\# \textbf{Biedenkopf}: Ich fand das ganz spannend. \#01:14:39-8\# 
		
		\item \#01:14:39-8\# \textbf{Maischberger}: Jajaja ((lacht)) das ist gut, aber es ist, es ist (unv.)  \#01:14:45-0\#
	\end{linenumbers}
	\captionof{transkript}{Menschen bei Maischberger (18.10.2011)}
	\label{lis:35}
\end{description}

\begin{description}
	\begin{linenumbers}
		\item \#01:07:57-3\# \textbf{Beckmann}: Dann machen wir es anders, machen wir es anders an dieser Stelle und fragen. Frage ich sie Herr Kemmer, wo sind wir in fünf Jahren? \#01:08:03-0\# 
		
		\item \#01:08:03-0\# \textbf{Kemmer}: Das ist ne ganz schwierige Frage. \#01:08:05-2\# 
		
		\item \#01:08:05-2\# \textbf{Beckmann}: Mit welcher Währung? \#01:08:05-9\# 
		
		\item \#01:08:05-9\# \textbf{Kemmer}: Wir haben den Euro in fünf Jahren davon bin ich felsenfest überzeugt. // Wir haben noch Krise in fünf Jahren. // \#01:08:10-3\# 
		
		\item \#01:08:08-4\# \textbf{Beckmann}: // Herr Schäfler? // \#01:08:08-3\# 
		
		\item \#01:08:10-8\# \textbf{Schäfler}: Wir haben schon noch einen Euro. Aber der wird mit hoher Inflation ausgestattet sein. \#01:08:15-3\# 
		
		\item \#01:08:15-3\# \textbf{Beckmann}: Herr Müller? \#01:08:16-6\# 
		
		\item \#01:08:16-6\# \textbf{Müller}: Sehr schwer. Griechenland wird nicht mehr dabei sein. Ob der Gesamteuro erhalten bleibt offen. Wirtschaft deutlich nach unten und in der Folge starke Inflation. Die wird aber nicht gleich kommen, jetzt haben wir deflationär also schwank äh sank sinkenden Preise. Die Inflation wird erst in einigen Jahren sehr massiv kommen.  \#01:08:33-2\# 
		
		\item \#01:08:33-2\# \textbf{Beckmann}: Frau Schmidt, was ist ihre Prophezeiung? \#01:08:35-5\# 
		
		\item \#01:08:35-5\# \textbf{Schmidt}: Euro wird da sein. Aber wer weiß ob in der gegenwärtigen Konstellation. Ich denke auch, deflationäre Tendenzen werden auf die kürzere Sicht vorherrschen. Sachen wir mal in vier, fünf, sechs Jahren wird man sehen, ob sich die Wirtschaft wieder so erholen wird, dass die inflationären Tendenzen die Oberhand gewinnen. Aber bis dahin rezessive Tendenzen, eher Deflation. \#01:09:03-9\# 
		
		\item \#01:09:03-9\# \textbf{Beckmann}: Ich bedank mich sehr bei meiner Runde. Ich kann sagen, dass jetzt im Anschluss das Nachtmagazin folgt mit Gabi Bauer und da wird das Thema nochmal fortgesetzt. Ich würd mich freuen, wenn sie nächste Woche wieder da sind. Machen sie's gut. Noch eine schöne Woche. Schönes Wochenende. Dann bis nächsten Donnerstag. Herzlichen Dank. \#01:09:19-2\#
	\end{linenumbers}
	\captionof{transkript}{Beckmann (06.09.2012)}
	\label{lis:36}
\end{description}

\begin{description}
	\begin{linenumbers}
		\item \#00:55:28-5\# Passadakis: Diese Polemik gegen das griechische Rentensystem ist völlig fehl am Platze. Wenn man sich die Zahlen tatsächlich anschaut, ist das Durchschnittsrenteneintrittsalter in Griechenland 61 Jahre und in Deutschland ist es auch 61 Jahre. Da tut sich also nicht viel. Also diese Polemik ist völlig fehl geleitet. Und was mich auch stört an äh ihrem Zungenschlag ist immer diese Konzentration auf Griechenland. Äh. Und auch von ihnen Herr Blome als failed state. Es gibt eben die Eurozone und die Eurozone hat insgesamt ein Problem. Denn die Probleme die es in Griechenland grade gibt, die spiegeln sich ähnlich in Portugal, Spanien, Irland auch in Frankreich bröckelt es langsam weg. Weil die Eurozone insgesamt eine Fehlkonstruktion ist. Und genau über diese strukturellen Gründe muss man sprechen. Wenn in einem, in einer Währungszone unt unterschiedliche Ökonomien sind, mit unterschiedlicher Stärke, dann werden die Ungleichgewichte immer größer. Dann bricht es irgendwann auseinander. Und die Politik der Bundesregierung und auch der Troika mit ihrer Rezessionspolitik, ihrer Kürzungspolitik in Griechenland äh Portugal, Irland dazu führt, dass diese Ungleichgewicht immer größer werden. Die wird den Euro letztendlich ähm an die Wand fahren. ((Applaus)) \#00:56:33-3\# 
	\end{linenumbers}
	\captionof{transkript}{hart aber fair (18.06.2012)}
	\label{lis:37}
\end{description}

\section{Codebuch zur quantitativen Inhaltsanalyse}\label{anhang:codebuch}

\subsection{Definitionen}

\begin{itemize}
	\item \textbf{Sendung} \\
	Bezeichnet eine Fernsehsendung inklusive aller ihrer einzelnen Folgen, also beispielsweise hart aber fair.
	\item \textbf{Folge} \\
	Bezeichnet eine einzelne Ausgabe innerhalb einer Sendung, also beispielsweise die Folge von hart aber fair vom 03. Dezember 2012.
\end{itemize}

\subsection{Formale Kategorien}

(Codiereinheit: Folge | Kontexteinheit: Folge)

\begin{itemize}
	\item \textbf{ID} \\
	Fortlaufende Nummer der codierten Talkfolge.
	\item \textbf{Datum} \\
	Datum der Erstausstrahlung der jeweiligen Folge. Format: TT.MM.JJJJ
\end{itemize}

\subsection{Inhaltliche Kategorien}

(Codiereinheit: Folge | Kontexteinheit: Folge)

\subsubsection{Themenbereich}

In dieser Variable wurde der Themenbereich der Folge codiert. Die Angabe wurde dabei aus dem Titel der jeweiligen Folge und den Begleitinformationen auf der Sendungshomepage entnommen.
	
\begin{list}{}{}
	\item \textbf{Mögliche Ausprägungen}
	\begin{description}
		\item[10] Politik
		\begin{description}
			\item[11] Innenpolitik
			\item[12] Außenpolitik
			\item[13] Umwelt- / Energiepolitik
			\item[14] Sozialpolitik
			\item[15] Bildungspolitik
			\item[19] Sonstige Politik
		\end{description}
		\item[20] Wirtschaft
		\item[30] Gesundheit
		\item[40] Sport
		\item[50] Prominenz / Personality
		\item[60] Service
		\item[70] Religion
		\item[80] Kriminalität / Katastrophen
		\item[90] Sonstige
	\end{description}
\end{list}

\subsubsection{Themencluster}

Hier wurde das übergeordnete Thema der jeweiligen Folge einer Talkshow mittels eines Schlagworts näher bestimmt, um so Themencluster bestimmen zu können.

\subsubsection{Gast}

Name des jew
eiligen Gastes so wie er auf der Homepage der Sendung bzw. in den Bauchbinden der jeweiligen Folge genannt wurde. Dabei war es egal, ob der Gast die komplette Sendung über anwesend war oder nicht.

\subsubsection{Rolle}

Gibt die gesellschaftliche Rolle des Gastes an, in der dieser in die jeweilige Talkfolge eingeladen wurde. Die Codierung wurde dabei primär auf Grundlage der auf der Sendungshomepage verfügbaren Angaben vorgenommen, war diese Information nicht verfügbar wurde auf die Vorstellung innerhalb der jeweiligen Folge zurückgegriffen. Bei mehreren genannten Rollen bzw. Funktionen, also beispielsweise „Finanzexperte und Autor“ wurde die Kategorie codiert, die im Informationstext stärker betont oder zuerst genannt wurde. Einzige Ausnahme sind Politiker, diese wurden immer als Politiker codiert auch wenn weitere Rollen bzw. Funktionen genannt waren.

\begin{list}{}{}
	\item \textbf{Mögliche Ausprägungen}
	\begin{description}
		\item[10] Politik
		\begin{description}
			\item[11] EU-Politik
			\item[12] Bundespolitik
			\item[13] Landespolitik
			\item[14] Kommunalpolitik
			\item[15] Ausländische Politik
			\item[16] Behördenvertreter
			\item[19] Sonstige Politik
		\end{description}
		\item[20] Wirtschaft
		\begin{description}
			\item[21] Unternehmer
			\item[22] Wirtschaftsverband
			\item[23] Gewerkschafter
			\item[24] Arbeitnehmer
		\end{description}
		\item[30] Journalist
		\item[40] Expert
		\begin{description}
			\item[41] Wissenschaftler
			\item[42] Rechtsanwalt / Rechtsanwältin / Jurist
			\item[43] Ärtzin / Arzt
			\item[49] Sonstige Expert
		\end{description}
		\item[50] Kultur / Sport
		\begin{description}
			\item[51] Schauspieler / Moderator
			\item[52] Künstler
			\item[53] Sportler
			\item[54] Sonstige Prominente
			\item[55] Autor / Publizist
		\end{description}
		\item[60] Zivilgesellschaft
		\begin{description}
			\item[61] Sozialverbände
			\item[62] NGOs / Vereine / Soziale Bewegungen
			\item[69] Sonstige Zivilgesellschaft
		\end{description}
		\item[70] Religion
		\item[80] „Normalerbürger“ / Betroffene Person
		\item[99] Sonstige
	\end{description}
\end{list}

\subsubsection{Partei}

Hier wurde die Parteizugehörigkeit des Gastes codiert. Wenn es sich nicht um Politiker handelt wurde diese nur dann erfasst, wenn sich eine entsprechende Angabe auf der Sendungshomepage bzw. in der Folge selbst fand. Angehörige der Jugendorganisationen der Parteien wurden der Mutterpartei zugeordnet.

\begin{list}{}{}
	\item \textbf{Mögliche Ausprägungen}
	\begin{description}
		\item[1] CDU / CSU
		\item[2] SPD
		\item[3] FDP
		\item[4] Bündnis 90 / Die Grünen
		\item[5] Die Linke
		\item[6] Piratenpartei
		\item[7] Sonstige Parteien
	\end{description}
\end{list}

\subsubsection{Geschlecht}

Gibt das Geschlecht des Gastes an.

\begin{list}{}{}
	\item \textbf{Mögliche Ausprägungen}
	\begin{description}
		\item[1] weiblich
		\item[2] männlich
	\end{description}
\end{list}

\subsubsection{Alter}

Das Alter der Gäste wurde entweder durch Angaben auf der Sendungshomepage oder eine kurze Internetrecherche versucht zu ermittelt. Der Einfachheit und Praktikabilität halber wurde das Alter dabei über das Geburtsjahr und nicht das genaue Geburtsdatum ermittelt, sodass sich geringe Abweichungen zum tatsächlichen Alter der Gäste zum jeweiligen Sendedatum ergeben können.