\chapter{Einleitung}

Viel ist schon über Talkshows geschrieben worden und oft wurde das Genre bereits totgesagt – dennoch sind sie aus der deutschen Fernsehlandschaft weiterhin nicht wegzudenken. Zumindest das was landläufig als politische Talkshow verstanden wird, erfreut sich fortgesetzt großer Beliebtheit. So weitet die ARD im letzten Jahr erst ihre abendliche Talkschiene aus. Seit dem 11. September 2011 strahlt so alleine Das Erste von Sonntag bis Donnerstag fünf abendliche Polittalks\footnote{Konkret sind dies \textit{Günther Jauch}, \textit{Anne Will}, \textit{hart aber fair}, \textit{Menschen bei Maischberger} und \textit{Beckmann}.} aus. Hinzu kommt eine Sendung im ZDF\footnote{Gemeint ist \textit{maybrit illner}. Bei \textit{Markus Lanz}handelt es sich um eine Personality-Show und nicht um einen Polittalk.} und diverse weitere bei den privaten Sendern und den Dritten Programmen. Und obwohl diese Masse zur „Inflation“ \parencite[116]{gaeblerUndUnserenTaeglichen2011} geführt habe und „das einzelne Exemplar dadurch entwertet“ \parencite[116]{gaeblerUndUnserenTaeglichen2011} worden sei, kommen weitere Polittalks auf den Markt. So hatte am 11. November 2012 \textit{Absolute Mehrheit} bei ProSieben Premiere\footnote{\textit{Absolute Mehrheit} soll eine „frische politische Talkshow“ sein und „will vor allem junge Zuschauer für die politische Diskussion begeistern“ \parencite{prosiebenShowAbsoluteMehrheito.J.}. Dazu sollen fünf Gäste 75 Minuten miteinander über drei vorgegebene Themen diskutieren – auf den ersten Blick wird hier altbewährten Konzepten gefolgt. Hauptunterschied zu den etablierten politischen Talkshows ist, neben dem Moderator, der Umstand, dass am Ende der Sendung die Zuschauer über die Gäste abstimmen und demjenigen der, über 50 \% der Stimmen bekommt, ein Preisgeld von 100.000 € winkt. Offenkundig wurde sich hier an den verbreiteten Game- und Quizshows orientiert. Raab steht als Ko-Moderator Peter Limbourg, Anchorman der Sat.1 Nachrichten, zur Seite \parencite{prosiebenShowAbsoluteMehrheito.J.}.}, die mit Stefan Raab als Moderator ein eher ungewohntes Gesicht für einen Polittalk bietet\footnote{Bereits vor der Premiere hatte \textit{Absolute Mehrheit} einen ersten Aufreger: Angeblich wurde auf Druck von Bundesumweltminister Peter Altmaier der Grüne Volker Becker wieder ausgeladen. Altmaier dementierte und sagte selbst ab, das Thema kam in die Medien und vielleicht liegt hierin auch die wirkliche Erklärung für diesen „Skandal“.}.

Während die Ära der sogenannten Daily Talks tatsächlich an ihrem Ende angelangt zu sein scheint und sich mit Britt aktuell nur noch ein Vertreter dieser Art im deutschen Fernsehen findet \parencite{niggemeierEndeDailyTalkAra2009}, deuten allein schon die beiden oben beschriebenen Entwicklungen auf eine weiterhin gegebene Relevanz dieser Form der Politikvermittlung und -darstellung hin.

Dass nicht nur Programmplaner, sondern auch die breitere Öffentlichkeit diese Relevanz sieht, machen zwei Vorfälle beispielhaft deutlich: Zum einen die Störung der \textit{Günther Jauch} Sendung am 6. Mai 2012 durch einen Studenten der Ernst-Busch-Schauspielschule. Dieser hatte durch einen lautstarken Zwischenruf gegen Ende der Sendung auf die geplante Schließung selbiger hingewiesen \parencite{brucknerStudiogastSorgtFuer2012}. Zum anderen die Proteste gegen den Auftritt von Thilo Sarrazin bei \textit{Günter Jauch} am 20. Mai 2012. Dessen neustes Buch „Europa braucht den Euro nicht“ stand zu diesem Zeitpunkt kurz vor der Veröffentlichung und die darin enthaltenen kontroversen Thesen veranlassten diverse Politiker dazu die Einladung Sarrazins in die Talkshows scharf zu kritisieren. Renate Künast (Grünen-Fraktionsvorsitzende) etwa äußerte: „Nationalistischer Unsinn von Sarrazin passt nicht zum Bildungsauftrag eines öffentlich-rechtlichen Senders“, Patrick Döring (FDP-Generalsekretär) meinte, dies habe „im öffentlich-rechtlichen Fernsehen nichts zu suchen“, und Reinhold Robbe (SPD) sprach davon, dass sich mit Sarrazin „niemand mehr in eine Talkshow setzen“ sollte \cite{hansRittAufEmpoerungswelle2012}. Darüber hinaus fand während der Sendung vor dem Studio eine Protestkundgebung statt. Beide Beispiele machen deutlich, dass Polittalks ein hoher Einfluss auf den politischen Diskurs der Bundesrepublik unterstellt wird. Wäre dies nicht der Fall, so hätten nicht führende Politiker gegen den Auftritt Sarrazins derart vehement protestiert und auch die protestierenden Studenten hätten diese Plattform nicht genutzt. Auch andere Medien schreiben zumindest einigen der Polittalks eine immerhin so hohe Relevanz zu, dass sie regelmäßig Kritiken einzelner Folgen veröffentlichen.

Gleichzeitig kommen Studien allerdings zu wenig schmeichelhaften Ergebnissen. Bereits 2006 stellte eine Untersuchung von \textit{Sabine Christiansen} fest, die Sendung sei vor allem eine „Schaubühne der Einflussreichen und Meinungsmacher [\ldots] die sich für eine neoliberal geprägte Reform des Sozialstaats einsetzen“ \parencite[17]{muellerSchaubuehneFuerEinflussreichen2006}. In der aktuellsten vorliegenden Studie spricht \textcite{gaeblerUndUnserenTaeglichen2011}  davon, dass „die Erfüllung der selbst gesetzten Dramaturgie” \parencite[112]{gaeblerUndUnserenTaeglichen2011} die journalistische Aufklärung verdrängt habe, es werde psychologisiert, statt rationalisiert, Themen und Akteure seien sehr homogen, Neues komme nicht vor.

Die mediale Kritik ist ebenfalls negativ. In der Debatte um die Einführung der ARD-Talkschiene etwa wurde „die ewig repetitive Gästeauswahl, die jeden Erkenntnisgewinn zuverlässig verhindert“ \parencite{harnaschUeberfluessigeARDTalkshows2012}, bemängelt und empfohlen: „Abschaffen“ \parencite{harnaschUeberfluessigeARDTalkshows2012}. Aufgrund der Vielzahl der Talks wüsste man nicht mehr, welcher wirklich wichtig sei, zumal die Qualität von Folge zu Folge sehr unterschiedlich sei \parencite[vgl.][]{barfussPolitikTalkshowsARD2012}. \textcite{doernerFuenfPolitischeTalkshows2012} meint „eher die geringen Produktionskosten der Formate als der politische Mehrwert [hätten] die Programmverantwortlichen zu dieser Offensive motiviert“. Andere befürchten, dass die Ausweitung der ARD-Talks die Sendeplätze für politische Dokumentationen noch knapper machen könnte \parencite{o.a.TalkGegenDokus2012}. Für wieder andere ist die Vielzahl an Polittalks nur ein Symptom für „die Schwäche des Nachrichtenjournalismus“ \parencite{schloemannUeberschussrechnung2012} in Deutschland.

Auch innerhalb der ARD stößt die ARD-Talkschiene nicht auf ungeteilte Gegenliebe. So wurde die Reihe sowohl vom ARD-Programmbeirat \parencite*{ard-programmbeiratTalkformateImErsten2012}, als auch vom WDR-Rundfunkrat \parencite*{wdr-rundfunkratStellungnahmeProgrammausschussesFuer2012} und dem NRD-Programmausschuss \parencite{harbuschGeheimPapierARDKritisiert2012} scharf kritisiert.

Vereinzelt wird außerdem von Seiten der Politik Kritik vorgebracht. Insbesondere Bundestagspräsident Nobert Lammert äußert sich regelmäßig derart und bemängelt etwa, dass es bei Talkshows „vor allem um Unterhaltung und weniger um Information“ \parencite{kammholzLammertPolitikerSind2012} gehe, das ernsthafte Gespräche werde der „Entertainisierung von allem und jedem“ geopfert \parencites{brauckMichDuerfteEs2011}[vgl. auch][]{gaeblerInterviewMitDr2011}.

Doch es finden sich auch positive Stimmen zu einzelnen politischen Talkshows. \textcite{harnaschUeberfluessigeARDTalkshows2012} lobt etwa „Benjamin von Stuckrad-Barres sensationelle[n] Polit-Talk“ bei ZDF Neo. \textcite{weisTalkshowFlutImErsten2012} hält Plasberg für den „vielleicht aktuell beste[n] Polit-Talker“. Programmverantwortliche verteidigen – wenig überraschend – ebenso die Sendungen. Der ARD-Programmdirektor Herres beispielsweise entgegnet den Kritikern: „Die Talking Heads des Ersten sind kompetent und populär. Das sind erstklassige Journalistinnen und Journalisten“ \parencite{o.a.EindrucksvolleIntensitaet2012}. Auch dem Format Talkshow an sich wird weiterhin ein positives Potenzial zugesprochen. Exemplarisch äußert dies etwa \textcite{doernerFuenfPolitischeTalkshows2012}:

\begin{quote}
	[$\ldots$] Talkshows könnten in der modernen Mediendemokratie ein wichtiges Moment der Veranschaulichung und Verlebendigung politischer Positionen und Prozesse sein. Politische Gesprächssendungen sind daher grundsätzlich als ein relevantes und legitimes Mittel demokratischer Kommunikation zu betrachten. Talksendungen markieren durchaus eine Chance zur Inklusion politikferner Bürger in den politischen Diskurs einer zunehmend unterhaltungsorientierten Medienkultur.
\end{quote}

\section{Forschungsstand}

Führt man sich die ungebrochen große Popularität von politischen Talkshows und die anhaltende öffentliche Debatte über sie vor Augen, so ist es verwunderlich, dass es im wissenschaftlichen Bereich recht still um das Genre geworden ist. Die jüngste Untersuchung stammt von \textcite{gaeblerUndUnserenTaeglichen2011}, daneben stehen in den letzten Jahren bloß einige publizierte Diplom- bzw. Magisterarbeiten \parencite{amelnQualitaetPolitischerTalkshows2010, koenningAllesBlossGeschwaetz2009, bockPolitainmentImDeutschen2009, schmidtQualitaetPolitischerTalkshows2007, eisentrautPolitTalkAlsForm2007}, sowie die Erfahrungsberichte ehemaliger oder aktueller Akteure \parencite{michelPolitTalkshowsBuehnenMacht2009} und die umfassende Darstellung der deutschen Talkgeschichte von \textcite{kellerGeschichteTalkshowDeutschland2009}.

Dennoch ist die vorliegende Forschung zu politischen Talkshows in ihrem Umfang schwer fassbar. Die Ansätze reichen von Untersuchungen einzelner Sendungen \parencite{muellerSchaubuehneFuerEinflussreichen2006, rossumMeineSonntageMit2004, tenscherSabineChristiansenUnd1999, nielandTalkshowisierungWahlkampfesAnalyse2002} oder spezifischer Aspekte \textcite{schultzModerationPolitischerGesprachsrunden2004, schultzJournalistenTalkPolitischeKommunikation2002, doernerPolitainmentPolitikMedialen2001, schrottElefantenUnterSich1996} bis zu umfassenderen Betrachtungen des Genres  \textcite{kellerGeschichteTalkshowDeutschland2009, plakeTalkshowsIndustrialisierungKommunikation1999, hollyPolitischeFernsehdiskussionenZur1986, mahloZurDiskussionUm1956, gaeblerUndUnserenTaeglichen2011, barloewenTalkShow1975}.

\subsection{Ergebnisse}

Der Tenor der wichtigsten Forschungsergebnisse ist zwiespältig. Zum einen kommen viele Untersuchungen, die sich vornehmlich konkret einzelnen Sendungen widmen, zu dem Ergebnis, dass politische Gesprächssendungen im Fernsehen vor allem Plattformen für Parteien- und Politikerwerbung seien \parencites[199\psq]{hollyPolitischeFernsehdiskussionenZur1986}{schrottElefantenUnterSich1996}. Auch hinsichtlich anderer Gästegruppen wurde früh dargestellt, dass diesen der Auftritt in einer Talkshow vor allem zur Selbstdarstellung diene \parencite[162]{bayerTalkShowInszenierte1975}. Bucher weist darauf hin, dass die Medien dabei aktiv mitwirkten und erkennt etwa in den jüngeren TV-Duellen „eine neuartige Variante der Symbiosetheorie“ \parencite[302]{bucherMedienrealitaetPolitischenZur2004}.

Mediale und institutionelle Rahmenbedingungen sorgten zudem dafür, dass die Sendungen stets nach festem Schema abliefen. Der Moderator sei „vor allem Wahrer von (anstaltsgerechtem) Proporz und Stimulator für Unterhaltsames. Argumentation und Rationalität, Kriterien für eine gute Diskussion, überhaupt textsortenbezogene Konsistenz von Gesprächen sind demgegenüber sekundär“ \parencites[202]{hollyPolitischeFernsehdiskussionenZur1986}[vgl. auch][32\psq]{barloewenGrosseVorbildFernsehproduktion1975}[390\psqq]{nielandTalkshowisierungWahlkampfesAnalyse2002}. Die Moderatoren werden dafür kritisiert, dass sie „ihren Gästen teilweise wenig entgegen“ \parencite[314]{schultzModerationPolitischerGesprachsrunden2004} setzten und wenig dafür täten eine gehaltvolle politische Debatte anzuregen \parencite[151, passim]{tenscherShowdownImFernsehen1998}. Gäste und Themen seien voraussehbar \parencite[390\psqq]{nielandTalkshowisierungWahlkampfesAnalyse2002} und die Diskursqualität auch eher mäßig \parencite[226]{schichaInszenierungPolitischerDiskurse2002}.

Auch die visuelle Inszenierung unterstütze den Eindruck einer simulierten Diskussion zusätzlich. Letztlich ließen sich die Sendungen viel treffender als „Propaganda-Talk-Show“ \parencite[204]{hollyPolitischeFernsehdiskussionenZur1986} beschreiben. In diesem Sinne attestiert \textcite[13]{rossumMeineSonntageMit2004} Sabine Christiansen „eine unschlagbare journalistische Unbedarftheit“ und machte dies mit dafür verantwortlich, dass die Sendung vor allem eine einseitige neoliberale Botschaft vermittelt habe, eine Diagnose der sich auch \textcite{muellerSchaubuehneFuerEinflussreichen2006} anschließen.

Von einer „Talkshowisierung der Politik“ könne zwar nur teilweise die Rede sein \parencite[389]{nielandTalkshowisierungWahlkampfesAnalyse2002}, den politischen Talkshows gelinge es dennoch „als Faktor und Motor in aktuellen politischen und gesellschaftlichen Diskursen zu wirken“ \parencite[390]{nielandTalkshowisierungWahlkampfesAnalyse2002}.

\textcite{plakeTalkshowsIndustrialisierungKommunikation1999} spricht unter Bezugnahme auf die Kulturindustrietheorie von Adorno und Horkheimer von einer „Industrialisierung der Kommunikation“. Die Talkshows könnten nichts zur Klärung allgemeiner, gesellschaftlicher Fragen beitragen, stattdessen fragmentierten sie das Bewusstsein ihrer Zuschauer bloß noch mehr \textcite[144, passim]{plakeTalkshowsIndustrialisierungKommunikation1999}. Das Problem bestehe unter anderem „in falschen journalistischen Etiketten und den nicht eingehaltenen Versprechungen von plebiszitärer Demokratie und neuer Öffentlichkeit“ \textcite[11]{plakeTalkshowsIndustrialisierungKommunikation1999}.

Andererseits sprechen einigen Autoren Talkshows, meist auf einer allgemeinen Ebene, ein positives Potenzial zu. So sind für \textcite{doernerPolitainmentPolitikMedialen2001} Talkshows eines der Paradebeispiele für das von ihm diagnostizierte Phänomen des „Politainment“, gemeint ist die Entwicklung einer „immer enger werdenden symbiotischen Beziehung zwischen Medienunterhaltung und Politik“ \parencite[11]{doernerPolitainmentPolitikMedialen2001}. Bei Talkshows werde den Gäste Medienpräsenz zuteil, den Sendern hohe Einschaltquoten \parencite[135]{doernerPolitainmentPolitikMedialen2001}. Auch wenn es um die Deliberativität der Debatten schlecht bestellt sei und Politik personalisiert und vereinfacht würde, so leisteten sie doch eine wichtige Orientierungsleistung in dem sie die politischen Akteure und deren Interessengegensätze sichtbar machten \parencite[139\psq, 239\psq]{doernerPolitainmentPolitikMedialen2001} und damit „den Raum des legitimen Diskurses“ (Ebd., 142) absteckten. Deswegen möchte er seine Diagnose auch nicht negativ verstanden wissen \parencite[243\psq]{doernerPolitainmentPolitikMedialen2001}.

\textcite{bockPolitainmentImDeutschen2009} stimmt dem zu und meint ebenfalls, dass die politischen Talkshows ein „Prototyp des Politainment“ \parencite[52]{bockPolitainmentImDeutschen2009} seien, bei „denen sich der Grad der Inszenierung im Ausmaß an Selbstdarstellungshandeln, Personalisierung und Emotionalisierung beobachten“ ließe \parencite[52]{bockPolitainmentImDeutschen2009}. Dennoch sei eine vollständige Ablehnung fehl am Platz, denn die Talks würden Menschen dazu bringen, sich überhaupt erst mit Politik zu beschäftigen, zudem seien sich diese des Inszenierungscharakters bewusst \parencite[51\psqq]{bockPolitainmentImDeutschen2009}.

Das eher praxisorientierte Buch „Das journalistische Interview“ spricht dem Interviewer in der Talkshow die Möglichkeit zu, „kritische Fragen zu stellen, ferner, durch den Wechsel von sachlicher und persönlicher Ebene Informationen zu erhalten, die in stärker formellen Sendungen [\ldots] nicht möglich wären“ \parencite[269]{friedrichsJournalistischeInterview2009}. Entsprechend sollte der Moderator versuchen zur Privatperson hinter dem Amtsinhaber vorzudringen, hierfür zentral sei nicht nur die richtige Fragestrategie, sondern auch eine möglichst ungezwungene Studiogestaltung \parencite[276]{friedrichsJournalistischeInterview2009}.

Unterbelichtet bleibt hingegen die Rolle des Zuschauers, sowohl im Hinblick auf die Rezeption der Sendungen als auch hinsichtlich der aktiven Partizipation an diesen. Laut \textcite[174\psqq]{reinemannMedienmacherAlsMediennutzer2003} sahen 2000 63 \% der befragten Journalisten regelmäßig eine politische Talkshow. Spitzenreiter war \textit{Sabine Christiansen} mit einem Anteil von 52 \%. Politische Magazine wurden hingegen weit seltener rezipiert. Eine frühere Untersuchung widmet sich der Demographie der Zuschauer und stellt unter anderem fest, dass bei politischen Talkshows das Durchschnittsalter der Zuschauer relativ hoch ist, genauso wie der Frauenanteil \parencite{gerhardsTalkshownutzungUndTalkshownutzer2002}. \textcite{hollyFernsehkommunikationUndAnschlusskommunikation2002} meint Talkshows würden sich besonders für die Anschlusskommunikation der Zuschauer untereinander eigenen, die „Deutungen, Aneignungen und Vergnügen in spielerischer Weise miteinander verknüpfen“ \textcite[366]{hollyFernsehkommunikationUndAnschlusskommunikation2002} könnten.

Hinsichtlich des zweiten Bereichs kommt eine Studie aus dem Jahr 1989 zu dem Ergebnis, dass der „wichtigste – wenn auch nicht unbedingt erfolgreichste – Versuch, politische Diskussionen zu entritualisieren, [$\ldots$] die Beteiligung des Rezipienten“ \parencite[140]{burgerDiskussionOhnRitual1989} sei, wobei die „unmittelbarsten Formen der Beteiligung (Politiker und Rezipienten am gleichen Tisch) [$\ldots$] am seltensten genutzt“ \parencite[140]{burgerDiskussionOhnRitual1989} würden.

\subsubsection{Interviewforschung}

Hinsichtlich politischer Fernsehinterviews hält \textcite[143–152]{hoffmannPolitischeFernsehinterviewsEmpirische1982} fest, dass diese im Untersuchungszeitraum von Vertretern der Parteien und der Exekutive dominiert wurden, Frauen und Vertreter der Zivilgesellschaft, der Arbeitnehmer und von Minderheiten oder auch „Normalbürger“ hingegen kaum interviewt wurden. Ähnliches zeige sich bei den Themen, wo Innen-, Wirtschafts- und Außenpolitik vorherrschten, wobei letztere im Vergleich zu ihrer Bedeutung unterrepräsentiert sei. Der Journalist sei vor allem Stichwortgeber und weniger Stellvertreter des Zuschauers gewesen und „die grundlegenden Interessenkonstellationen und Ursachen, mögliche Alternativen, die Relevanz für die Mehrheit der Rezipienten sowie die historische Dimension des politischen Handelns“ \textcite[148]{hoffmannPolitischeFernsehinterviewsEmpirische1982} seien kaum zur Sprache gekommen. Es scheint also, dass die für politische Talkshows diagnostizierte Probleme auch auf politische Interviews im Fernsehen allgemein zu treffen.

\subsection{\ldots und Defizite}

Neben dem bereits angesprochenen aktuellem wissenschaftlichen Desinteresse am Gegenstand, fallen beim obigen Blick auf den Forschungsstand mehrere Defizite auf: Erstens gibt es kaum international vergleichende Forschung die politische Talkshows in verschiedenen Ländern analysiert\footnote{Eine der wenigen Ausnahmen auf diesem Gebiet ist der Vergleich amerikanischer und deutscher Talkshows von \textcite{krauseLocalTalkGlobal1995} sowie die frühen Analysen deutscher Talkshows, die noch stark vom Blick auf die USA geprägt waren \parencite{barloewenGrosseVorbildFernsehproduktion1975, barloewenGesellschaftlicheKontextTalk1975}.}. Wenn überhaupt wird in der deutschsprachigen Forschung auf die USA eingegangen und dann auch fast ausschließlich im Rahmen kurzer historischer Abrisse. Gleiches gilt für die Berücksichtigung nicht-deutschsprachiger Forschungsergebnisse.

Ebenfalls Mangelware sind historisch vergleichende Untersuchungen. Positiv sind hier das Buch „Die Geschichte der Talkshow in Deutschland“ von \textcite{kellerGeschichteTalkshowDeutschland2009}, sowie einzelne weitere Aufsätze bzw. Bücher \parencite{foltinZurEntwicklungTalkshow1990, foltinTalkshowGeschichteSchillernden1994, timbergTelevisionTalkHistory2002} hervorzuheben.

Drittens beschränken sich die meisten Untersuchungen nur auf ein oder zwei „Vorzeigetalks“ beziehungsweise auf „Ausnahmesendungen“. Das sind die jeweils in der öffent"-lich"-en Wahrnehmung oder hinsichtlich ihrer Einschaltquoten dominierenden Sendungen, wie beispielsweise \textit{Sabine Christiansen} oder \textit{Talk im Turm} \parencite{tenscherSabineChristiansenUnd1999}, genauso wie Kanzlerduelle \parencite{lippJournalistischeWahlkampfvermittlungAnalyse1983, bucherLogikPolitikLogik2007, schrottElefantenUnterSich1996}. Das alltägliche Talkgeschehen in den zahlreichen anderen Polittalks wird hingegen nicht weiter beachtet.

Weiterhin bedenklich ist die Dominanz quantitativer Inhaltsanalysen in jüngeren Untersuchungen\footnote{Folgende Untersuchungen basieren wesentlich auf quantitativen Inhaltsanalysen: \cite{schrottElefantenUnterSich1996, schultzJournalistenTalkPolitischeKommunikation2002, schultzModerationPolitischerGesprachsrunden2004, bockPolitainmentImDeutschen2009}.}. Diese Methode erscheint für die Untersuchung eines durch dynamische Kommunikationssituationen charakterisierten Gegenstandes nur bedingt geeignet, ist sie doch nicht in der Lage „das subtile Wechselspiel zwischen den Teilnehmern [\ldots] zu erfassen“ \parencite[104]{plakeTalkshowsIndustrialisierungKommunikation1999}\footnote{Diesem Umstand sind sich einige Autoren durchaus bewusst. \parencite[305]{schultzModerationPolitischerGesprachsrunden2004} etwa schreibt, dass es ohne den kommunikativen Kontext nicht ersichtlich sei, ob eine Unterbrechung durch den Moderator adäquat oder inadäquat ist.}. Hinzu kommt, dass die untersuchten Korpora meist relativ klein sind und oftmals nur eine Handvoll untersuchter Folgen umfassen, was wiederum Fragen der Repräsentativität aufwirft.

Methodisch anders gelagerte Analysen finden sich primär in Forschungen älteren Datums, darunter wiederum vor allem sprachwissenschaftliche die sich zum Teil nicht mit politischen Talkshows sondern (Nachrichten-) Interviews befassen \parencite{roettgerInterviewstileUndNeue1996, heritageAnalyzingNewsInterviews1989, hoffmannPolitischeFernsehinterviewsEmpirische1982, brinkerThematischeMusterUnd1988, hollyPolitischeFernsehdiskussionenZur1986}. Außerdem gibt es eine Reihe weiterer Publikationen, die meist eher essayistische Beurteilungen \parencite{rossumMeineSonntageMit2004, bommariusUebersaettigungsbeilageExpertenUnd2005, brunstJeSpaeterAbend2005} und anekdotische Erzählungen ehemaliger oder aktueller Akteure \parencite{michelPolitTalkshowsBuehnenMacht2009} sind, denn klassische wissenschaftliche Untersuchungen.

Fünftens und letztens ignoriert die Rezeptionsforschung politische Talkshows fast komplett \parencite[16]{schichaInszenierungPolitischerDiskurse2002}. Kaum etwas ist über die Rezeption und vor allem Wirkung von Polittalks bekannt, obwohl ihnen, wie in der Einleitung gezeigt, eine enorme Wirkung auf die öffentliche Meinung und Debatte unterstellt wird.

\section{Fragestellung}

Diese Arbeit soll bei der Schließung einiger dieser Forschungslücken helfen und zugleich einen Beitrag zur inzwischen wieder abgeflauten Debatte über die ARD-Talkschiene leisten. Es soll das sonst vernachlässigte Talkgeschehen ebenso in den Blick genommen werden, wie die jeweiligen Talkflaggschiffe. Gewissermaßen geht es um eine (teilweise) Vermessung des deutschen Talkgeschehens und seiner Rolle bei der Verarbeitung gesellschaftlich relevanter Themen.

Viel wurde darübergeschrieben, ob Unterhaltung und Information zusammenpassen, ob Talks Politainment sind und ob das gut oder schlecht ist, aber wenig Wert wurde darauf gelegt welche Positionen und politischen Interessen eigentlich in den Polittalks zu Wort kommen. Entsprechend soll es hier nicht um die Rationalität der Diskussionen, den Grad der Inszenierung oder Personalisierung gehen. Wie der vorstehende Forschungsüberblick gezeigt hat, liegen dazu mehr als genug Untersuchungen mit eindeutigen Ergebnissen vor.

Die hier zentralen Fragen sind hingegen, inwiefern die Ausweitung der ARD-Talkschiene tatsächlich zu einer Themen- und Gästeinflation geführt hat? Wie präsentieren und bearbeiten die Talkshows gesellschaftlich relevante Themen und können sie dabei nicht nur ihren selbstgesteckten Ansprüchen, sondern auch denen einer demokratischen Politikvermittlung und -repräsentation gerecht werden? Kann also letztlich die „Talkshow im Fernsehen als Stellvertreter-Medium für das demokratische Gespräch“ \parencite{hieberTalkshowMarathonARDNach2011} dienen oder reproduziert sie doch nur gesellschaftliche Ungleichheiten?

Um diese Fragen zu beantworten, ist es erforderlich nicht nur quantitativ die Gäste- und Themenstruktur der ARD-Talkschiene zu untersuchen, sondern zum einen auch Talkshows jenseits dieses Quintetts einzubeziehen und zum anderen die quantitative Inhaltsanalyse mit einer qualitativen Methode zu ergänzen.

Im Konkreten heißt dies, erstens, dass zusätzlich zu den fünf ARD-Talks – \textit{Anne Will}, \textit{Beckmann}, \textit{Günther Jauch}, \textit{hart aber fair} und \textit{Menschen bei Maischberger} – noch das Talkflaggschiff des ZDF, \textit{maybrit illner}, sowie \textit{log in} in die Analyse einbezogen wurden. Diese sieben Sendungen repräsentieren zum einen die reichweitenstärksten Polittalks in Deutschland, zum anderen aber mit log in auch ein junges, sich selbst als innovativ verstehendes Format. Zweitens wurden jeweils zwei Ausgaben der sieben Sendungen zusätzlich einer qualitativen Analyse unterzogen. Zurückgegriffen wurde hierzu auf Methoden der Gesprächs- und Diskursanalyse. Auf diesem Weg wurde das bereits beschriebene Problem umgangen, dass rein quantitative Untersuchungen für die Untersuchung dynamischer Kommunikationszusammenhänge nicht ausreichen.

Das gesellschaftliche relevante Thema, dessen Darstellung in den Polittalks dabei in den Blick genommen werden soll, ist die sogenannte Staatsschuldenkrise im Euroraum oder auch einfach nur kurz die Eurokrise.

Im folgenden Kapitel wird die verwendete Methodik näher vorgestellt. Dort wird zudem der Gegenstandsbereich näher definiert und charakterisiert sowie der untersuchte Korpus vorgestellt. Im dritten Kapitel folgt eine kurze Darstellung der Geschichte der politischen Talkshow und eine Vorstellung der untersuchten Sendungen. Die Kapitel vier und fünf widmen sich dann den Ergebnissen der quantitativen bzw. qualitativen Analyse.

In der Arbeit wurden die Namen von Fernsehsendungen \textit{kursiv} gesetzt, um so eine Unterscheidung zu den oftmals gleichlautenden Namen der Moderatoren zu ermöglichen. Zitate wurden behutsam an die neue deutsche Rechtschreibung angepasst.