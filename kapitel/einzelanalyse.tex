% !TeX root = ../PolitischeTalkshows.tex

\chapter{Einzelanalyse – Die Eurokrise im Polittalk}

Ob und inwiefern sich solche Verstöße nicht nur auf der Makroebene der Gäste- und Themenstruktur, sondern auch auf der Mikroebene einzelner Talkfolgen finden lassen, wird in der folgenden qualitativen Analyse untersucht. Hierzu werden zuerst die untersuchten Folgen kurz tabellarisch vorgestellt, danach folgt ein Blick auf ausgewählte Sendungselemente – Eröffnung und Abschluss, Serviceelemente, Moderation, Einspieler und Umgang mit verschiedenen Akteursgruppen. Es folgt dann ein Blick auf die in den Diskussionen vorkommenden Topoi hinsichtlich Ursache und Lösung der Eurokrise sowie bestimmter Schlüsselbegriffe.

\section{Die untersuchten Folgen}

Bereits aus der näheren Betrachtung der untersuchten Folgen wird eine bei allen Sendungen vorhandene Grundkonstellation sichtbar (vgl. Tabelle \vref{tab:korpus-einzelanalyse}). Die Gäste lassen sich grob in zwei Lager einteilen – Befürworter der gegenwärtigen Eurorettungspolitik und Kritiker selbiger. Hinzu kommt bei einigen Folgen ein ausgesprochener Gegner des Euro an sich.

In fast allen Folgen tritt mindestens ein Mitglied der Regierungsparteien als Verteidiger der „Eurorettung“ an, ein Oppositionspolitiker ist hingegen nicht immer eingeladen (vgl. auch Kapitel \vref{chap:parteizugehoerigkeit}). Politiker der Grünen oder der Piratenpartei kommen nicht vor. Einzig bei \textit{log in} ist eine Vertreterin der Grünen eingeladen, darf aber nur zeitweilig mitdiskutieren. Dazu kommen dann meist nochmals Wirtschaftsvertreter, diverse Wissenschaftler und Experten – wobei Dirk Müller\footnote{Bei Dirk Müller handelt es sich um einen Börsenmakler und Buchautor. Anfang 2009 erschien sein Buch „Crashkurs“ zur Finanzkrise, das ihn auch einem breiteren Publikum bekannt und zu einem gefragten Experten für die Medien machte \parencite{gierschCrashkursMisterDax2009}.} am häufigsten auftritt – sowie Journalisten.

Die große Ausnahme ist auf den ersten Blick die \textit{Günther Jauch} Folge vom 23. Oktober 2011. Dort diskutieren einzig die beiden SPD-Granden Helmut Schmidt und Peer Steinbrück, der zum damaligen Zeitpunkt zwar schon als möglicher Kanzlerkandidat seiner Partei im Gespräch, aber noch nicht offiziell ernannt war. Die Folge war für beide eine perfekte Bühne zur Selbstinszenierung, ohne mit allzu viel Widerspruch rechnen zu müssen. Wahrscheinlich war sie darüber hinaus der Ausgleich für die \textit{Günther Jauch} Folge mit Angela Merkel als Exklusivgast am 25. September 2011.

\begin{table}[]
	\centering
	\caption{Korpus der Einzelanalyse im Detailüberblick}
	\label{tab:korpus-einzelanalyse}
	\resizebox{\textwidth}{!}{%
		\begin{tabular}{@{}ll@{}}
			\toprule
			\multicolumn{1}{c}{Sample 1} &
			\multicolumn{1}{c}{Sample 2} \\ \midrule
			\multicolumn{2}{c}{\textbf{Anne Will}} \\ \midrule
			28. September 2011 &
			12. September 2012 \\
			\textbf{Die Euro-Abstimmung – Riskieren wir morgen alles?} &
			\textbf{Das Euro-Urteil – ein guter Tag für Deutschland?} \\
			\begin{tabular}[c]{@{}l@{}}Gäste:\\ Rudolf Wöhrl (Unternehmer)\\ Steffen Kampeter (CDU)\\ Silvia Wadhwa (Journalistin)\\ Klaus von Dohnanyi (SPD)\\ Richard Sulik (Slowakischer Politiker)\end{tabular} &
			\begin{tabular}[c]{@{}l@{}}Gäste:\\ Gregor Gysi (Die Linke)\\ Herta Däubler-Gmelin (SPD)\\ Ursula von der Leyen (CDU)\\ Hans-Werner Sinn (Wissenschaftler)\\ Heiner Bremer (Journalist)\end{tabular} \\ \midrule
			\multicolumn{2}{c}{\textbf{Beckmann}} \\ \midrule
			27. Oktober 2011 &
			06. September 2012 \\
			\textbf{Europa vor dem Abgrund – wie sicher ist unser Geld?} &
			\textbf{Regieren Banken die Politik?} \\
			\begin{tabular}[c]{@{}l@{}}Gäste:\\ Theo Waigel (CDU)\\ Philipp Rösler (FDP)\\ Dirk Müller (Experte)\\ Günter Hörmann (Verbraucherzentrale)\\ Franz Hörmann (Wissenschaftler)\\ Christian Gellerie (Zivilgesellschaft)\end{tabular} &
			\begin{tabular}[c]{@{}l@{}}Gäste:\\ Susanne Schmidt (Ökonomin)\\ Michael Kemmer (Bankenverband)\\ Dirk Müller (Experte)\\ Frank Schäffler (FDP)\end{tabular} \\ \midrule
			\multicolumn{2}{c}{\textbf{Günther Jauch}} \\ \midrule
			23. Oktober 2011 &
			09. September 2012 \\
			\textbf{Klartext in der Krise – Helmut Schmidt und Peer Steinbrück zu Gast bei Günther Jauch} &
			\textbf{Im Namen des Volkes! Müssen wir die Euro-Rettung stoppen?} \\
			\begin{tabular}[c]{@{}l@{}}Gäste:\\ Helmut Schmidt (SPD)\\ Peer Steinbrück (SPD)\end{tabular} &
			\begin{tabular}[c]{@{}l@{}}Gäste:\\ Beatrice von Weizäcker (Journalistin)\\ Ottmar Issing (Ökonom)\\ Winfried Hassemer (Richter)\\ Markus Söder (CSU)\\ Hans-Ulrich Jörges (Journalist)\end{tabular} \\ \midrule
			\multicolumn{2}{c}{\textbf{hart aber fair}} \\ \midrule
			24. Oktober 2011 &
			18. Juni 2012 \\
			\textbf{Bürger gegen Banken – Wut und Angst im Euroland} &
			\textbf{Die Griechenwahl – statt Ende mit Schrecken jetzt Schecks ohne Ende?} \\
			\begin{tabular}[c]{@{}l@{}}Gäste:\\ Frank Lehmann (Journalist)\\ Herbert Walter (Dresdner Bank)\\ Hannelore Kraft (SPD)\\ Hermann Solms (FDP)\\ Heiner Geißler (CDU)\end{tabular} &
			\begin{tabular}[c]{@{}l@{}}Gäste:\\ Hermann Gröhe (CDU)\\ Costa Cordalis (Musiker)\\ Dirk Müller (Experte)\\ Nikolaus Blome (Journalist)\\ Alexis Passadakis (Attac)\\ Tanja Nettersheim (Normalbürgerin)\end{tabular} \\ \midrule
			\multicolumn{2}{c}{Menschen bei Maischberger} \\ \midrule
			18. Oktober 2011 &
			08. Mai 2012 \\
			\textbf{Die Angst wächst – Eurokalypse now?} &
			\textbf{Alle gegen Merkel – Europa in Gefahr?} \\
			\begin{tabular}[c]{@{}l@{}}Gäste:\\ Ursula von der Leyen (CDU)\\ Kurt Biedenkopf (CDU)\\ Sarah Wagenknecht (Die Linke)\\ Hans-Olaf Henkel (Wissenschaftler)\\ Marie-Christin Ostermann (Bund Junger Unternehmer)\end{tabular} &
			\begin{tabular}[c]{@{}l@{}}Gäste:\\ Richard von Weizäcker (CDU)\\ Arnulf Baring (Historiker)\\ Peter Scholl-Latour (Journalist)\\ Jakob Augstein (Journalist)\end{tabular} \\ \midrule
			\multicolumn{2}{c}{\textbf{log in}} \\ \midrule
			26. Oktober 2011 &
			12. September 2012 \\
			\textbf{Der Euro-Poker – Retten wir die Richtigen?} &
			\textbf{Zahlen, bis es kracht – Zum Euro verurteilt?} \\
			\begin{tabular}[c]{@{}l@{}}Gäste:\\ Peter Altmaier (CDU)\\ Wilhelm Hankel (Wissenschaftler) \\ Katrin Henneberger (Grüne)\end{tabular} &
			\begin{tabular}[c]{@{}l@{}}Gäste:\\ Dietmar Bartsch (Die Linke)\\ Norbert Barthle (CDU)\\ Roman Huber (Mehr Demokratie e.V.)\end{tabular} \\ \midrule
			\multicolumn{2}{c}{\textbf{maybrit illner}} \\ \midrule
			13. Oktober 2011 &
			13. September 2012 \\
			\textbf{Griechen pleite, Banken in Not – wer rettet den Steuerzahler?} &
			\textbf{Zur Rettung verurteilt - Was ist uns Europa wert?} \\
			\begin{tabular}[c]{@{}l@{}}Gäste:\\ Sarah Wagenknecht (Die Linke)\\ Michael Kemmer (Bankenverband)\\ Dirk Müller (Experte)\\ Volker Wissing (FDP)\\ Wolfram Siener (Occupy)\\ Thorsten Hinrichs (Standard \& Poors)\\ Ulrich Wickert (Journalist)\end{tabular} &
			\begin{tabular}[c]{@{}l@{}}Gäste:\\ Hans Dietrich Genscher (FDP)\\ Joachim Stabbaty (Ökonom)\\ Volker Kauder (CDU)\\ Marie-Christine Ostermann (Bund Junger Unternehmer)\\ Martin Schulz (SPD)\end{tabular} \\ \bottomrule
		\end{tabular}%
	}
\end{table}

\section{Sendungselemente}\label{chap:sendungselemente}

\subsection{Eröffnung}\label{chap:eroeffnung}

Beginnen soll die Darstellung der Analyseergebnisse mit den Eröffnungen der Sendungen und damit des Framings der folgenden Diskussion.

Die Eröffnungen zeigen thematisch bei allen untersuchten Folgen ein ähnliches Bild. Die jeweiligen Moderatoren und Moderatorinnen führen in die Diskussion meist dergestalt ein, dass sie versuchen als Anwalt des „kleinen Mannes“ aufzutreten und zudem eine Verbindung zu aktuellen Ereignissen herstellen – bevorstehende Bundestagsabstimmung oder Demonstrationen, Entscheidungen des Bundesverfassungsgerichts oder Wahlen in Frankreich bzw. Griechenland. Manchmal ist die Prominenz der Gäste aber auch bereits Anlass genug.

\subsubsection{Komplexität und Protest}

Das Anliegen, so wird suggeriert, der folgenden Runden sei zu verdeutlichen was die – unüberschaubare, komplexe und scheinbar ausweglose – Eurokrise und die getroffenen Gegenmaßnahmen für den verunsicherten Bürger bedeuten und ob er jetzt Angst um sein Erspartes und seine Zukunft haben muss. Exemplarisch ist hier die folgende Eröffnung bei \textit{Anne Will}. Das aktuelle Ereignis, auf das sich Will bezieht, ist die anstehende Bundestagsabstimmung über die Ausweitung des Gewährleistungsrahmen für den EFSF \parencite{deutscherbundestagDeutscherBundestagBeschluesseo.J.}.

\begin{description}
	\begin{linenumbers}[1]
		\item \#00:08:19-1\# \textbf{Will}: Guten Abend meine Damen und Herren, herzlich willkommen. Schön, dass sie alle bei uns sind. In der Tat ist es morgen so weit, dann stimmt der Bundestag über den erweiterten Eurorettungsfonds mit dem sehr bürokratischen Namen EFSF ab. Das klingt sehr nüchtern, ist aber gewaltig. Da geht es morgen um viele viele Milliarden Euro die Deutschland zusätzlich riskieren will, zusätzlich zu den Milliarden, die längst in den Rettungsfonds stecken. Wir wollen darüber heute diskutieren, sagen, dass sind Größenordnungen, die mittlerweile ja längst jedes Vorstellungsvermögen sprengen. Und man darf fragen, sollte man dann jetzt noch schnell sein Erspartes in Sicherheit bringen oder ist das gar nicht nötig, weil alles schon gut gehen wird? [$\ldots$] \#00:09:25-4\#
	\end{linenumbers}
	\captionof{transkript}{Anne Will (28.09.2011)}
\end{description}

Zum Teil wird auch nicht so sehr auf die Komplexität der Krise als auf die Wut der Bürger über Politik und Banken, die – schon wieder – mittels Steuergeld gerettet werden müssten, angespielt. Als Beispiel soll hier eine Einführung bei \textit{maybrit illner} dienen. Dort wird kein aktuelles Ereignis angesprochen, Illner nimmt aber nach der Vorstellung der Gäste Bezug auf Occupy Wallstreet und die in diesem Zusammenhang geplanten Demonstrationen in Deutschland:

\begin{description}
	\begin{linenumbers}[1]
		\item \#00:02:31-2\# ((Applaus)) \textbf{Illner}: Einen schönen guten Abend. Seien sie ganz herzlich begrüßt. Sie sind live und in Farbe im Zweiten Deutschen Fernsehen. Hallo, schönen guten Abend. Ja wochenlang wurde uns erzählt, Griechenland muss gerettet werden, sonst gehen wahlweise der Euro oder Europa kaputt. Mittlerweile ist die Katze aus dem Sack, den Griechen ist kaum mehr zu helfen. Aber die Banken müssen Schulden abschreiben und brauchen dringend Milliarden. Schon wieder die Banken? Gerade einmal drei Jahre ist es her, da musste der Steuerzahler den Geldhäusern aus der selbstgemachten Patsche helfen. Haben die Banker und die Politiker nichts daraus gelernt? Vielen Bürgern platzt jedenfalls der Kragen. Unser Thema heute: Griechen pleite, Banken in Not - wer rettet eigentlich den Steuerzahler? Und das sind unsere Gäste. \#00:03:16-2\#
		\item [$\ldots$]
		\item \#00:04:52-8\# ((Applaus)) \textbf{Illner}: (unv.) Seit drei Wochen Seit drei Wochen besetzen Demonstranten, sie haben das verfolgt, die Wallstreet, den weltberühmten Finanzdistrikt. ((Einblendung Wallstreet Besetzung)) Und das ist eine richtige Volksbewegung geworden. Die Menschen protestieren gegen einen scheinbar übermächtigen Feind, den Finanzkapitalismus. Occupy-Wallstreet heißt diese Bewegung und die gibt es jetzt auch bei uns. Am kommenden Samstag werden in Berlin, Hamburg und auch im ((Rückblende ins Studio)) Frankfurter Bankenviertel möglichst viele Menschen demonstrieren. 
	\end{linenumbers}
	\captionof{transkript}{maybrit illner (13.10.2011)}
\end{description}

Auf Proteste in Verbindung mit der Eurokrise nehmen in der Eröffnung allerdings nur drei der vierzehn Folgen Bezug, dies passt zu den Ergebnissen des quantitativen Teils, die zeigten, dass zivilgesellschaftliche Akteure kaum in den Sendungen vorkommen\footnote{Zum Umgang mit Vertretern der Zivilgesellschaft siehe Kapitel \vref{chap:akteursgruppen}.}.

\subsubsection{Einheitliche Ausgangssituation}

Über alle Sendungen hinweg zeigt sich zudem eine erstaunliche Einheitlichkeit in der Beschreibung der jeweiligen Ausgangssituation. Die Situation sei so komplex, dass die Bürger die Übersicht in der Krise verloren hätten und es nun mit der Angst zu tun bekämen:

\begin{description}
	\begin{linenumbers}[1]
		\item \#00:00:09-6\# \textbf{Maischberger}: [\ldots] Die Mehrheit ist zu Hause geblieben, wahrscheinlich weil ein ganz anderes Gefühl sie umtreibt – eine große Unsicherheit. Was passiert mit dem eigenen Land, was passiert mit dem eigenen Geld in dieser nicht enden wollenden Krise? Ist es noch sicher und wie kommen wir da wieder raus? Unser Thema heute heißt: Die Angst wächst – Eurokalypse Now! Also droht der Untergang des Euro? [\ldots] \#00:01:42-1\# 
	\end{linenumbers}
	\captionof{transkript}{Menschen bei Maischberger (18.10.2011)}
\end{description}

Dabei stechen vor allem die Katastrophenmetaphern hervor in der die Eurokrise beschrieben wird: „Eurokalypse“ (\textit{Maischberger}, 18.10.11), „Schuldenschrecken“ (\textit{hart aber fair}, 18.06.12), „Europa am Abgrund” (\textit{log in}, 26.10.11), wobei auffällt, dass die Beschreibungen im zweiten Sample tendenziell nüchterner werden. Dies mag damit zusammenhängen, dass hier der aktuelle Aufhänger mehrheitlich die Klage gegen den ESM vor dem Bundesverfassungsgericht ist, welche bereits aus sich heraus eine derartige Relevanz mitbrachte, dass sie nicht noch mittels angsterzeugender Rhetorik weiter aufgebauscht werden muss. Darauf deuten auch die Titel der Folgen hin. Während im ersten Sample noch fünf der sieben Titel eine bevorstehende Katastrophe suggerierten, war dies im zweiten Sample nur noch bei dreien der Fall.

\subsubsection{Nationalisiertes Framing}

Dadurch, dass die Moderatoren versuchen ihrer anwaltschaftlichen Rolle gerecht zu werden, kommt es zudem zu einer gewissen „Nationalisierung” des Framings in den Eröffnungssequenzen – und wie wir sehen werden auch innerhalb gesamter Folgen. Die zu erörternden Probleme werden fast immer aus deutscher Perspektive – entweder der des Staates oder der des Steuerzahlers – eingeführt. Wiederum exemplarisch lässt sich dies an dem folgenden Ausschnitt aus \textit{maybrit illner} zeigen:

\begin{description}
	\begin{linenumbers}[1]
		\item \#00:00:08-3\# ((Applaus)) \textbf{Illner}: Einen schönen guten Abend. Seien sie ganz herzlich willkommen. Zweites Deutsches Fernsehen heute Abend. Tja das Bundesverfassungsgericht hat gesprochen, die Politiker in Berlin und Brüssel jubeln, die Börsen auch und die Bürger? Die Bürger zweifeln und verzweifeln. Der Ball ist wieder im Feld der Politik. Sie, die Politik, muss die grummelnde Mehrheit endlich davon überzeugen, dass das Ziel, Europa nämlich, den aberwitzig hoch wirkenden Einsatz lohnt. Denn darüber hatten die Richter ja gar nicht zu entscheiden. Ob die Milliarden für den Dauerrettungsschirm ESM gut angelegtes Geld sind, dafür verbürgt sich die deutsche Kanzlerin und dafür bürgen die deutschen Steuerzahler. Ist das verantwortbar? Gar unumkehrbar? Verliert Deutschland dabei immer mehr seine Souveränität und merken wir das nur deshalb nicht, weil diese Revolution in Zeitlupe stattfindet? Darüber wollen wir heute reden mit diesen Gästen. \#00:01:11-0\#
	\end{linenumbers}
	\captionof{transkript}{maybrit illner (13.09.2012)}
\end{description}

\subsubsection{log in: Eins gegen Eins}

Eine gewisse Sonderstellung haben die Eröffnungen bei log in. Inhaltlich unterscheiden  sich diese zwar nicht, allerdings in ihrer Form. Die Sendung ist entsprechend ihrer  geringeren Gästezahl – normalerweise treten zwei Hauptdiskutanten und weitere ein bis zwei weitere Personen auf – viel stärker auf die Konfrontation bloß zweier Positionen angelegt. Diese Standpunkte werden anders als bei den übrigen Sendungen bereits zu Beginn explizit mittels eines Einspielers dargestellt und dann nochmals mittels einer etwas später folgenden Abstimmung zwischen den beiden Positionen zugespitzt. Durch diese Vorgehensweise werden die grundlegenden Meinungsgegensätze wenn auch nicht allzu differenziert so doch klar herausgearbeitet. Zwar werden die Gäste und ihre Meinung auch in den anderen Sendungen zu Beginn vorgestellt, allerdings nur mit ein, zwei Sätzen und da es sich zudem meist um vier oder fünf Personen handelt, nicht derart stark auf zwei diametral gegensätzliche Positionen zulaufend. Typischerweise sieht das dann wie im folgenden Transkriptauszug aus:

\begin{description}
	\begin{linenumbers}[1]
		\item \#00:00:32-7\# \textbf{Ulrich}: Grade noch im Sommer wollte die Kanzlerin von einem Schuldenschnitt nichts wissen, jetzt liegt er auf dem Tisch. Grade noch im Sommer hieß es noch Kredit hebeln wie die bösen Bankenzocker, das wollen wir auf keinen Fall und genau das hat der Bundestag heute morgen beschlossen. Es sei die schwerste Stunde für Europa seit dem Zweiten Weltkrieg, hat die Kanzlerin heute morgen bei dieser Debatte gesagt. Eins ist sicher die Eurokrise betrifft uns alle. \#00:00:57-6\#
		
		\item \#00:00:57-6\# ((Einspieler)) \textbf{Off}: Europa am Abgrund. Heute geht es im Milliardenpoker um alles, mal wieder. Am Morgen Abstimmung im Bundestag, zur Stunde Eurogipfel in Brüssel. Die Regierenden ringen um das große Rettungspaket, der Währungsunion droht die Kernschmelze und seit Wochen wiederholt die Kanzlerin Gebetsmühlenartig: „Scheitert der Euro, dann scheitert Europa.” Griechenland im Epizentrum, jetzt kommt der Schuldenschnitt. Versicherer und Banken müssen auf die Hälfte ihrer Forderungen gegenüber Athen verzichten. Im Juli noch sollten es nur ein Fünftel der Ansprüche sein. Alles Tropfen auf den heißen Stein? Pumpen wir unser Geld in ein Pleiteland? Und retten wir überhaupt die Richtigen? Ja, meint CDU-Politiker Peter Altmaier. Er ist so etwas wie Merkels Eurotrommler, der Mann, der dem Rettungsschirm die Kanzlermehrheit besorgt. An der Griechenlandhilfe darf kein Zweifel bestehen sagt er. \#00:01:57-6\#
		
		\item \#00:01:57-6\# \textbf{Ulrich}: Und hier ist der Eurotrommler der Kanzlerin. Herzlich Willkommen, Peter Altmaier. ((Applaus)) Schön, dass sie bei uns sind. Nehmen Sie doch Platz. \#00:02:10-8\#
		
		\item $\ldots$
		
		\#00:02:35-4\# \textbf{Ulrich}: Wenn es den anderen gut geht in Europa, dann geht's uns auch gut, sagt Peter Altmaier. Unser nächster Gast sieht das völlig anders. Er ist entsetzt über diese Regierung, er hat gegen den Euro geklagt, er ist der Meinung man sollte Griechenland aus der Eurozone rausschmeißen und heute Abend ist er hier. Herzlich Willkommen, Professor Wilhelm Hankel. ((Applaus)) \#00:02:57-9\#
		
		\item $\ldots$
		
		\#00:03:53-5\# \textbf{Huwendiek}: Genau und ihr habt jetzt 55 Minuten, jetzt wo die Positionen klar sind, darüber abzustimmen was ihr denn am besten findet. Ganz einfach unter login.zdf.de, da stellen wir nämlich die Frage „Euro-Poker: Retten wir die Richtigen? Ja, wir sind in einer Schicksalsgemeinschaft und müssen uns gegenseitig helfen.” Ist die eine Position und die andere: „Nein, die Eurorettung wird nur funktionieren, wenn die Währungssünder aus der Eurozone ausscheiden.” Also macht da bitte mit und sagt uns was ihr besser findet. [\ldots] \#00:04:45-6\#
	\end{linenumbers}
	\captionof{transkript}{log in (26.10.2011)}
\end{description}

Diese Vereinfachung und Zuspitzung zu Beginn hat neben Vorteilen – erhöhte Ver"-ständ"-lich"-keit und Anschaulichkeit – allerdings auch Nachteile. Andere Standpunkte als die der beiden Hauptgäste können höchstens durch die im späteren Sendungsverlauf auftretenden zusätzlichen Gäste zeitweilig eingebracht werden oder durch die Moderation. Im konkreten Fall heißt dies, dass nur die zwei Meinungen „Weiter so!“ und „Griechenland raus!“ vorkommen. Wie später noch gezeigt wird, kommen allerdings auch in den anderen Sendungen selten wesentlich mehr Positionen vor.

\subsection{Abschluss}

Während der Talks halten sich die Moderatoren mit Fragen, die auf eine persönliche, themenfremde Ebene zielen, weitgehend zurück. Das Ende der jeweiligen Diskussionsrunden hingegen ist oftmals der Platz für derartige Fragen an die Gäste. Am stärksten ritualisiert ist dieses Vorgehen bei \textit{hart aber fair}. Die Sendung endet immer damit, dass Plasberg die Gäste reihum um ein Statement auf eine persönliche, lustig gemeinte Frage bittet:

\begin{description}
	\begin{linenumbers}[1]
		\item \#01:13:59-4\# \textbf{Plasberg}: Schlussrunde bei hart aber fair, da ist wie immer Fantasie gefragt. Herr Walter, ich fang bei ihnen an. Stellen sie sich vor ähm sie müssten oder dürften eine Nacht, sie haben da ja auch Sympathie geäußert, im Protestcamp verbringen. Bedenken sie, der Mensch eben hat's gesacht, es ist kalt. Mit dem, mit wem aus dieser Runde würden sie gerne ihr Zelt teilen, wenn es nachts wirklich kalt wird? \#01:14:22-6\#
	\end{linenumbers}
	\captionof{transkript}{hart aber fair (24.10.2011)}
\end{description}

Aber auch bei \textit{Anne Will} und \textit{Günther Jauch} finden sich Sendungsabschlüsse, die nichts mit dem eigentlichen Thema der jeweiligen Diskussionsrunde zu tun haben. Günther Jauch fragt beispielsweise Peer Steinbrück, ob er als SPD-Kanzlerkandidat antreten will  und Anne Will versucht von Ursula von der Leyen zu erfahren, ob sie Merkel nachfolgen möchte (vgl. Transkript \vref{lis:34}). Bei Menschen bei Maischberger hingegen enden die Runden recht abrupt (vgl. Transkript \vref{lis:35}). Beckmann bittet seine Gäste immerhin in einer Folge um eine Prognose zum weiteren Verlauf der Eurokrise (vgl. Transkript \vref{lis:36}) und Illner lässt in einer Schlussrunde nochmal jeden Gast zu Wort kommen.

Insgesamt versuchen die Moderatoren jedoch fast nie am Ende ein Ergebnis, gar einen Konsens der zurückliegenden Diskussion aufzuzeigen oder diese zumindest zusammenzufassen. Stattdessen stehen die verschiedenen Positionen zur Eurokrise (vgl. Kapitel \vref{chap:diskurstopoi}) wie zu Beginn unversöhnt nebeneinander.

Eine Ausnahme ist wiederum \textit{log in}. Dort besteht die Schlusssequenz aus der Präsen"-tation der Er"-gebnisse der Zuschauerabstimmung. Dadurch wird zumindest suggeriert welche der beiden Standpunkte der bessere war und die Zuschauer mehr überzeugt hat (vgl. Transkript \vref{lis:7}). Zahlen dazu wie viele Zuschauer sich beteiligt haben fehlen und es fällt damit schwer die Aussagekraft der Abstimmung einzuschätzen. Dennoch erscheint der Zuschauer hier als eine Art „Souverän des Geschehens“ \parencite[153]{doernerPolitainmentPolitikMedialen2001}. Bei den beiden Folgen gewinnen die Abstimmung übrigens die gegen die Regierungspolitik gerichtete Position – das Griechenland aus dem Euro ausscheiden sollte und dass das Urteil des BVerfG der Europolitik Merkels Grenzen gesetzt habe.

\begin{description}
	\begin{linenumbers}[1]
		\item \#00:58:05-8\# \textbf{Ulrich}: Und jetzt gehen wir nochmal ganz schnell ins Netz und ich möchte wissen, Frederik, wie unsere Abstimmung gelaufen ist? \#00:58:09-6\#
		
		\item \#00:58:09-6\# \textbf{Huwendiek}: Tja, einen Moment, da schau ich hier. Und zwar sieht es so aus, dass 20 \% sagen, ja wir leben in einer Schicksalsgemeinschaft und müssen uns gegenseitig helfen. Und 80 \% sagen, nein, die Eurorettung wird nur funktionieren, wenn die Währungssünder aus der Eurozone ausscheiden. \#00:58:25-6\#
		
		\item \#00:58:25-6\# \textbf{Ulrich}: Und gibt's noch was Neues aus Brüssel? // (unv.) // \#00:58:27-3\#
		
		\item \#00:58:27-3\# \textbf{Huwendiek}: // Nein bislang nix //, nur weiter unbestätigte Gerüchte. \#00:58:31-3\#
		
		\item \#00:58:31-3\# \textbf{Ulrich}: Das ist nicht ganz ihre Position gewesen, Herr Alt"-maier. \#00:58:33-8\#
		
		\item \#00:58:33-8\# \textbf{Altmaier}: Nein, aber ich finde, dass wir Politiker auch dafür gewählt werden, dass wir das tun was wir für richtig halten und alle vier Jahre entscheiden die Menschen, ob sie uns wiederwählen, ob sie uns noch vertrauen. Ich bin jetzt seit 1994 im Deutschen Bundestag. Ich hab bei jeder Wahl ein äh flaues Gefühl im Bauch und im Magen, ob's denn wieder ein Vertrauensvorschuss gibt. Und bisher haben mich meine Wählerinnen und Wähler jedes Mal wiedergewählt. \#00:58:58-1\#
		
		\item \#00:58:58-1\# \textbf{Ulrich}: Herr Altmaier, herzlichen Dank, dass sie trotz ihres Stress heut' Abend bei uns waren und mit uns zusammen diskutiert haben. Herr Hankel, ganz herzlichen Dank, dass sie heute Abend bei uns waren. [\ldots] \#00:59:24-4\# 
	\end{linenumbers}
	\captionof{transkript}{log in (26.10.2011)}
	\label{lis:7}
\end{description}

Auch versuchen die Moderatoren durch den Blick auf die Reaktionen der Zuschauer im Internet ein Fazit der Gesprächsrunde zu ziehen wie im folgenden Ausschnitt:

\begin{description}
	\begin{linenumbers}[1]
		\item \#00:59:30-8\# \textbf{Ulrich}: Herzlichen Dank, dass sie alle bei uns mitgemacht haben. Feed"-back-Netz gibt es natürlich noch. Janine, wie haben die User die Sendung gesehen? \#00:59:37-3\#
		
		\item \#00:59:37-3\# \textbf{Michaelsen}: Das kann ich äh sofort sagen. Wenn es ein allgemein akzeptiertes Fazit gibt, das hat uns quasi jemand geliefert. ((liest Kommentar vor))  „Mehr Demokratie auf EU-Ebene. Die Meinung zum ESM bleibt gespalten.“ Und auch Twitter wollen wir nicht vernachlässigen an diesem Abend. ((liest Tweet vor)) „Man muss so langsam entscheiden, ob man mehr Europa will in Richtung Vereinigte Staaten oder weniger. Euro weg. So geht's nicht weiter.“ \#00:59:57-3\#
	\end{linenumbers}
	\captionof{transkript}{log in (12.09.2012)}
\end{description}

Zusammenfassend kann man also festhalten, dass es von Seiten der Moderation zwar tatsächlich darum geht Europa und „Deutschland erst in Gefahr zu wiegen“ \parencite[17]{rossumMeineSonntageMit2004}, es dann aber versäumt wird beide „anschließend zu retten“ \parencite[17]{rossumMeineSonntageMit2004}. Nun ist dies allerdings nicht unbedingt negativ zu bewerten. Zum einen kann dem mündigen Zuschauer durchaus zugetraut werden, die in den Talksendungen geäußerten Argumente und Positionen selbst zu bewerten \parencite[18]{tolsonTalkingTalkAcademic2001}. Zum anderen verschleiert ein falscher Konsens des Moderators möglicherweise die Unvereinbarkeit bestimmter Positionen und die dahinterstehenden Interessengegensätze.

\subsection{Serviceelemente}

Eine Besonderheit, die in einigen Folgen vorkommt und wohl besondere Bürgernähe suggerieren soll sind kurze Beratungssequenzen. Bei diesen Gesprächen zwischen Moderator und meist einem Gast geht es um (Zuschauer-) Fragen nach der richtigen Anlagestrategie in der Krise, ob das Geld in der Lebensversicherung oder auf dem Sparkonto noch „sicher“ ist. Eine typische Sequenz ist im nachstehenden Transkript dokumentiert.

\begin{description}
	\begin{linenumbers}[1]
		\item \#01:13:34-4\# \textbf{Will}: Eine Frage haben wir da noch und dann.  \#01:13:36-2\# 
		
		\item \#01:13:36-2\# \textbf{Zuschauerin}: Mich würde interessieren, was sie mir empfehlen würden, wenn ich einen bestimmten kleineren Betrag monatlich von meinem Gehalt äh investieren, anlegen möchte und das doch relativ sicher sein sollte aber doch auch ertragreich? \#01:13:49-7\# 
		
		\item \#01:13:50-3\# \textbf{Wadhwa}: Gut. Also. Hohe Rendite, hohes Risiko. Also wenn sie eher auf Ertrag setzten, dann müssen sie sich vielleicht bald ein paar Baldriantropfen kaufen und ein bisschen dickere Nerven entwickeln. Wenn sie aber auf sicher setzten, dann sind die Rezepte immer noch die alten. Kleines, gut gemischtes äh Anlageportfolio. Bisschen Anleihen, bisschen Aktien, bisschen Währung. Und äh man hofft das der Anlageberater bei der Bank ihnen da die richtigen Tipps geben kann. \#01:14:17-5\# 
	\end{linenumbers}
	\captionof{transkript}{Anne Will (28.09.2011)}
\end{description}

Derartige, bis zu zehn Minuten lange, Sequenzen finden sich bei \textit{Anne Will} (28.09.2011), \textit{Beckmann} (27.10.2011) und \textit{hart aber fair} (24.10.2011), also drei Folgen, die mit sehr kurzem zeitlichen Abstand zueinander ausgestrahlt wurden. Fragen und auch Antworten ähneln sich und bleiben, wie im angeführten Beispiel,  meist auf einer relativ allgemeinen Ebene, weshalb der Nutzen für die Zuschauer eher gering sein dürfte. Die Antworten sind meist beruhigender Natur, der Tenor ist, dass das Geld der Bürger in Anlageportfolios und auf Sparbüchern auch weiterhin „sicher“ sei. Es ist sicherlich legitim auch in Polittalks auf derartige Fragen einzugehen, zumal wenn sie von Zuschauern an die Redaktion herangetragen wurden. Allerdings bleibt in den Sendungen nicht der Raum um diese Fragen fundiert zu beantworten und das ganze gleich dreimal innerhalb von vier Wochen zu exerzieren erscheint zudem wenig sinnvoll.

\subsection{Gesprächsorganisation und restliche Moderation}

Während hinsichtlich der verschiedenen Einführungen in die Sendungen als auch bei deren Abschluss eine relativ große Übereinstimmung festgestellt werden konnte, zeigen sich bei der weiteren Moderation Unterschiede. Zwar stellen alle Moderatoren und Moderatorinnen (Nach-) Fragen, fordern und liefern Erklärungen, organisieren den Gesprächs"-ver"-lauf und teilen das Rederecht zu. Jedoch zeigen sich Abstufungen beim Grad und Erfolg dieser Handlungen bei einzelnen Sendungen.

Beckmann stellt kaum kritische Nachfragen, er beschränkt sich meist auf Aufforderungen einen bestimmten Sachverhalt näher zu erklären und Stichworte zu liefern. Auch lässt er das Gespräch relativ lange ohne Eingriffe laufen, was recht gut gelingt, ohne dass die Runde ins Chaos entgleiten würde. Möglicherweise liegt der Grund hierfür in dem ruhigen Kommunikationsklima und einer dazu passenden Besetzungsstrategie. Beckmann ist es auch, der immer mal wieder persönliche Fragen einstreut. Trotz des dadurch teilweise „menschelnden“ Charakters der Sendung, zählen die zwei Folgen zu den informativsten des untersuchten Samples\footnote{Zu einem ähnlichen Ergebnis gelangt der ARD-Programmbeirat in seine Beurteilung der Talkschiene \parencite[8]{ard-programmbeiratTalkformateImErsten2012}.}. Dies ist auch dem Umstand zu verdanken, dass Beckmann die Folge vom 27. Oktober 2011 fast vollständig Zuschauerfragen widmet.

Wesentlich schlechter hat Maischberger ihre Gäste im Griff. Zwar versucht sie im Gegensatz zu Beckmann mehrfach kritisch Fragen zu stellen, ihr fehlt aber die Durchsetzungskraft, um diesen zur Geltung zu verhelfen. Insbesondere in der zweiten hier untersuchten Folge entgleitet der Moderatorin immer wieder die Leitung der Talkrunde, was sich letztlich in einem thematischen Chaos manifestiert. Dies ist so groß, dass sie sogar von ihren Gästen mehrfach darauf hingewiesen wird (vgl. Tanskript \vref{lis:10}).

\begin{description}
	\begin{linenumbers}[1]
		\item \#00:25:54-4\# \textbf{Weizäcker}: Ordnen sie uns. \#00:25:55-5\#
		
		\item $\ldots$
		
		\item \#01:00:52-0\# \textbf{Baring}: Ja wir reden hier über alles gleichzeitig. \#01:00:53-4\#
	\end{linenumbers}
	\captionof{transkript}{Menschen bei Maischberger (08.05.2012)}
	\label{lis:10}
\end{description}

Durch das häufige Durcheinanderreden und die zahlreichen Themensprünge ist es bei \textit{Menschen bei Maischberger} am schwersten dem Sendungsverlauf zu folgen. Hinsichtlich ihrer Informativität und Verständlichkeit liegen die Folgen entsprechend am unteren Ende des Samples. Zwar haben auch die anderen Moderatoren immer mal wieder Probleme ihre Gäste „unter Kontrolle“ zu halten, aber nie in einem so extremen Maße wie bei \textit{Menschen bei Maischberger}. Der Verzicht auf Studiopublikum scheint also nicht per se zu einer ruhigeren Gesprächsatmosphäre zu führen.

\textit{log in} fällt durch seine Doppelmoderation und die starke Einbindung der Zuschauer aus der Reihe. Moderator Wolf Ulrich fragt regelmäßig seine Co-Moderatorin nach Reaktionen und Fragen der Zuschauer, die diese dann wiederum an die Gäste weiterreicht. Da Ulrich selbst kaum eigene Fragen stellt, nehmen die Moderatoren hier am stärksten eine anwaltschaftliche Rolle ein, da sie über weite Strecken tatsächlich als direktes „Sprachrohr“ der Zuschauer agieren. Gleichzeitig führt das ständige Zurückkommen auf „die User“ dazu, dass kein rechtes Gespräch zustande kommen will. Auffällig ist zudem, dass in den beiden vorliegenden Folgen jeweils der Regierungsvertreter an der sogenannten \textit{log in direkt} Runde teilnehmen durfte und damit bevorzugt wurde. Bei dieser Runde muss man in zwei Minuten zu möglichst vielen Userfragen Stellung nehmen, die anderen Talkgäste können sich in dieser Zeit nicht äußern.

Maybrit Illners Moderationsstil zeichnet sich durch eine relativ große Strenge und Schärfe aus. Sie unterbricht und kritisiert die Ausführungen ihrer Gäste recht häufig und bedient sich des öfteren speziellen Fragetechniken wie geladenen Fragen. Im folgenden Beispiel rügt Illner die Äußerungen des Deutschlandchefs der Ratingagentur Standard \& Poors als unverbindliche Floskeln (Z. \lref{11:1}f.):

\begin{description}
	\begin{linenumbers}[1]
		\item \#00:53:45-7\# \textbf{Illner}: Die amerikanische Börsenaufsicht haben sie grade erwähnt eben diese SEC. Diese ist mit dem was sie da gemacht haben in den Jahren 2007 und 2008 auch ausgesprochen unzufrieden. Die stellt ihnen äh erwägt Betrugsklagen äh anzustrengen eben wegen den falschen Bewertungen die es gegeben hat in den Zeiten der Finanzkrise. Rechnen sie da mit irgendwelchen Konsequenzen sich selbst betreffend, ihre Firma betreffend? \#00:54:07-0\#
		
		\item \#00:54:07-5\# \textbf{Hinrichs}: Ich halt es für völlig normal, dass äh die Ereignisse von damals nochmal aufgearbeitet werden. Das ist auch nicht nur bei uns so, das ist auch bei den anderen Agenturen so. Ähm. Und wir arbeiten vertrauensvoll auch mit der SEC hier zusammen. \#00:54:18-9\#
		
		\item \#00:54:19-0\# \textbf{Illner}: Mhm (bejahend). \llabel{11:1}Das ist ein guter Satz. Der passt auch in jede Pressemitteilung. [$\ldots$] \#00:54:1-7\#
	\end{linenumbers}
	\captionof{transkript}{maybrit illner (13.10.2011)}
\end{description}

\textit{Anne Will}, \textit{Günther Jauch} und \textit{hart aber fair} zeigen keine Merkmale, die sie deutlich gegeneinander abgrenzen würden. Wirklich negativ sticht lediglich die \textit{Günther Jauch} Folge vom 23. Oktober 2011 hervor. Den Gästen Helmut Schmidt und Peer Steinbrück bietet Jauch hier eine reichweitenstarke Plattform zur Selbstinszenierung, die diese gerne nutzen und dabei vom Moderator nichts entgegengesetzt bekommen. Teils werden sie sogar von Jauch tatkräftig unterstützt, beispielsweise wenn es um Werbung für das neuste Buch der beiden Talkgäste geht (vgl. Transkript \vref{lis:10}).

\begin{description}
	\begin{linenumbers}[1]
		\item \#00:25:20-3\# \textbf{Jauch}: Es gibt ein Buch das auf mehreren Gesprächen zwischen ihnen beiden (..) basiert. Das in dieser Woche (..) herauskommt. Genauso wie ein umfänglicher Vorabdruck am Donnerstag in der Zeit. Ein Interview mit ihnen beiden im neuen Spiegel. Und im ersten Kapitel ihres Buches unterhalten sie sich über globale Verschiebungen. Und da fragen sie Herr Steinbrück, ich zitiere, aber was hieße es für die Welt insgesamt, wenn eines Tages mit China die stärkste Volkswirtschaft der Welt keine Demokratie mehr wäre. Und sie Herr Schmidt, sie antworten, dass bedeutet für die Welt zunächst gar nichts. Man darf die Bedeutung der Demokratie für die Weltbevölkerung nicht überschätzen. Man darf die Demokratie auch nicht übermäßig idealisieren. Wie meinen sie das? \#00:26:13-5\#
	\end{linenumbers}
	\captionof{transkript}{Günther Jauch (23.10.2011)}
	\label{lis:12}
\end{description}

Bei einer genaueren Untersuchung der verschiedenen Moderationen ließen sich vermutlich weitere und teils deutlichere Unterschiede herausarbeiten, jedoch ist dies nicht das Hauptanliegen dieser Arbeit. Stattdessen sei noch ein kurzer Blick auf die Einspieler geworfen, bevor wir uns dem Umgang mit verschiedenen Akteursgruppen und den Diskurstopoi widmen.

\subsection{Einspieler}

\subsubsection{Straßenumfragen}

Die vom ARD-Programmbeirat noch aufgrund ihres häufigen Einsatzes und der nicht vorhandenen Repräsentativität kritisiert"-en Straßen"-um"-frag"-en \parencite[4]{ard-programmbeiratTalkformateImErsten2012} fehlen fast vollständig. Einzig bei \textit{Anne Will} wird am 28. September 2011 ein derartiger Einspieler gezeigt. Selbiger belegt dann auch direkt die Untauglichkeit von Straßenumfragen, weshalb das Transkript im Folgenden in voller Länge wiedergegeben wird.

\begin{description}
	\begin{linenumbers}[1]
		\item \#00:46:06-3\# \textbf{Will}: Aber handeln die Abgeordneten tatsächlich im Sinne der Menschen. Denn das hat er Sulik angesprochen, man wagt nicht die Menschen zu fragen. \llabel{13:1}Umfragen gab's ja noch nöcher die haben gesagt zweidrittel, manche andere Umfrage sagt äh 75 \% gar sind gegen den Rettungsfonds. Sie sind direkt gewählter Abgeordneter Herr Kampeter aus der schönen Ecke Minden-Lübbecke. Sind sie noch sicher, dass sie im Sinne derer handeln die sie dort vertreten? \#00:46:32-6\#
		
		\item \#00:46:32-6\# \textbf{Kampeter}: Frau Will. Ist ne Frage äh wie sie die Leute fragen. Wenn sie die Leute fragen, wollen wir Geld nach Griechenland verschenken, werden sie wahrscheinlich 90 \% oder 100 \% haben die ((unv. wegen Fehler der Tonspur)). Wenn sie sie fragen, wenn sie sie fragen, wollt ihr das die politischen Verhältnisse in Europa stabil, friedlich und äh freiheitlich sind und das ihr eine stabile Währheit Währung habt. Wenn sie die Frage so stellen, werden sie eine höhere Zustimmung erhalten als // bei der anderen Frage. // \#00:46:57-5\# 
		
		\item \#00:46:56-7\# \textbf{Will}: // Stimmt. // \llabel{13:2}Weisen die Umfragen aus. Aber ich zeig ihnen wie wir ihre Bürger dort gefragt haben. \#00:47:01-8\#
		
		\item \#00:47:02-7\# ((Einspieler)) \textbf{Off}: Minden-Lübbecke. Schon seit 1990 der Wahlkreis unseres Gastes Steffen Kampeter. Zuletzt hat er das Direktmandat gewonnen und \llabel{13:3}mit ihrem Abgeordneten, so erzählen es die Menschen auf der Straße, sind sie im Allgemeinen auch zufrieden. ((Bürger 1)) „Ich kenne Herrn Kampeter auch persönlich, also in sofern ($\ldots$)” ((Reporter)) „Macht er gut?” ((Bürger 1)) „Bin ich damit eigentlich (unv.) zufrieden was er macht.” ((Bürger 2)) „Er macht seine Arbeit, soweit wie ich weiß, gut (lacht)” ((Bürger 3)) „Er wollt ja mal Bundeskanzler werden und ich, an für sich würd' ich's ihm wünschen.”\llabel{13:4} ((Langsame Musik)) \llabel{13:5}Doch in der Eurokrise fühlen sich viele Mindener von ihrem Volksvertreter und der Politik insgesamt nicht mitgenommen. ((Bürger 4)) „Da steigt man nicht mehr durch.” ((Interviewer)) „Zu kompliziert?” ((Bürger 4)) „Ich glaube nicht genug richtig erklärt.” ((Bürgerin 5)) „Die reden immer so drum rum, ne?” ((Bürger 6)) „Politik an sich gibt sich, find ich, keine Mühe das wirklich äh klar zu machen was jetzt auf dem Spiel steht.” ((Bürger 7)) „Die Auf Parteien haben ja wohl die Aufgabe äh an der politischen Willensbildung des Volkes mitzuwirken und nicht die politische Willensbildung des Volkes zu ersetzen.”\llabel{13:6} ((Langsame Musik)) \llabel{13:7}Zweifel haben viele Menschen in Minden-Lübbecke an der Rettung Griechenlands. ((Bürger 8)) „Das ist ein Fass ohne Boden. Da schmeißen wir immer mehr rein und es kommt nix raus.” ((Bürger 9)) „Man will unbedingt Griechenland retten. Meine Meinung.” ((Interviewer)) „Und ist das richtig?” ((Bürger 9)) „Nein! Es ist nicht richtig. Die kommen ja nie wieder auffe Beine. Die können doch nie wieder ihre Schulden bezahlen, das gibt ja gar nicht. Das weiß auch jeder Politiker der ein bisschen Verstand hat. Die Griechen müssen wieder zurück zu ihren Drachmen oder wie die Währung da hieß.”\llabel{13:8} ((Langsame Musik)) \llabel{13:9}Und was raten die Mindener ihrem Abgeordneten Steffen Kampeter für die Abstimmung morgen im Bundestag? ((Bürgerin 10)) „Ich würde mit Nein stimmen. Es wär' ja ganz gut, wenn der Bürger auch mehr gefragt würde, ne?” ((Bürgerin 11)) „Schwierig. Also bei mir ist die Meinung durchwachsen und ich bin froh, dass ich die Entscheidung nicht zu treffen habe.” ((Bürger 12)) „Ich würde eher sagen, mit Nein stimmen. Denn äh die Ausweitung kann nicht äh endlos weiter gehen und irgendwo ist die Grenze erreicht.” ((Bürger 13)) „Ich denke er wird keine andere Wahl haben, als dass er sich dafür ausspricht. Obwohl ich selber nicht so dafür bin.” ((Bürger 14)) „Er soll dagegen stimmen. Er darf überhaupt nicht dafür stimmen, weil wir ja in die Gesamteuropa bald mit unsern Steuergeldern hier unterhalten und das ist nicht im Sinne des wahren Jakobs.”\llabel{13:10} \#00:49:19-4\#
		
		\item \#00:49:20-6\# ((Applaus)) \textbf{Will}: Herr Kampeter ($\ldots$) Also ein. \llabel{13:11}Ein nicht un"-be"-trächt"-lich"-er Teil der Mindener fühlt sich durch sie nicht mehr gut vertreten. Sind sie da noch ein guter // Volksvertreter? // \#00:49:31-4\#
		
		\item \#00:49:31-1\# \textbf{Kampeter}: // Also erstmal // find ich toll wie äh meinungsfreudig, wie engagiert die Minden-Lübecker das Thema sagen. \llabel{13:12}Es gibt ein hohes Maß an persönlichem Vertrauen, find' ich gut. Ist ja auch in Ordnung, das gefällt einem. Und ich kann im übrigen auch die Zweifel verstehen. Wir jagen ja äh in einer Gesch/ Wir machen ja Politik auf der Überholspur und die Zweifel gibt's ja nicht nur in der Bevölkerung in Minden-Lübbecke die gibt’s in ganz Deutschland und viele Kollegen im Deutschen Bundestag/  \#00:49:53-0\#
		
		\item \#00:49:53-1\# \textbf{Will}: Was heißt Überholspur? Kann man nicht mehr richtig sehen was man macht?  \#00:49:55-5\#
	\end{linenumbers}
	\captionof{transkript}{Anne Will (28.09.2011)}
\end{description}

Wie dem obenstehenden Transkript zu entnehmen ist, wird der Einspieler von Will in den Kontext von anderen Meinungsumfragen eingeordnet (Z. \lref{13:1}f., \lref{13:2}f.), was bereits suggeriert, die folgende Straßenumfrage sei ebenfalls repräsentativ. Der Einspieler  selbst besteht aus O-Tönen von Bürger die sich in vier Teile gliedern. Zuerst wird Zufriedenheit mit Kampeters Arbeit als Abgeordneter geäußert (Z. \lref{13:3}-\lref{13:4}), danach kommen O-Töne in denen Unzufriedenheit über die mangelnde Information durch die Politik bezüglich der Eurokrise geäußert wird (Z. \lref{13:5}-\lref{13:6}). Im dritten Teil kommen dann Bürger zu Wort, die die Hilfen für Griechenland ablehnen (Z. \lref{13:7}-\lref{13:8}) und schließlich werden die Interviewten noch um Ratschläge für Kampeters Abstimmungsverhalten in der anstehenden Bundestagsabstimmung über den EFSF gebeten (Z. \lref{13:9}-\lref{13:10}). Will selbst deutet das Gezeigte dann so, dass „ein nicht unbeträchtlicher Teil der Mindener“ (Z. \lref{13:11}f.) sich nicht mehr durch Kampeter vertreten fühle, obwohl der Einspieler darüber nur indirekt etwas aussagt und O-Töne von vierzehn Bürgern bei einem Wahlkreis mit fast 270.000 Einwohnern \parencite{derbundeswahlleiterStrukturdatenWahlkreis1352009} kein „nicht unbeträchtlicher Teil“ ist. Entsprechend ist es auch nur folgerichtig, dass Kampeter selbst in seiner Antwort den Einspieler vor allem als Bestätigung seiner Arbeit interpretiert (Z. \lref{13:12}f.).

\subsubsection{Einseitige Einspieler}

Ebenfalls fragwürdig ist ein mehrfach verwendeter Einspieler der einen Ausschnitt aus dem ARD-Politmagazin \textit{Panorama} zeigt. Dieser illustriert mittels mehrerer O-Töne die angebliche Unwissenheit von Bundestagsabgeordneten über die Eurokrise und wird sowohl bei Beckmann am 27. Oktober 2011 als auch bei \textit{Günther Jauch} am 09. September 2012 eingesetzt. Hier wird ein populäres Ressentiment – „die da oben wissen eh nicht was sie machen” – bestärkt, dessen Beitrag zur Klärung der zentralen Fragen rund um die Eurokrise genauso gering sein dürfte, wie es schon bei der Finanzkrise war \parencite[12ff.]{kohlerImBlindflugDurch2009}, das aber immerhin als Diskussionsimplus dienen kann. Etwas, das von folgendem Einspieler bei hart aber fair nicht unbedingt gesagt werden kann.

\begin{description}
	\begin{linenumbers}[1]
		\item \#00:21:55-7\# \textbf{Plasberg}: Stell ich mir grade vor in Griechenland kann man über Satellit natürlich ARD gucken, da sitzen jetzt äh Griechen vorm Fernseher die deutsch können, haben die Diskussion verfolgt. Eine wichtige Diskussion, eine Systemdiskussion, die sie auch voran getrieben haben. Die hier auch mit viel Emotion gelaufen ist. Wir sollten uns nochmal angucken in welche Situation die in Griechenland kommt diese Diskussion. Was wir von den Griechen erwarten können und da geht es jetzt mal gar nicht um Emotion sondern es geht um \llabel{14:2}Fakten. Das sind sie. \#00:22:23-0\#
		
		\item \#00:22:23-9\# ((Einspieler)) \textbf{Off}: Jeder zweite junge Grieche ist arbeitslos. Löhne sind im Schnitt um ein Drittel gesunken. Renten wurden halbiert. Bei Strom und Gas droht der Blackout. Die Steuerverwaltung ist marode. Das zeigt dieser Blick in ein Athener Finanzamt das eher einer Müllhalde gleicht. Wichtige Unterlagen sind nicht digitalisiert und deshalb für die Steuerprüfer nicht einsehbar. Das Gesundheitssystem kollabiert. In Thessaloniki werden schon keine Herz-OPs mehr durchgeführt. Apotheken und Ärzte arbeiten nur noch gegen Vorkasse. Pharmafirmen liefern nur noch das nötigste. Medikamente werden knapp. ((O-Ton eines Griechen aus dem Auslandsjournal vom 26.10.2011)) „Inzwischen bekommen auch die Versicherten nicht mehr alle ihre Medikamente. Gestern noch habe ich es im staatlichen Krankenhaus versucht. Sechs Stunden lang, vergeblich.” ((Schwarzer Hintergrund mit weißer Schrift)) So sieht es aus in Griechenland. Auf den ersten Blick eine Folge des europäischen Spar-Diktats. Letztlich aber das Ergebnis jahrelanger \llabel{14:1}Misswirtschaft der beiden Volksparteien. Und ausgerechnet diese Parteien sollen Griechenland jetzt aus der Krise führen. \#00:23:37-1\# 
	\end{linenumbers}
	\captionof{transkript}{hart aber fair (18.06.2012)}
	\label{lis:14}
\end{description}

Problematisch macht diesen Einspieler primär, dass er eine politische Deutung vornimmt: Die Krise in Griechenland und ihre Auswirkungen auf die Bevölkerung werden hier einzig und allein der „Misswirtschaft der beiden Volksparteien” (Z. \lref{14:1}f.) zugeschrieben. Genau dieser Umstand war zuvor in der Diskussion umstritten gewesen, so hatte der Attac-Vertreter mehrfach darauf hingewiesen, dass die gegenwärtige „Griechenlandrettung” der falsche Weg sei, die Bevölkerung darunter leide und nur die Banken gerettet würden. Während die Repräsentanten von CDU und BILD-Zeitung, Gröhe und Blome, die gegenteilige Meinung vertraten. Plasberg beendet durch den Einspieler diesen Streit mit dem Verweis auf die „Fakten” (Z. \lref{14:2}) einseitig zugunsten der Position von Gröhe und Blome. Der Verweis auf Fakten statt Emotionen steht zudem im Widerspruch zu dem später folgenden Versuch Plasbergs eine Behauptung des Attac-Vertreters durch Verweis auf ein emotionales Einzelbeispiel zu entkräften (vgl. Kapitel \vref{chap:akteursgruppen}).

Es lassen sich noch weitere ähnliche Beispiele finden in denen die eingesetzten Einspieler eine fragwürdige Qualität besitzen, dennoch liefern viele der Einspieler solide Informationen, Hintergründe oder Erklärungen zur Eurokrise und stören den Diskussionsverlauf nicht, wenn auch ihr Einsatz selten zwingend ist, da viele nur bereits Gesagtes wiederholen bzw. illustrieren.

Die Sendungen setzten zudem auf unterschiedliche Arten und Häufigkeiten von Einspielern. Bei \textit{log in} dienen sie meist dazu Zuschauerfragen einzubringen, wenn sie zusätzliche Informationen oder Diskussionsimpulse liefern, dann ist ihre Aufmachung schriller und moderner als bei den anderen Sendungen. Die Einspieler bei \textit{maybrit illner} und \textit{Menschen bei Maischberger} enthalten hingegen meist O-Töne oder Zitate von Politikern. Die der übrigen Sendungen sind eine relativ ausgewogene Mischung aus Einspielern mit O-Tönen und solchen in denen ein Offsprecher Informationen ausgibt\footnote{Zu \textit{Beckmann} kann hier allerdings keine Beurteilung abgegeben werden, da die Mehrheit der Einspieler in den vorliegenden Aufzeichnungen aus rechtlichen Gründen nicht enthalten sind.}. Am häufigsten werden bei log in Einspieler eingesetzt. In den beiden Folgen des Samples waren es ingesamt elf Stück. Danach folgen \textit{Anne Will} und \textit{hart aber fair} mit jeweils zehn Stück. \textit{Menschen bei Maischberger} setzt mit sechs Einspieler relativ selten auf dieses Sendungselement, genauso wie \textit{maybrit illner} und Beckmann mit jeweils fünf.

\subsection{Umgang mit Akteursgruppen}\label{chap:akteursgruppen}

Beleuchtet werden soll auch der Umgang mit den verschiedenen Akteursgruppen innerhalb der Sendungen. Wie bereits mehrfach erwähnt, haben sich im ersten Teil der Analyse einige – quantitative – Ungleichgewichte zwischen den verschiedenen Gruppen, insbesondere zwischen wirtschaftsnahen und arbeitnehmerfreundlichen respektive zivilgesellschaftlichen Vertretern gezeigt. Diese lassen sich auch auf der Ebene der qualitativen Analyse feststellen.

\subsubsection{Zivilgesellschaft}

Bereits in den Eröffnungssequenzen der Sendungen finden zivilgesellschaftliche Proteste gegen die Krisenpolitik kaum Erwähnung und gleiches gilt für den weiteren Verlauf der untersuchten Polittalks.

Zum einen kommen kaum Gäste aus der Zivilgesellschaft zu Wort. Nur in vier Folgen sind entsprechende Akteure zu Gast. Bei \textit{Beckmann} (27.10.11) und \textit{log in} (12.09.12) kommt jeweils ein Vertreter des sog. Regionalgeldmodells zu Wort. Bei \textit{maybrit illner} (13.10.11) ein Vertreter von Occupy Frankfurt und bei \textit{hart aber fair} (18.06.12) ein Attac-Aktivst. Nur bei \textit{hart aber fair} kann der entsprechend Gast als vollwertiger Teilnehmer der Runde agieren, bei den anderen drei Beispielen werden hingegen nur Einzelgespräche außerhalb der eigentlichen Talkrunde geführt.

Bei der erwähnten Folge von \textit{maybrit illner} wird der Vertreter von Occupy Frankfurt, Wolfgang Siener, zwar zu Beginn der Sendung circa fünf Minuten lang befragt, kommt dann aber in der ganzen Sendung nur ein weiteres mal kurz zu Wort und muss zudem im Publikum sitzen bleiben. Anders wird hingegen mit Thorsten Hinrichs, seines Zeichens Deutschlandchef der Ratingagentur Standard \& Poors, umgegangen. Er sitzt Anfangs ebenfalls im Publikum neben Siener und wird dort relativ kurz befragt, später allerdings kommt Illner auf ihn zurück und er wird – nach einem circa sieben minütigem Einzelgespräch an einem gesonderten Stehtisch – in die Runde der restlichen Talkgäste gebeten.

In diesem Fall und auch in dem Fall der beiden Advokaten des Regionalgelds wird die zivilgesellschaftliche Beteiligung auf die Funktion eines Stichwortgebers reduziert. Sie sollen Impulse für die „richtige” Gesprächsrunde liefern, als wirklich gleichwertige Gäste werden sie nicht behandelt.

Das Vertreter der Zivilgesellschaft nicht unbedingt ernster genommen werden, wenn sie mit in der Runde sitzen, zeigt der Umgang mit dem Attac-Aktivisten Alexis Passadakis in der \textit{hart aber fair} Folge vom 18. Juni 2012. Vorgestellt wird er mit den folgenden Sätzen:

\begin{description}
	\begin{linenumbers}[1]
		\item \#00:01:02-4\# \textbf{Off}: [\ldots] Alexis Passadakis, der Attac-Aktivist und Politologe klagt an: In Deutschland werden massiv Vorurteile über den Volkscharakter der Griechen verbreitet, das ist gefährlich, denn Deutschlands Dominanz in Europa ist schon ausgeprägt genug. ((Applaus)) \#00:02:26-2\# 
	\end{linenumbers}
	\captionof{transkript}{hart aber fair (18.06.2012)}
	\label{lis:15}
\end{description}

Der Zuschauer erfährt hier drei Dinge über Passadakis. Erstens, dass er aufgrund des Namens wahrscheinlich griechischer Abstammung ist und zweitens explizit, dass er sich wissenschaftlich mit Politik beschäftigt sowie drittens, dass er politisch zu Attac gehört. Authentizität qua Herkunft und Expertise liefern die Legitimation für seinen Auftritt. Desweiteren werden ihm zwei Positionen zugeschrieben. Nämlich, dass in Deutschland Vorurteile über Griechen verbreitet würden und dass Deutschland eine Vormachtstellung in Europa inne habe.

In der sich entfaltenden Diskussion versucht Passadakis immer wieder den Fokus auf die Strukturen zu legen, also auf das Wirtschaftssystem und seine Rolle in der Krise und weniger auf die Mentalität der Griechen oder Korruption und Vetternwirtschaft in Griechenland. Dabei argumentiert er häufig mit Zahlen und Fakten und unterscheidet sich so von den meisten seiner Diskussionspartner. Allerdings wenig erfolgreich, wie nachstehende Sequenz zeigt:

\begin{description}
	\begin{linenumbers}[1]
		\item \#00:26:50-5\# \textbf{Plasberg}: Ja Herr Passadakis, sie hoffen. Ja Herr Passadakis, dass haben sie ja eben klar gemacht äh das durch die starke Linkspartei, auch wenn sie in der Opposition ist, sich daran etwas ändert. Trauen sie dieser Partei zu genau die Strukturen zu schaffen die Herr Blome anspricht? Äh die dafür sorgen, dass die Einnahmeseite gestärkt wird, dass Steuern kommen und nicht nur auf der anderen Seite gekürzt wird. \#00:27:10-1\#
		
		\item \#00:27:10-1\# \textbf{Blome}: Er wollte nochmal die Beamte einstellen, ne? \#00:27:11-6\#
		
		\item \#00:27:10-5\# \textbf{Passadakis}: // Also es ist klar, //dass die äh sch/ // \#00:27:12-8\#
		
		\item \#00:27:12-1\# \textbf{Plasberg}: // Das hat er versprochen. //  \#00:27:12-7\#
		
		\item \#00:27:13-3\# \textbf{?}: Ja. \#00:27:13-4\#
		
		\item \#00:27:13-8\# \textbf{Passadakis}: Also es ist klar, dass die Schwesterpartei der CDU also die Nea Dimokratia, aber auch die Pasok, letztendlich für diese alte Politik von äh Steuerflucht äh stehen. Und für eine ja Steuerbefreiung, Steuererleichterung für die Reichen und Superreichen. Von der Seite aus wird nichts Neues kommen und ich denke das die Parteien in der Opposition, wie äh Dimar, die Demokratische Linke, und äh Syriza. Die Parteien sein könnten, die ein faires Steuersystem etablieren werden. Weil das ist natürlich deren Ziel. Auch Umverteilung zu organisieren in Griechenland. // Und dafür stehen die alten Parteien sicherlich nicht. \#00:27:42-6\#
		
		\item \#00:27:41-1\# \textbf{Plasberg}: // Herr Blome hat. // Entschuldigung, Herr Blome hat eben den Einwurf hier zurecht gemacht äh. Erstmal hat äh die haben die Linken versprochen mehr Beamte einzustellen. Das ist/ \#00:27:48-9\#
		
		\item \#00:27:48-9\# \textbf{Blome}: // Es geht mir nicht um die Umverteilung ($\ldots$) bevor sie es umverteilen können müssen sie es ja reinkriegen  und hier (unv).// \#00:27:56-0\#
		
		\item \#00:27:50-2\#  \textbf{Plasberg}: // Das scheint nicht das zu sein, was Griechenland braucht (.) mehr Beamte // \#00:27:52-7\#
		
		\item \#00:27:52-9\# \textbf{Passadakis}: // Also ich denke das Griechenland sicherlich einen vernünftig"-en öffentlichen // \#00:27:55-7\#
		
		\item \#00:27:55-6\# \textbf{Plasberg}: Der Reihe nach, Herr Blome! \#00:27:56-0\#
		
		\item \#00:27:56-7\# \textbf{Passadakis}: Das Griechenland sicherlich einen vernünftigen öffentlichen Dienst braucht und äh auch all diese Zahlen, die argumentieren, dass der öffentliche Dienst äh aufgebläht ist ähm stimmen mit der Realität nicht überein. \llabel{16:1} Es ist zwar so, dass in Griechenland viele Leute im öffentlichen Dienst arbeiten. Aber die Löhne sind im Durchschnitt so gering, dass man nicht davon sprechen kann, dass es einen aufgeblähten Staatsapparat gibt. Die meisten. 50 \% der Angestellten im öffentlichen Dienst verdienen nur den Mindestlohn, dass führt zu keiner Aufblähung der Staatsaufgaben. \llabel{16:2} // Und dieser Fokus/ // ((verhaltener Applaus))  \#00:28:24-1\#
		
		\item \#00:28:23-0\# \textbf{Plasberg}: // Herr Passadakis, sie werden gleich Frau Nettersheim // hören. \llabel{16:4} Die wird uns von einem Fahrer erzählen äh der im staatlichen Diensten war, was er verdient zu anderen. Also ich wäre vorsichtig mit so steilen Behauptungen.\llabel{16:3} \#00:28:33-6\#
		
		\item \#00:28:34-0\# \textbf{Passadakis}: // Nein. Das sind die Zahlen/ // \#00:28:35-1\#
		
		\item \#00:28:33-7\# \textbf{Blome}: Vor allen Dingen // was leistet denn ein öffentlicher Dienst // der so groß ist? Unabhängig von der Frage, wie hoch er bezahlt ist. Was kann ein öffentlicher Dienst produzieren? Was sie verkaufen können nachher im Ausland? Was? Ein Viertel der Angestellten in Griechenland ist beim Staat angestellt. Was machen die Leute? Völlig unabhängig davon ob sie viel oder wenig kriegen.  Machen sie irgendetwas was die Wirtschaft des Landes stärkt? Ich glaub nicht. \#00:28:53-9\# 
		
		\item \#00:28:53-9\# \textbf{Passadakis}: Also so wie der öffentliche Dienst im Moment organisiert ist und daran sind Nea Dimokratia und Pasok im wesentlichen Schuld, die ja vierzig Jahre regiert haben, liegt natürlich vieles im Argen. Das man aber mit einem gut aufgestellten öffentlichen Dienst äh ne vernünftige Ökonomie organisieren kann, da bin ich mir ziemlich sicher. Außerdem ist glaube ich dieser Fokus auf diese Probleme ähm in Griechenland viel zu stark und die systemischen Probleme werden nicht äh betrachtet. Denn wenn man sich genau anschu äh anschaut wie die Staatsverschuldung in Griechenland gestiegen ist, ist es ja insbesondere durch die Finanzkrise ab 2007, 2008. Bankenrettungsprogramme, Konjunkturprogramme, Einbruch der Steuereinnahmen. Da sind die Schulden dann tatsächlich explodiert und haben diese Dynamik ausgelöst. Zwei/  \#00:29:34-7\# 
		
		\item \#00:29:35-5\# \textbf{Plasberg}: Ich find' das interessant, dass sie diese Struktur"-de"-batte führen, wir haben ja zwei Menschen mit/ \#00:29:37-7\#
		
		\item \#00:29:37-7\# \textbf{Passadakis}: Natürlich. \#00:29:38-3\#
		
		\item \#00:29:38-3\# \textbf{Plasberg}: Ja, ist ja auch wichtig. Mit griechischem Namen hier. Er ((zeigt auf Cordalis)) kommt von der Volksseele und erzählt uns mit über"-zeugendem Beispiel von der Mentalität. Sie sagen es ist vor allen ein Strukturproblem. Auf der Suche nach der Wahrheit (.) möcht' ich jetzt mal zu Frau Nettersheim gehen. \#00:29:52-2\#
		
		\item $\ldots$
		
		\item \#00:32:16-0\# \textbf{Plasberg}: Der Sprengstoff der in Griechenland äh liegt. Die Lunte brennt ja schon. Besteht ja offenbar auch in dieser sozialen Ungerechtigkeit. Passadakis hat eben gesagt, dass mit der Überversorgung ähm im Staatsdienst gar nicht mehr so der Fall ist. Äh/ \#00:32:30-5\#
		
		\item \#00:32:30-5\# \textbf{Nettersheim}: \llabel{16:5}Herr Passadakis lebt nicht in Griechenland. \#00:32:32-4\# 
		
		\item \#00:32:32-4\# \textbf{Plasberg}: \llabel{16:8}Erzählen sie. Sie haben mir es schon erzählt, deswegen kenn' ich das Beispiel. Erzählen sie uns doch nochmal von dem Vergleich äh den sie in der weiteren Bekanntschaft haben.\llabel{16:9} Ich möchte dass sie da auch noch gerne nachhause kommen aber äh ((Nettersheim lacht)) Was kann man da an unterschiedlichen Zahlungen sehen? \#00:32:45-6\# 
		
		\item \#00:32:45-6\# \textbf{Nettersheim}: \llabel{16:11}Ja. Also Leute die in der Privatwirtschaft  arbeiten. Die ja hauptsächlich jetzt auch die Zeche zahlen, weil wir sind ja die die Steuern zahlen, kommen mit vergleichsweise geringen Renten aus. Staatsangestellte und dazu gehören auch Angestellte äh beim Strom beim staatlichen Stromkonzern oder beim bei den staatlichen Wasserwerken verdienen sehr viel mehr als Privatangestellte und haben auch relativ hohe Renten.\llabel{16:12} \#00:33:15-2\# 
		
		\item \#00:33:15-2\# \textbf{Plasberg}: \llabel{16:10} Sie haben uns das Beispiel eines Hilfsfahrers erzählt. \#00:33:17-3\# 
		
		\item \#00:33:17-3\# \textbf{Nettersheim}: Ja wir haben da jemanden in der Verwandtschaft, der war mehr oder weniger äh Hilfsfahrer. \#00:33:23-6\# 
		
		\item \#00:33:23-6\# \textbf{Plasberg}: Was heißt Hilfsfahrer? Was hat der gemacht? \#00:33:24-6\# 
		
		\item \#00:33:24-6\# \textbf{Nettersheim}: \llabel{16:6}Jaaa, der hat Kaffee von A nach B gefahren oder vielleicht mal ne' Kabelrolle. Der kriegt 45.000 Euro Rente und dann kenn' ich aber auch Leute die haben ihr Leben lang in der Privatwirtschaft gearbeitet, die müssen mit 700 auskommen. (..) Und da ist ne' große Ungerechtigkeit da und ich hoffe, dass da was geändert werden kann.\llabel{16:7} \#00:33:47-4\#
	\end{linenumbers}
	\captionof{transkript}{hart aber fair (18.06.2012)}
\end{description}

Alexis Passadakis versucht hier der Vorstellung vom aufgeblähten Staatsapparat mit dem Verweis auf das niedrige Lohnniveau der Staatsangestellten zu entkräften und nennt hierzu konkrete Zahlen (Z. \lref{16:1}-\lref{16:1}). Plasberg wiederum zieht diese „steilen Behauptungen“ (Z. \lref{16:3}) nicht mit Fakten in Zweifel, sondern verweist auf den Erfahrungsbericht einer kurz danach auftretenden Deutsch-Griechin (Z. \lref{16:4}f.) – die abstrakten Zahlen sollen also mit dem konkreten Einzelfall widerlegt werden. Nach einem kurzen Austausch zwischen Blome und Passadakis leitet Plasberg zur Deutsch-Griechin Nettersheim damit über, dass diese, als jemand am „Puls des Geschehens“, die wahre Ursache der Probleme Griechenlands – Strukturen oder Mentalität – erhellen werde. Die Glaubwürdigkeit des Arguments von Passadakis wird dabei im Folgenden in zwei Schritten unterminiert. Zuerst indem Nettersheim feststellt, er lebe nicht in Griechenland (Z. \lref{16:5}), ihm fehle folglich die Grundlage zur Beurteilung der Situation. Zum zweiten durch das Beispiel des „Hilfsfahrers“, der eine üppige Rente für eine unanstrengende und einfache Arbeit bekäme (Z. \lref{16:6}-\lref{16:7}), das emotional an das Gerechtigkeitsempfinden der Zuschauer appelliert.

Es ist genau dieser Einzelfall – ob paradigmatisch oder nicht wird im weiteren Verlauf nicht geklärt – auf den Plasberg hinarbeitet (Z. \lref{16:8}-\lref{16:9}). Er insistiert darauf (Z. \lref{16:10}) auch als Nettersheim bereits eine allgemeinere aber sinngemäß gleichwertige Antwort gegeben hat (Z. \lref{16:11}-\lref{16:12}). Genau genommen belegt oder widerlegt der Einzelfall nichts. Stattdessen wird die abstrakte Diskussion, über die von Passadakis genannten Zahlen, von Plasberg vermieden, indem er auf das konkrete Einzelbeispiel ausweicht. Dem Moderator selbst ist das auch bewusst, stellt er doch kurz danach in Bezug auf die Aussagen zu einem anderen Thema fest: „Das was sie empfinden das sind immer natürlich Momentaufnahmen.“. Auch im sogenannten Faktencheck werden die Zahlen im Nachhinein nicht nochmal aufgegriffen \parencite{o.a.FaktencheckHartAber2012}.

Letztlich führt das Abkanzeln von Passadakis „steiler Behauptung“ durch Plasberg  und die Widerlegung mittels eines anschaulichen Einzelfalls zum Gegenteil dessen was in der Eröffnung noch als Ziel ausgegeben wurde. Statt Vorurteile abzubauen, bestätigt er sie. Hinzu kommt, dass Plasberg mit dem BILD-Mann Blome nicht so hart ins Gericht geht, obwohl dessen Zeitung regelmäßig mit SchlagZ.n wie „Schummel-Griechen machen unseren Euro kaputt“\footnote{BILD-Zeitung vom 2. März 2010.} glänzt.

\subsubsection{„Normalbürger“}

Damit wird auch zugleich die Rolle klar, die Normalbürger spielen. Ihre Funktion ist es die Debatte über – insbesondere bei der Eurokrise zu gewissem Teil notwendigerweise – abstrakte Sachverhalte mit ihrer eigenen konkreten Lebenswirklichkeit zu konfrontieren. Dies kann sinnvoll sein, muss es aber nicht, wie das obenstehende Beispiel zeigt.

Die „normalen“ Talkgäste und die eingeladenen Bürger können sich durchaus auf Augenhöhe bewegen und zwar dann, wenn sie gleichberechtigt in die Gesprächsrunde integriert werden. Im konkreten Fall wurde Frau Nettersheim nach dem Einzelgespräch zu den anderen Gästen gebeten und konnte weiter am Gespräch teilnehmen. Die Feststellung, dass der „'Betroffene' [\ldots]  immer nur als Betroffener in eigener Sache zu sprechen [hat] und auf keinen Fall als gleichberechtigter Citoyen mitzureden“ \parencite[7]{gaeblerUndUnserenTaeglichen2011} trifft folglich nicht immer zu. Frau Nettersheim kann solange als „gleichberechtigter Citoyen“ agieren, wie sie mit den anderen Gästen in einer gemeinsamen Runde sitzt. Bereits zuvor hat sie im Einzelgespräch versucht diese Rolle einzunehmen, Plasberg tat allerdings sein Bestes, um sie dort auf die Rolle als „Betroffene in eigener Sache“ zurückzustutzen (vgl. Transkript \vref{lis:17}; Z. \lref{17:1} und \lref{17:2}f.).

\begin{description}
	\begin{linenumbers}[1]
		\item \#00:33:47-7\# \textbf{Plasberg}: Über die einen Fälle, die sie jetzt geschildert haben, auch der Überversorgung, der Verschwendung, des Nicht-Sparen-Wollens wird in Deutschland breit berichtet. (.) Können sie verstehen, dass sich die Begeisterung über weitere Hilfspakete. Vielleicht über noch mehr Schecks äh die jetzt auch nach diesem Wahlergebnis Richtung Griechenland gehen, dass sich die Begeisterung da in Grenzen hält? Um es vorsichtig zu sagen. \#00:34:07-3\#
		
		\item \#00:34:07-6\# \textbf{Nettersheim}: Ja. (..) Ich verstehe (.) meine deutschen Landsleute, ich kann aber auch die Griechen verstehen. Und da/  \#00:34:14-6\#
		
		\item \#00:34:15-5\# \llabel{17:1} \textbf{Plasberg}: Bleiben wir bei bei ihrer deutschen Rolle. \#00:34:17-0\#
		
		\item \#00:34:16-0\# \textbf{Nettersheim}: // Ja. //  \#00:34:16-2\#
		
		\item \#00:34:17-0\# \textbf{Plasberg}: Ich. Sie sprechen griechisch äh aber ich nehme an, dass man immer noch einen Akzent raushört. ((Nettersheim nickt)) \llabel{17:2}Wie ist das denn, wenn man in diesen Tagen durch Athen. Wir hören ja immer im Land auf den Insel ist das nicht so schlimm. Aber wenn man in Athen. Da wo sie leben. Äh als Deutscher auffliegt. Was passiert dann? \#00:34:30-1\#
		
		\item \#00:34:30-5\# \textbf{Nettersheim}: Also, man muss da ganz klar unterscheiden. Der deutsche Bürger und die Regierung Merkel sind zwei paar Schuhe. Und da find' ich wurde auch von den Medien 'ne Feindlichkeit äh (.) äh (.) da wird 'ne Feindlichkeit äh (.) dargestellt die in dem extremen Maße nicht existiert. \#00:34:49-9\#
		
		\item \#00:34:50-3\# \textbf{Plasberg}: Mhm. Wenn sie blöd angemacht werden zum Beispiel für Frau Merkel, halten sie dagegen? \#00:34:53-4\#
		
		\item \#00:34:53-6\# \textbf{Nettersheim}: Ja. \#00:34:54-1\#
	\end{linenumbers}
	\captionof{transkript}{hart aber fair (18.06.2012)}
	\label{lis:17}
\end{description}

\subsubsection{Prominente}

Dass es nicht immer eine gute Idee ist prominente Personen aus Kultur und Sport zu jedem politischen Thema einzuladen, zeigt sich ebenfalls deutlich in der im Abschnitt zuvor bereits näher betrachteten Sendung von \textit{hart aber fair}.

Geladen hatte Plasberg unter anderem auch den Schlagersänger Costa Cordalis. Vorgestellt wird dieser wie folgt:

\begin{description}
	\begin{linenumbers}[1]
		\item \#00:01:02-4\# \textbf{Off}: [\ldots] Costa Cordalis, der Musiker steht im ständigen Kontakt mit seiner Familie in Griechenland und sagt, hierzulande wird leider ein falsches Bild von meiner Heimat vermittelt. Denn Griechen sind weder Deutschenfeindlich, noch faul. Wir sind ein stolzes Volk das bisher noch jede Krise gemeistert hat. [\ldots] \#00:02:26-2\#
	\end{linenumbers}
	\captionof{transkript}{hart aber fair (18.06.2012)}
\end{description}

Auf der Suche nach einer auch in Deutschland bekannten griechischstämmigen Person stieß die Redaktion wohl auf Cordalis. Ebendiese Herkunft und der Umstand, dass ein Teil seiner Familie in Griechenland lebt, qualifizieren ihn offenbar für das Thema Eurokrise. Aus der Vorstellung wird deutlich, dass die ihm zugedacht Funktion ist, die in der Diskussion um die Eurokrise in Deutschland verbreiteten Ressentiments über Griechenland mit der Realität zu kontrastieren. Da in der Runde auch der stellvertretende BILD-Chefredakteur zugegen ist, sollte eine konfrontative Diskussion garantiert sein.

Dazu kommt es allerdings nicht. Cordalis ergreift nur selten das Wort und wenn dann sind seine Äußerungen meist derart belanglos, dass eine weitere Diskussion sich von selbst erübrigt. Besonders drastisch ist das in der nachstehenden Sequenz sichtbar, in der Cordalis zweimal hintereinander auf die Fragen von Plasberg Antworten gibt, die in diesem Zusammenhang keinen Sinn ergeben:

\begin{description}
	\begin{linenumbers}[1]
		\item \#01:11:29-2\# \textbf{Plasberg}: Da sitzt aber auch schon eine ganze Generation auf gepackten Koffern. Es gibt Sprachkurse in Griechenland und anderen Ländern die haben den Spitznamen Nichts-wie-weg-Kurse. Herr Cordalis, wie gefällt ihnen das als Frühauswanderer aus äh Griechenland, dass Deutschland ähm einerseits äh nicht grade in der Krise geschätzt wird, aber andererseits das gelobte Land ist, um eine Rettungslebensperspektive für sich und Kinder zu sehen? \#01:11:51-4\#
		
		\item \#01:11:53-1\# \textbf{Cordalis}: Ich bin mit sechzehn von Griechenland weggegangen. Hab hier das Goethe-Institut besucht, die deutsche Sprache erlernt, dann fünf Jahre Musik studiert. (.) Dann war gleich mein Sohn unterwegs da hab ich geheiratet. Als ein Mann und heute bin ich siebzehn Mann. (.) Hab meine drei Kinder, die haben wieder geheiratet und dieses Jahr haben alle Zwilli Zwillinge bekommen. (.) Und plötzlich (.) Die Eltern von den, ja von den Schwiegersöhnen und so weiter, da sind wir zusammen siebzehn Mann. // Das ist eine Macht, eine Kraft. //  \#01:12:25-7\#
		
		\item \#01:12:24-8\# \textbf{Plasberg}: Ne starke Fraktion. (.) Wenn es jetzt wieder eine Einwanderungswelle gibt und äh zum Beispiel Griechenland äh die Besten, die Jungen, die da keine Perspektive mehr sehen aber eine akademische Ausbildung gemacht haben. Die aber auch verwerten wollen. Wenn die zu uns kommen, dass wäre doch eine Doppelbestrafung für dieses Land.  \#01:12:40-9\#
		
		\item \#01:12:41-8\# \textbf{Cordalis}: Zum Beispiel in New York. Da sind fertige Rechtsanwälte, Ärzte und was verkaufen sie dort mit diesen Wägen? Frankfurter Würstchen. \#01:12:50-7\# 
	\end{linenumbers}
	\captionof{transkript}{hart aber fair (18.06.2012)}
\end{description}

Dies ist ein klassisches Beispiel für die Einladung eines Gastes nur wegen seiner Prominenz und einer aufgrund äußerlicher Merkmale unterstellten Sachkompetenz, die sich bei genauerem Hinsehen als falsch entpuppt\footnote{Die Einladung von Costa Cordalis ist kein Einzelfall. In den 47 Talkfolgen zur Eurokrise waren fünf Gäste aus dem kulturellen Bereich eingeladen – fast alle Deutsche griechischer Abstammung: Hans Christoph Buch, Costa Cordalis (2x), Hermes Hodolidis und Vicky Leandros.}. In solchen Fällen wäre es sinnvoller ganz auf die Einladung zu verzichten und dann eventuell mit weniger Personen die Sendung zu bestreiten.

\section{Diskurstopoi}\label{chap:diskurstopoi}

Im Methodenteil wurde bereits erläutert dass die Erschließung der Diskussionsinhalte mittels zweier Leitfragen – nach Ursache und Lösung der Eurokrise – erfolgt (vgl. Kapitel \vref{chap:qualitativeanalyse}). Im Weiteren werden zuerst die Diskurstopoi zur Frage nach den Ursachen der Krise dargestellt, danach die zur Frage nach den möglichen Lösungen.

\subsection{Ursachen der Krise}

Es lassen sich grob vier in den Talks diagnostizierte Ursachentopoi für die Eurokrise unterscheiden. Zum einen wird die Schuld an der jetzigen Situation der Politik in den betroffenen Staaten zugeschoben. Zum anderen wird die Krise als Folge der Finanzkrise und der verbundenen Rettung von Banken gesehen. Drittens wird der Grund in einem grundsätzlichen Konstruktionsfehler des europäischen Währungssystems ausgemacht und viertens vertreten einige Talkgäste die Ansicht die Ursache liege im gegenwärtigen Wirtschaftssystem an sich.

Die vier Topoi sind nicht vollkommen trennscharf zu unterscheiden, teils benutzen Gäste Argumente aus verschiedenen Topoi. Dennoch geben sie unterschiedliche Ebenen an, auf denen die Ursache für die Eurokrise verortet wird und zeichnen damit auch verschiedene Lösungswege vor.

\subsubsection{Politik der Staaten}

Der am häufigsten geäußerte Erklärungsansatz sieht die Krisenursache in der Politik der einzelnen Nationalstaaten. Diese hätte aufgrund verantwortungslosen Handelns so viele Schulden angehäuft, dass es nun zur Krise gekommen sei. Beigetragen dazu habe auch, dass die Politiker Regeln ignoriert hätten, etwa die sogenannten Stabilitätskriterien des Maastrichtvertrags. Es handele sich folglich nicht um eine Banken- oder Finanzmarktkrise sondern um eine Staatsschuldenkrise. Exemplarisch sei hier die Argumentation von Ursula von der Leyen zitiert:

\begin{description}
	\begin{linenumbers}[1]
		\item \#00:26:57-1\# \textbf{Leyen}: Also, wir hörten eben, dass sie zumindest sagen, dass man dann auch mal einen Cut machen muss. Einen Schuldenschnitt haben sie gesagt muss man machen. Das heißt aber für Investoren und das ist ja das schwierige an der Situation, dass die sagen nie wieder in europäische Staatsanleihen investieren. Denn ähm die sind überhaupt nicht sicher! Wenn ich mir anschaue wie die Staaten wirtschaften, dann werde ich doch darein nicht investieren. Das ist unsere Krux. ((Gysi nuschelt etwas unverständlich vor sich hin)) \llabel{20:1}Wir haben lange über unsere Verhältnisse gelebt, wir haben uns lange verschuldet. Wir unterschiedlich jetzt in den äh europäischen Ländern, aber wir haben lange so gelebt als könnte es so weiter gehen. Viel Geld ausgeben, über unsere Verhältnisse leben und nicht wirklich wirtschaftlich soweit uns modernisieren, dass wir der Globalisierung auch Stand halten können.\llabel{20:2} Jetzt und das ist das anstrengende an der Globalisierung sagen, testen die Märkte. Auch die sagen ihr haltet nicht zusammen als Eurozone. Wie die Piranhas fallen wir über euch her. Wir brechen erst Griechenland raus, dann Portugal dann Irland, dann geht's weiter. Und zweitens wir glauben euch nicht, dass ihr von euren Schulden wieder runterkommt und deshalb Zinsen rauf. Wir kaufen eure Staatsanleihen nicht. Dann können wir doch nicht sagen, unterschwellig, das ist richtig, man kann unseren Staatsanleihen nicht trauen. Denn wir werden. Sie werden nichts mehr wert sein. Wir werden Schuldenschnitt machen und dann werdet ihr eure Geld verlieren. \#00:28:14-7\# 
	\end{linenumbers}
	\captionof{transkript}{Anne Will (12.09.2012)}
\end{description}

Zwar spricht Leyen hier auch von „den Märkten“ als aktiven Akteuren, die Schuldzuschreibung liegt aber eindeutig auf den Staaten, die über ihre Verhältnisse gelebt hätte (Z. \lref{20:1}-\lref{20:2}). Ähnlich argumentiert auch der ehemalige Vorstandsvorsitzende der Dresdner Bank Bernhard Walter:

\begin{description}
	\begin{linenumbers}[1]
		\item \#00:04:45-9\# \textbf{Plasberg}: Sie waren bis vor zweieinhalb Jahren, wir haben es gesagt, Vorstandsvorsitzender der Dresdner Bank. Mal ganz ehrlich. Haben sie sich zu ihrer Zeit, als sie danach im Job waren, mal heimlich gewünscht, dass genau das jetzt passiert. Das so 'ne riesige Schuldenblase. Die müssten sie ja gesehen haben eigentlich, dass die mal mit so 'ner griechischen Nadel gepikst wird? \#00:05:03-9\#
		
		\item \#00:05:04-9\# \textbf{Walter}: Gut, wenn sie zurück gucken. Äh. Man ist natürlich immer wenn ma aus der Kirche rauskommt noch etwas klüger als vorher. Wenn sie insgesamt über einen längeren Zeitraum hingucken, dann hat sich natürlich nicht nur eine \llabel{21:2}Schuldenblase entwickelt. Nicht nur die amerikanische Immobilienblase. Sondern eben auch wenn sie unsere Verschuldung, Herr Solms hat's grad erwähnt äh, auch in Deutschland sehen. Obwohl wir international ja sehr gut angesehen äh sind, wenn sie das, diese ja \llabel{21:1}antizyklische Konjunkturpolitik über 30, 40 Jahre sehen. \llabel{21:3}Dann hat sich da natürlich schon was aufgetürmt und das ist natürlich etwas wo die Märkte bis zu einem gewissen Punkt mitmachen und ab irgendeinem Punkt sagen, bis hierher und nicht weiter. Und das ist halt bei ein paar Peripheriestaaten jetzt passiert. \#00:05:47-8\#
	\end{linenumbers}
	\captionof{transkript}{hart aber fair (24.10.2011)}
\end{description}

Auch hier ist die Politik die Schuldige, konkret die „antizyklische Konjunkturpolitik“ (Z. \lref{21:1}), die eine „Schuldenblase“ (Z. \lref{21:2}) hervorgebracht habe. Diese Blase sei inzwischen so groß, dass „die Märkte“ nicht mehr mitmachen wollten da ihnen das Vertrauen fehle (Z. \lref{21:3}f.). Dies erscheint, genauso wie bei Leyen, als natürliche und logische Folge.

Wie es in den beiden aufgeführten Beispielen schon anklingt, wird diese Position vornehmlich von Politikern der Regierungsparteien sowie von Wirtschaftsvertretern eingenommen. Aber auch einige SPD-Politiker und Wirtschaftswissenschaftler äußern derartige Argumente. Finden lässt sich der Topos in dreizehn der vierzehn Folgen und ist in diesen zumeist der am häufigsten geäußerten\footnote{Der naheliegende Versuch anhand des Applauses des Studiopublikums festzumachen, welches Topos sich am ehesten durchgesetzt hat, ist nicht zielführend, da in zahlreichen Folgen vollkommen „willkürlich“ für gegensätzlichste Argumente applaudiert wurde.}.

Eine Besonderheit dieses Topos ist es, dass er teils auch von den Moderatoren selbst vertreten wird. Ein Beispiel haben wir bereits näher betrachtet (vgl. Transkript \vref{lis:15}), ein weiteres findet sich in der \textit{log in} Folge vom 26. Oktober 2011 (Z. \lref{22:1}):

\begin{description}
	\begin{linenumbers}[1]
		\item \#00:40:44-1\# \textbf{Wolf}: (unv.) Sensationell. Zwei Minuten und wirklich fast alles ich glaube twitterbar. Äh ähm. Ich hätte jetzt 'ne schöne Überleitung, die lassen wir an dieser Stelle. So. Also. Man kann lange über Banken diskutieren und über Rettungsschirme philosophieren, so wie wir das in dieser Sendung gemacht haben. \llabel{22:1}Aber Herr Altmaier hat's auch schon angesprochen an einem Punkt, das größte Problem ist noch nicht gelöst, nämlich, dass zu viele Politiker noch nicht gelernt haben Schulden zu vermeiden.  \#00:41:08-9\#
	\end{linenumbers}
	\captionof{transkript}{log in (26.10.2011)}
\end{description}

\subsubsection{Banken und Finanzmärkte}

Die am zweithäufigsten geäußerte Ursachenvermutung schreibt die Schuld hingegen den Banken und Finanzmärkten zu: die Banken hätten sich in der Finanzkrise 2008 an den Finanzmärkten verspekuliert. Durch die damalige und bis heute weiter fortgesetzte Rettung dieser Banken vor der Pleite hätten sich die Staaten dann so stark verschuldet, dass die Krise quasi auf sie übergesprungen sei. Die Finanzmärkte nutzten diese Situation nun zudem aus und würden bewusst gegen den Euro vorgehen. Im Folgenden findet sich für dieses Argumentationsmuster wiederum ein Beispiel aus dem Sample (vgl. Transkript \vref{lis:23}).

Zwar werden hier die Staatsschulden als Teil des Problems anerkannt (Z. \lref{23:1}), allerdings sei dies nur ein Symptom. Die eigentliche Ursache seien die spekulierenden Banken die gerettet werden mussten (Z. \lref{23:2}-\lref{23:3}) und weiterhin gerettet würden (Z. \lref{23:4}-\lref{23:5}). Die Politik haben hierzu insofern ihren Teil beigetragen, als sie die Finanzmärkte zu stark dereguliert habe.

Vertreter dieser Linie sind in den Sendungen, wenig überraschend, Politiker der Opposition, vor allem der Linkspartei. Aber auch der FDP-Abweichler Frank Schäffler, der alle Argumente aus allen vier Ursachentopoi vorbringt, und das CDU-Mitglied Heiner Geißler sehen hier zumindest eine Teilschuld – beides allerdings Personen die nicht unbedingt für Kadertreue zu ihren Parteien bekannt sind. Zu den prominenten Verfechtern gehört ebenso der „Börsenexperte“ Dirk Müller. Er ist auch der einzige der in zwei Talks die Verflechtung zwischen Bankenlobby und Politik anspricht, ein Aspekt der sonst vollständig fehlt.

\begin{description}
	\begin{linenumbers}[1]
		\item \#01:01:11-0\# \textbf{Maischberger}: Frau Wagenknecht, sie sagen ja immer im Prinzip das, was jetzt vereinbart wird, ist nur dafür dafür gemacht damit die Banken wieder gerettet werden. Ähm. Er sagt aber jetzt. Herr Haasis sagt ja die Banken sind ja bereit was zu tun, aber eben nicht die gesamte Verantwortung zu übernehmen, die eigentlich andere tragen müsste. Hat er recht? \#01:01:25-0\#
		
		\item \#01:01:25-5\# \textbf{Wagenknecht}: Ja was heißt die gesamt Verantwortung? Ich mein die Banken verdienen an den Schulden. Die verdienen an sehr hohen Zinsen, die sie nehmen. Und Zinsen sind äh, das wird Herr Henkel als Marktwirtschaftler bestätigen, ein Preis für Risiko. Das heißt, wenn ich an großen Zinsaufschlägen verdiene, dann muss ich auch am Ende äh das Risiko eingehen und auch schultern, dass dieses Risiko etwas kostet. Dass es nämlich einen Schuldenschnitt gibt und das Problem ist ja. \llabel{23:4}So wie wir jetzt rangehen, retten wir eben nicht nur die Banken, wir retten auch die Spekulanten, die Hedgefonds. Wir retten alle die sozusagen diese hochverzinsten Griechenpapiere gekauft haben. Das find' ich komplett falsch und selbstverständliche glaube ich auch, dass, also 20 \% Schuldenschnitt heißt ja auch wieder, dass der Steuerzahler den Banken Verluste abnimmt. Weil auf dem Markt sind diese Papiere zur Zeit 50 \% wert. Das heißt bei dieser sogenannten Gläubigerbeteiligung werden den Banken eben statt 50 \%, die sie am Markt für ihre Papiere noch bekommen, von der öffentlichen Hand garantiert 80 \% abgesichert. Das ist völlig absurd. Das ist öffentliches Geld, was eben dann Herrn Ackermann und anderen in den Rachen geworfen wird und immer wieder sozusagen mit dieser Schaumschlägerei begründet wird, ja wenn ihr uns jetzt nicht sichert und rettet, dann entsteht der große Flächenbrand.\llabel{23:5} Und parallel dazu machen die Banken weiter Gewinne. Sie haben in diesem Jahr, allein in den letzten vier Quartalen, 40 Milliarden an Dividenden ausgeschüttet. Also wenn es so brenzlig wäre, dann hätten sie mal diese 40 Milliarden vielleicht einbehalten als Eigenkapital. Dann würden sie etwas besser dieser Krise gegenüberstehen. \llabel{23:2}Und dann find' ich auch noch, das ist am Anfang kurz angesprochen worden, wenn wir über Staatsschuldenkrise reden, dann müssen wir eben auch immer dazu sagen, \llabel{23:1}diese Staatsschuldenkrise in Europa ist auch eine Folge der Bankenkrise von 2008, 2009. Das heißt, wenn sie jetzt die Banken hinstellen und sagen, wir haben mit dem Problem ja nichts zu tun, die Staaten haben sich ja alle überschuldet. Also dann muss man bitteschön über die Billionen an Rettungspaketen reden, die damals aufgelegt wurden. Die allein in Deutschland über 300 Milliarden mehr an Schulden geführt haben. In anderen Ländern. Irland zum Beispiel, ist nur wegen der Bankenrettung in Probleme gekommen.\llabel{23:3} Also das find' ich auch viel zu einfach. Und dafür versteh ich auch nicht wieso Herr Haasis, der nun wirklich einen Sketor in Deutschland vertritt, den Sparkassensketor, der überhaupt nicht so rumgezockt hat. Wieso er sich jetzt so vor die privaten Banken stellt, da hat er eigentlich überhaupt keinen Grund dazu. \#01:03:36-4\#
	\end{linenumbers}
	\captionof{transkript}{Menschen bei Maischberger (18.10.2011)}
	\label{lis:23}
\end{description}

\subsubsection{Euro}

Beim dritthäufigsten Topos wird der Grund für die aktuelle Krise in der gemeinsamen Währung, dem Euro, gesehen. Bei genauerem Besehen zerfällt er in unterschiedliche Einzelbegründungen und entsprechend finden sich Vertreter dieses Muster quer durch die verschiedenen politischen Lager.

Zum einen wird das Grundproblem im Fehlen einer vollwertigen Union gesehen. Bisher gebe es nur eine Währungsunion was allein zu wenig sei, denn es fehle beispielsweise an Durchgriffsrechten der Europäischen Union, um die Haushaltspolitik der Mitgliedsstaaten wirksam kontrollieren zu können. Im untenstehenden Beispiel benutzt Leyen zur Umschreibung dieses Umstands die Metapher des europäischen Hauses, dem entscheidende Pfeiler fehlten (Z. \lref{24:1}-\lref{24:2}). Im Endeffekt ergänzt diese Argumentation das erste vorgestellt Topos – die ungezügelte Ausgabenpolitik der Nationalstaaten.

\begin{description}
	\begin{linenumbers}[1]
		\item \#00:04:53-7\# \textbf{Leyen}: Es wäre ohne sie gegangen, denn das Entscheidende ist ja, dass es uns gelingt in dieser Krise. \llabel{24:1}Die ja ausgelöst worden ist, weil wir die Währungsunion auf der einen Seite haben, aber ähm alle anderen Pfeiler, die dazu nötig gewesen wären. Wenn wir uns das als Haus vorstellen. Die Währungsunion ist ein Pfeiler, aber der Pfeiler Haushaltsdisziplin, der Pfeiler Wettbewerbsfähigkeit, aber auch der Pfeiler was machen wir eigentlich wenn Banken oder Länder ins Trudeln geraten, wie sind die Regeln die wir dann gemeinsam einhalten.\llabel{24:2} Diese Pfeiler werden jetzt ins europäische Haus eingezogen und das ist die Vision für Europa, die wir gemeinsam vortragen, das ist ja das Entscheidende hier auch Rückenwind zu erhalten. \#00:05:28-1\#
	\end{linenumbers}
	\captionof{transkript}{Anne Will (12.09.2012)}
\end{description}

Das andere Argument legt den Fokus darauf, dass die Währungsunion die wirtschaftlichen Ungleichheiten zwischen den nationalen Volkswirtschaften enorm verstärkt habe. Dieser strukturelle Konstruktionsfehler der Eurozone habe letztlich dazu geführt, dass die Staaten in die Krise gerieten. Im folgenden Transkript führt diese Argumentation exemplarisch der Euroskeptiker Hans-Olaf Henkel aus (Z. \lref{25:1}f.). Da die möglichen Schlussfolgerungen relativ offen sind, wird diese Position aber beispielsweise auch vom Attac-Aktivist Passadakis vertreten (vgl. Transkript \vref{lis:37}).

\begin{description}
	\begin{linenumbers}[1]
		\item \#00:38:13-2\# \textbf{Henkel}: [$\ldots$] Erstmal möcht' ich auf folgendes hinweisen. Bleiben sie mal bei Griechenland. Machen sie den Schuldenschnitt. Entlasten wir Griechenland völlig von Schulden (..) Das hilft der griechischen Industrie überhaupt nicht, die wird nicht wettbewerbsfähiger. Es hat in der ganzen Geschichte von Entschuldung, die letzten waren die berühmten in Argentinien und Russland, nicht eine einzige Staatsentschuldung gegeben ohne Abwertung. Griechenland kann nicht wettbewerbsfähig werden und die anderen Länder, und das haben sie ja sehr schön vorhin selbst entwickelt, sind auch nicht mehr wettbewerbsfähig. \llabel{25:1}Warum? Weil ihnen die Fähigkeit der Justierung der Währung, sprich Abwertung verloren gegangen ist durch eine Einheitswährung. Die ersten die von diesem, von einem abgewerteten Euro profitieren würden, wären genau diese Länder die heute nicht mehr wachsen, weil sie nicht mehr konkurrenzfähig sind. Und was die Aufwertung in Deutschland betrifft, so wäre das Maximum was sie sich vorstellen können das was in Schweden passiert ist. Und Schweden ist ein Land was in der Wirtschaftsstruktur unserer durchaus ähnlich ist. Das wächst schneller. Es hat eine geringere Inflation, hat eine geringere Staatsverschuldung und es exportiert auch noch. \#00:39:17-3\# 
	\end{linenumbers}
	\captionof{transkript}{Menschen bei Maischberger (18.10.2011)}
\end{description}

Die dritte in diesem Zusammenhang vorgebrachte Begründung geht von einer Art volkspsychologischen Inkompatibilität der Mitgliedsländer des Euro aus. Die Länder passten von ihrer Mentalität her einfach nicht zusammen, weswegen das Projekt Euro scheitern musste. Vertreten wird diese psychologisierende und naturalisierende These im Untersuchungssample nur von Arnulf Baring, einem bekennende Konservativen (vgl. Transkript \vref{lis:26}; Z. \lref{26:1}-\lref{26:2}).

\begin{description}
	\begin{linenumbers}[1]
		\item \#00:03:21-5\# \textbf{Baring}: Naja das ganze Unternehmen des Euro hab ich ja von Anfang an für eine verfehlte Idee gehalten. Eine politische Idee, die in der Wirtschafts- und Finanzpolitik keine Grundlage hatte. \llabel{26:1}Ich habe ja vor (.) vor sechszehn Jahren geschrieben, bevor der Euro kam, dass er schief gehen müsse und zwar deshalb schief gehen müsse, weil in Europa die Leistungsfähigkeit der Beteiligten, wie die Mentalität der Beteiligten so verschieden seien, dass sie sich einer einheitlichen, noch dazu der deutschen Vorstellung, was das heiße und wie man sich zu verhalten habe, nicht entspreche.\llabel{26:2} \#00:03:53-5\#
	\end{linenumbers}
	\captionof{transkript}{Menschen bei Maischberger (08.05.2012)}
	\label{lis:26}
\end{description}

Den einzelnen Argumentationssträngen gemein ist, dass sie die Schuld nicht bei den Banken oder Finanzmärkten sehen, sondern beim gemeinsamen Währungssystem an sich und damit letztlich wieder in Versäumnissen der Politik. Diesmal allerdings in gesamteuropäischem Maßstab und nicht bloß bei einzelnen Nationalstaaten.

\subsubsection{Wirtschaftssystem}

Am seltensten findet sich bei der Frage nach den Ursachen der Krise eine Rückführung auf die allgemeine Organisation und Funktionsweise des Wirtschaftssystems, also den Kapitalismus. Statt den einzelnen Nationalstaaten, der Währungsunion oder den Banken und Finanzmärkte die Schuld zuzuweisen, wird in diesem Topos auf die allgemeine Krisenhaftigkeit des Kapitalismus hingewiesen (vgl. Transkript \vref{lis:27}; Z. \lref{27:1}-\lref{27:2}).

\begin{description}
	\begin{linenumbers}[1]
		\item \#01:00:24-4\# \textbf{Müller}: Also ich bin nicht der Meinung, dass wir das Geld abschaffen werden, abschaffen sollten. Was sich klasse finde und äh das find' ich prima an dem was er auch macht ((gemeint ist Hörmann)). Äh. Er zeigt in, auf extreme Weise äh Alternativen auf und das fäng, äh, führt dazu, dass man drüber nachdenkt. Er sitzt hier in der Sendung. Millionen schauen das. Sie denken drüber nach, Moment mal, wenn's andere Alternativen gibt, welche Probleme haben wir in unserem jetzigen System. Ich glaube nicht, realistisch, dass wir das Geld abschaffen werden. Wir werden Veränderungen erleben. Ich hoffe, dass sie kommen. Ich hoffe, dass wir unser Geldsystem irgendwann so optimieren, dass es wirklich für die Masse der Menschen dauerhaft ein sinnvolles Zusammenleben ermöglicht. \llabel{27:1}Im Moment ist es tatsächlich ein perfides System mit, wo über Jahrzehnte die Masse der Menschen enteignet wird zum Wohl von wenigen. Äh. Das funktioniert ein paar Jahrzehnte wunderbar und dann kommt es immer wieder zu der Situation in der wir jetzt stehen und deshalb sind all das was wir momentan erleben, auch in Brüssel, Rückzugsgefechte. Wir kommen um diese große Umverteilung, die alle paar Jahrzehnte kommt, nicht herum. Die einzige Frage zu diskutieren ist wie.\llabel{27:2} Ob's am Ende irgendwann mal zu einem Geldlosen System kommt. Ich glaube es nicht. \#01:01:25-1\# 
	\end{linenumbers}
	\captionof{transkript}{Beckmann (27.10.2011)}
	\label{lis:27}
\end{description}

Die so geäußerte Kapitalismuskritik erschöpft sich allerdings meist in einer recht ober"-fläch"-lich"-en Kritik des Geld-, wie im obigen Beispiel, oder Zinssystems. Dies liegt auch daran, dass dieser Topos nur von sehr wenigen Gästen vorgebracht wird. Dazu gehört unter anderem der häufig eingeladene Dirk Müller, von dem auch das Beispiel stammt, aber auch ein Vertreter des Regionalgeldmodells\footnote{Auch wenn es nicht im Rahmen der Möglichkeiten dieser Arbeit liegt, die beschriebenen Positionen auf ihre Richtigkeit zu überprüfen, so sei an dieser Stelle doch ein Verweis auf \textcite{bierlSchwundgeldFreiwirtschaftUnd2012} erlaubt. In seiner ausführlichen Auseinandersetzung mit Regionalgeldmodellen gelangt dieser unter anderem zu der Einschätzung, dass „Regionalgeld sei eine Spielerei für wohlhabende Leute aus dem grünen und esoterischen Segment der akademischen Mittelschicht“ \parencite[37]{bierlSchwundgeldFreiwirtschaftUnd2012}. Zur Kritik an der Zinskritik siehe \textcite{bergerKritikZinskritik2011}.} und der Wirtschaftsprofessor Franz Hörmann äußern sich derartig.

\subsubsection{Zwischenfazit}

Wie gerade gezeigt lassen sich die Ursachenbeschreibungen der Eurokrise in vier Kategorien einteilen, die auf jeweils unterschiedlichen Ebenen ansetzen – nationalstaatliche Politik, europäische Politik, Banken / Finanzmärkte, Wirtschaftssystem – und in unterschiedlichem Umfang in den Sendungen vorkommen.

Zwar kommt in keiner Folge bloß ein Topos vor, allerdings muss festgestellt werden, dass der erste hier aufgezeigte Topos, der in der Krise eine Staatsschuldenkrise verursacht durch die unvernünftige Ausgabepolitik einzelner Staaten, sieht, in den allermeisten Folgen deutlich häufiger vertreten wird. Grundlegende Überlegungen zu den wirtschaftlichen Verhältnissen als mögliche Ursache finden sich hingegen nur äußerst selten. Es kann also nicht von einer eindeutigen Hegemonie eines Erklärungsansatzes, aber doch zumindest von Ansätzen hierfür gesprochen werden.

Differenziert man nach einzelnen Sendungen, so stellt man fest, dass in den beiden untersuchten Talkfolgen von \textit{Beckmann} und \textit{Anne Will} alle bzw. fast alle Topoi vorkommen. Während in den Diskussionen bei \textit{Günther Jauch} und \textit{maybrit illner} nur jeweils zwei Erklärungsansätze ausgeführt werden. Dies bestätigt zum einen die, bereits im Hinblick auf die Moderation diagnostizierte, relativ hohe Informationsdichte bei Beckmann. Zum anderen zeigt es, dass aus der bloßen Überrepräsentation von Regierungspolitikern, wie bei Beckmann (vgl. Kapitel \vref{chap:parteizugehoerigkeit}) nicht per se negative Konsequenzen folgen müssen.

\subsection{Lösungen der Krise}

Die nächste zu klärende Frage ist, welche Lösungsvorschläge, der zweite zentrale Aspekt der medialen Krisenbearbeitung, in den Sendungen diskutiert wurden. Da es sich dabei um eine Vielzahl an Einzelmaßnahmen handelt, ist es auch hier notwendig eine Kategorisierung der Vorschläge vorzunehmen. Konkret lassen sich drei große Diskurstopoi ausmachen. Der erste Lösungsweg möchte die betroffenen Länder mittels neoliberaler Reformen wettbewerbsfähig machen, der zweite Weg setzt dagegen auf Investitionsprogramme, Umverteilung und die Regulierung der Finanzmärkte, während beim dritten Topos die Lösung in der Abschaffung bzw. drastischen Veränderung des Euros gesehen wird.

Wie zuvor gilt auch hier der Hinweis, dass die drei Topoi nicht komplett von einander getrennt werden können. Dennoch geben sie die großen inhaltlichen Tendenzen der Diskussionsrunden gut wieder.

\subsubsection{Neoliberale Reformen}

Betrachten wir das erste Topos des Diskurses innerhalb der Talkfolgen näher. Logisch knüpft dieser an den ersten Ursachentopos an. Vertreter dieser Argumentationslinie plädieren dafür die finanziellen Hilfen an die in Not geratenen Staaten fortzusetzen. Allerdings sollte diese an klare Bedingungen geknüpft sein.

Die Staaten müssten in der durch die Hilfen gewonnenen Zeit ihre Haushalte konsolidieren, eine Schuldenbremse nach deutschem Vorbild in der Verfassung verankern, ebenfalls sollten im Zuge der sogenannten „Agenda 2010“ eingeführte Maßnahmen auf die anderen Staaten übertragen werden. Insgesamt müsste die Wirtschaft liberalisiert und Staatsbesitz privatisiert werden. Damit könnten die Länder ihre Wett"-bewerbs"-fähig"-keit zurückerlangen und so dann ihre Schulden zurückbezahlen. Zusätzlich solle die EU die Möglichkeit bekommen die Staatshaushalte zu kontrollieren, um eine erneute Schuldenkrise in Zukunft zu verhindern. Das hierfür bemühte Schlagwort lautet „Stabilitätsunion“. Exemplarisch für einen Argumentationsgang innerhalb dieses Topos steht hier eine im nachstehenden Transkript wiedergegebene kurze Sequenz, in der Martin Schulz (SPD) die Übertragung der deutschen Schuldenbremse auf ganz Europa (Z. \lref{28:1}-\lref{28:2}) und die eingeleiteten Sparmaßnahmen lobt (Z. \lref{28:3}-\lref{28:4}). Den Protest der Bevölkerung sieht Schulz dabei nicht als Kritik an dieser Politik, sondern vielmehr als Bestätigung ihrer Wirksamkeit\footnote{Die Äußerungen zu den verschiedenen thematisierten Protestbewegungen wurden hier nicht weiter untersucht. Allerdings war bei Sichtung der Folgen auffällig, dass sie von keinem Talkgast rundheraus abgelehnt wurden. Stattdessen wurde meist Verständnis geäußert, auch von Regierungsvertretern und Bankenvertretern, allerdings ist für sie der eingeschlagene Weg leider alternativlos.} (Z. \lref{28:5}-\lref{28:6}).

\begin{description}
	\begin{linenumbers}[1]
		\item \#00:50:12-5\# \textbf{Illner}: Herr Schulz haben wir einen Club von Drogenabhängigen versammelt? \#00:50:15-2\#
		
		\item \#00:50:16-0\# \textbf{Schulz}: Also ($\ldots$) Ich bin auch gegen diese Junkiepolitik. Absolut. Nein ich hab Verständnis für ihre Sorgen. Aber wir müssen uns einfach anschauen, was wir in den letzten Jahren gemacht haben. (..) \llabel{28:1}Um diese Versuchung zu vermeiden sich jetzt billiges Geld zu besorgen, damit wieder in den Konsum zu gehen, Wahlversprechen zu finanzieren. Haben wir Schranken eingebaut. Eine dieser Schranken lautet Fiskalpakt. Das ist deutschen Schuldenbremse auf ganz Europa ausgedehnt! Da redet kein Mensch drüber. Ratifizieren gerade viele Länder, Frankreich zum Beispiel. Dort werden über diesen Fiskalpakt, den wir im Bundestag ja durchgesetzt haben, die Länder verpflichtet Schuldengrenzen einzuhalten und Staatsschulden abzubauen.\llabel{28:2} Zweitens. \llabel{28:3}Wir haben im vergangenen Jahr eine Richtlinie zur Bekämpfung der exzessiven Defizite verabschiedet, hab ich eben schon mal drüber geredet. Herr Rajoy, der kann erzählen was er will. Der muss die Vorgaben, die aus Brüssel kommen, zum Abbau seiner Defizite einhalten. Wir haben also die Schranken aufgebaut, die man müssen, muss um genau zu verhindern, dass das eintritt was befürchtet wird. Nämlich, die kriegen jetzt wieder billig Geld und jetzt gehen sie wieder in den Konsum. Genau das verhindern wir.\llabel{28:4} Ich sage im übrigen, der Abbau der Staatsverschuldung ist, wenn wir mal, Herr Genscher, an die Generationenfrage denken. Ihre Generation, meine Generation, die Generation unserer Kinder. Ganz unabhängig von allem was wir über die EZB oder über den Euro sagen. Der Abbau von Staatsschulden ist eine Frage der Generationengerechtigkeit. Ich bin das Kind einer Generation von Eltern die hatte ein Ziel: Meinen Kindern soll's besser gehen als mir und mir geht's besser als meinen Eltern je gegangen ist. Ich bin aber auch Angehöriger einer Generation die nicht weiß, ob es den Kindern noch so gut gehen wird wie uns heute und eine Voraussetzung dafür dass es den Kindern noch so gut geht, ist der Abbau von Staatsverschuldung. Deshalb, diese Weichenstellung ist in Europa vorgenommen worden. Ich sag das nochmal, was glauben sie denn, warum die Leute in Madrid und Rom rebellieren? Weil ihnen der Herr Monti in Rom, den sie grade so in Zweifel gezogen haben, Sparprogramme aufdrückt, wie es sie noch nie gegeben hat. Also. \llabel{28:5}Wir sind was den Abbau der Staatsverschuldung angeht auf einem guten Weg und deshalb muss man diesen (unv.) auch nehmen.\llabel{28:6} \#00:52:24-8\# 
	\end{linenumbers}
	\captionof{transkript}{maybrit illner (13.09.2012)}
\end{description}

Vertreter dieses Topos sind vor allem im Lager der Regierungspolitiker zu finden, was wenig überrascht beschreibt es doch die aktuelle deutsche Krisenpolitik. Aber auch Vertreter der SPD und von Banken (-verbänden) schließen sich der Linie an. Die Akteure stimmen also mit denen des ersten Ursachentopos überein. Insgesamt wird dieser Topos etwas häufiger und vehementer in den Talks artikuliert als der im nächsten Abschnitt beschriebene. Allerdings ist der Unterschied hier nicht so groß wie bei den Ursachenbeschreibungen.

\subsubsection{Staatliche Regulierung}

Im zweiten Topos dagegen besteht die Lösung nicht aus einer Steigerung der Wettbewerbsfähigkeit mittels neoliberaler Sparanstrengungen – die die Krise nur verschlimmerten –, sondern in einer Kombination aus Umverteilung, Investitionsprogrammen („Marshallplan“) und der Regulierung des Finanzsektors.

Die Schulden sollten mittels eines Schuldenschnitts annulliert werden, da darunter nur die Verursacher der Krise leiden würden, sei dies gerecht. Zudem soll die Ungleichverteilung des gesellschaftlichen Reichtums mittels höherer Steuern auf große Vermögen ausgeglichen werden. Damit könne man dann unter anderem Investitions- und Konjunkturprogramme finanzieren, um die Wirtschaft anzukurbeln. Um eine zukünftige Wiederholung zu verhindern, müssten die Finanzmärkte stärker reguliert und Spekulation mittels einer Finanztransaktionssteuer eingedämmt werden. Zugleich sollten die mächtigen amerikanischen Ratingagenturen durch europäische Agenturen ergänzt werden. Die Europäische Union müsse zudem ihr Demokratiedefizit aufbauen, um wieder Akzeptanz bei den Bürgern zu finden.

\begin{description}
	\begin{linenumbers}[1]
		\item \#00:46:08-1\# \textbf{Gysi}: Wissen sie, es gibt drei Wege, Frau Will. Wenn ich das ganz kurz sagen kann. ($\ldots$) Was jetzt die EZB macht und was ja inzwischen gut geheißen wird von der Bundesregierung. \#00:46:17-1\# 
		
		\item \#00:46:18-0\# \textbf{Will}: Naja! Sie sagen ja nur es verbleibt im Mandat. \#00:46:19-7\# 
		
		\item \#00:46:19-8\# \textbf{Gysi}: Jaja, gut. Es ist mir auch wurscht. Aber zumindest wogegen sie nichts unternehmen, das ist doch die Variante des Gelddruckens. Wenn man aber Geld druckt, dann entwertet man die Sparguthaben, die sie gerade noch retten wollten, Herr Bremmer. Dann entwertet man die Gehälter, die Renten, die Sozialleistungen. Die zweite Variante ist das was SPD und Grüne vorschlagen. Das ist die gemeinschaftliche Verschuldung. So wie sie das vorschlagen geht das auch nicht, weil die Bürger nicht den geringsten Einfluss auf das haben was da passiert, aber haften. Und die dritte Variante. Wobei ein bisschen Gelddrucken, ein bisschen gemeinsame Verschuldung geht alles, es kommt nur auf den Grad an. \llabel{29:1}Aber die dritte Variante ist die Frage der Umverteilung. Endlich mal von oben nach unten. Und das sind wirklich drei völlig verschiedene Wege um den Euro zu retten.\llabel{29:2} Sehen sie mal wenn wir/ \#00:47:02-4\# 
	\end{linenumbers}
	\captionof{transkript}{Anne Will (12.09.2012)}
\end{description}

Ähnlich wie im obigen Beispiel (insb. Z. \lref{29:1}-\lref{29:2}) werden derartige Argumentationen  von Oppositionspolitikern, vor allem jenen der Linkspartei, aber auch von einigen der eingeladenen Experten und Journalisten vertreten. Also jenen Personen, die auch bei den Finanzmärkten oder im Wirtschaftssystem die Krisenursache sehen.

Der geforderten Regulierung des Finanzsektors schließen sich aber beispielsweise auch Vertreter des ersten Lösungstopos an. Ihre Forderungen haben dabei allerdings einen geringeren Umfang und zudem verweisen sie oft darauf, dass entweder bereits viel in diese Richtung getan worden sei oder dass man so etwas nur gemeinsam mit allen anderen Ländern der Europäischen Union und den USA tun könne, deren Überzeugung aber leider viel Zeit brauche.

\subsubsection{Austritt aus dem Euro}

Der dritte und zugleich radikalste Topos ist die Forderung nach einem Austritt aus dem Euro. Die Pleite Griechenlands sei vernünftigerweise sowieso nicht mehr zu verhindern und auch die anderen Rettungsmaßnahmen würden nichts helfen, weil der Euro nicht abwertet werden könne oder weil die Rettungspakete den Druck der Märkte auf die Länder so stark verringerten, dass diese die notwendigen Sparmaßnahmen wegen politischem Opportunismus nicht umsetzten. Deshalb sollten entweder Griechenland (und andere schwächelnden Eurostaaten des Südens) oder Deutschland (und einige weitere nördliche Eurostaaten) die gemeinsame Währung verlassen bzw. der Euro gleich komplett aufgegeben werden. Im folgenden Beispiel führt Hans-Olaf Henkel seine Idee einer Teilung des Euro aus (Z. \lref{30:1}-\lref{30:2}):

\begin{description}
	\begin{linenumbers}
		\item \#00:35:07-1\# \textbf{Maischberger}: Sie sagen, was die Deutschen brauchen ist eine Teilung des Euro in einen Nordeuro, mit den starken Ländern, mit den äh ähm seriös (Henkel: Richtig!) ähm haushaltenden und den Olivenländern, wie sie sie nennen. Also einen Nordo und einen Südo. Ähm. Wer profitiert denn davon mehr? Die nördlichen oder die südlichen Länder? \#00:35:25-2\#
		
		\item \#00:35:25-6\# \textbf{Henkel}: Darum. Alle beide. Äh. Erstmal muss ich sagen, es geht mir darum, die Alternativen aufzuzeigen und wir haben ja über die Entschuldung von Griechenland gesprochen. Einige reden ja davon, dass man Griechenland ausschließen müsste. Das ist wenn sie so wollen der Plan B. Der dazu führt, und da hab ich wirklich größtes Verständnis für die zögernde Haltung der Bundesregierung, dass das nicht zu kontrollieren ist. Denn ein Bankensturm in Athen könnte am nächsten Tag ein Bankensturm in Lissabon folgen und dann einer in Madrid und so weiter. Niemand weiß wie das endet. Deshalb habe ich gesagt, wenn wir daraus irgendwie heraus wollen, aus dieser Einheitswährung die niemanden wirklich hilft, dann müssen wir einen anderen Weg finden. \llabel{30:1}Und mein Vorschlag ist, dass wir gehen. Zusammen mit drei anderen Ländern. Das wäre also äh Holland, Finnland und Österreich zusammen mit Deutschland.\llabel{30:2} Den Euro da lassen wo er ist. \#00:36:14-0\# 
		
		\item \#00:36:14-1\# \textbf{Maischberger}: Frankreich tun sie den südlichen Ländern zu? \#00:36:15-2\# 
		
		\item \#00:36:15-2\# \textbf{Henkel}: Aber selbstverständlich. Frankreich leidet. Ich hab dort elf Jahre gelebt, Frau Maischberger ((Maischberger lacht)). \#00:36:20-0\#
	\end{linenumbers}
	\captionof{transkript}{Menschen bei Maischberger (18.10.2011)}
\end{description}

Diese Argumentationslinie wird allerdings in den untersuchten Folgen nur recht selten vertreten und dann auch nur von einzelnen Wirtschaftsvertretern und Experten, nicht jedoch von deutschen Politikern. Dies zeigt, dass diese Position zumindest innerhalb der Führungsriegen der etablierten Parteien keine ist, die man öffentlich vertritt.

Am ausführlichsten dargestellt wird sie in den beiden \textit{Menschen bei Maischberger} Folgen, da dort mit Hans-Olaf Henkel und Arnulf Baring jeweils zwei ausgewiesene Eurogegner anwesend sind. Hinzu kommt, dass Henkel Unterstützung von Marie-Christine Ostermann erhält, die Vorsitzende des Verbands „Die Jungen Unternehmer“. Sie fordert zwar keine Aufteilung des Euro, aber aus einer marktradikalen Warte heraus die Einstellung aller Hilfszahlungen. Ostermann ist auch ein Beispiel dafür, wie sich einzelne Akteure nur Teile der in den Topoi zusammengefassten Argumente zu eigen machen können, wobei dies allerdings relativ selten passiert.

\subsubsection{Zwischenfazit}

Ähnlich wie bei den Ursachentopoi lassen sich auch die Vorschläge zur Lösung der Krise in drei Topoi zusammenfassen, die in den Sendungen in unterschiedlichem Maße vertreten werden. Die Forderung nach neoliberalen Reformen der Krisenstaaten ist der am häufigsten vorkommende, was insofern folgerichtig ist, als er logisch auf dem ersten dargestellten Ursachentopos basiert. Entsprechend sind jeweiligen Akteure auch fast deckungsgleich. Weniger häufig kommt bereits das Topos der staatlichen Regulierung vor, bei dem eine logisch und personelle Nähe zum zweiten Ursachentopos festzuhalten bleibt. Äußerst selten wird die Forderung nach einem Austritt aus dem Euro in die Gesprächsrunden eingebracht. Der Topos befindet sich bereits stark am Rande der im medialen Diskurs als legitim erachteten Positionen und erhält in den Sendungen entsprechend viel Widerspruch durch die anderen Gäste.

Wirklich alternative Wirtschaftsweisen hingegen werden so gut wie gar nicht diskutiert. Auch die Auftritte von Franz Hörmann und den beiden Regionalgeldverfechtern leisten dies nicht, da die Ausführungen Hörmanns vollkommen marginal bleiben und die beiden Letzteren in ihrer Rolle bloßer Stichwortgeber verbleiben (vgl. Kapitel \vref{chap:akteursgruppen}).

\section{Besetzung von Schlüsselbegriffen}\label{chap:schluesselbegriffe}

Bei der Analyse der Sendung anhand der beiden Leitfragen und der folgenden Kategorisierung in vier bzw. drei Topoi fällt auf, dass sich trotz aller Unterschiede immer wieder auf die gleichen Begriffe bezogen wird. Es findet hier ein Deutungskampf um als zentral erachtete Begriffe statt. Im Folgenden sollen beispielhaft einige dieser Schlüsselbegriffe vorgestellt werden.

Da wäre zum einen das allgegenwärtige Bekenntnis zu Europa, gewissermaßen als Eintrittskarte zur Diskursteilnahme. Selbst explizite Eurogegner, wie im nachstehenden Beispiel Wilhelm Hankel (Z. \lref{31:1}-\lref{31:2}, \lref{31:3}-\lref{31:4}), bekennen sich hierzu, wollen dieses Europa aber natürlich anders verstanden wissen, als die Eurobefürworter:

\begin{description}
	\begin{linenumbers}[1]
		\item \#00:52:47-3\# \textbf{Hankel}: Äh. \llabel{31:1}Wir brauchen ein anderes Europa. Wir brauchen Europa, weil wir Europäer sind und weil wir Nachbarn sind und weil wir Freunde bleiben und weil, wie Altmaier richtig sagt, ja auch keine Kriege führen.\llabel{31:2} Übrigens wir könnten sie gar nicht mehr führen. Wir könnten sie ja nicht mal mehr bezahlen. Aber wir brauchen nicht dieses Einheitseuropa, dass mit der Einheitswährung beginnt und mit der Wirtschaftsregierung, sprich der Einheits äh regierung, aufhört. \#00:53:10-8\#
		
		\item \#00:53:11-2\# \textbf{Wolf}: Aber was ist denn dann ein Europa für sie? Wenn das nicht ein gemeinsam // zusammenwachsendes Gebilde ist. // \#00:53:15-8\#
		
		\item \#00:53:14-0\# \textbf{Hankel}: // \llabel{31:3}Ein Völkerbund // unabhängiger Nationen. Eine Art Commonwealth, wie es die Engländer lange Zeit hatten, in dem jeder seine eigene Demokratie hat. Jeder. Aber in dem wir uns auf gewisse Weltaufgaben einigen, wie den Umweltschutz, wie auch die Verteidigung. Aber eines muss demokratisch bleiben, das ist die Innenpolitik. Die Innenpolitik lässt sich weder europäisieren noch globalisieren. Dafür sind unsere nationalen Parlamente zuständig.\llabel{31:4} \#00:53:41-6\# 
	\end{linenumbers}
	\captionof{transkript}{log in (26.10.2011)}
\end{description}

Ähnliches lässt sich feststellen bei den Begriffen „Soziale Marktwirtschaft“, die es zu erhalten gelte, dem „Mittelstand“, der gegen die Globalisierung verteidigt werden müsse, und dem kleinen Mann, dessen Sparguthaben zu schützen sei. Alle Seiten versuchen so ihre eigene Position zu legitimieren und begründen. Eng verwoben mit den bereits erwähnten Schlüsselbegriffen ist die Figur des nationalen Interesses. Dessen Verteidigung, so scheint es, liegt vor allem den Vertretern des ersten Lösungstopos, also der aktuellen Regierungspolitik, am Herzen. Benutzt wird das Motiv aber auch von Eurogegner wie Arnulf Baring oder Oppositionspolitikern, die sich gegen den Eindruck wehren, sie würden gegen „deutsche Interessen“ handeln, wie im folgenden kurzen Gespräch zwischen Barthle (CDU) und Bartsch (Linkspartei) (Z. \lref{32:1}-\lref{32:2}, 10f.):

\begin{description}
	\begin{linenumbers}[1]
		\item \#00:08:15-9\# \textbf{Barthle}: Herr Bartsch. Äh. Ich halte nichts davon hier ein Schreckgespenst was die Banken anbelangt an die Wand zu malen. \llabel{32:1}Jeder normale deutsche Bürger, der ein Sparkonto oder ein Girokonto hat, legt Wert darauf, dass seine Bank intakt bleibt und sein Geld nicht verloren geht. ((Bartsch: Sehr richtig.)) Also wenn wir dafür sorgen, dass unser Bankensystem nicht zusammenbricht, dann handeln wir nicht nur im Interesse der Banken. Dann handeln wir im Interesse aller Menschen dieses Landes, aber vor allem auch im Interesse der Wirtschaft.\llabel{32:2} Denn ohne eine funktionierendes Bankensystem kann keine Wirtschaft funktionieren. Das ist der Blutkreislauf der Wirtschaft und wenn wir den abdrehen, dann ist Ende, dann ist Schicht im Schacht. \#00:08:49-5\#
		
		\item \#00:08:49-5\# \textbf{Bartsch}: // Ja, wer will denn das! Ist doch völlig richtig. Das will doch kein Mensch. (.) Kein Mensch (.) Kein Mensch ($\ldots$) Kein // \#00:08:56-3\# 
		
		\item \#00:08:49-5\# \textbf{Barthle}: // Und deshalb macht es keinen Sinn hier Schreckgespenste an die Wand zu malen wir würden nur Banken retten. // \#00:08:55-0\# 
	\end{linenumbers}
	\captionof{transkript}{log in (12.09.2012)}
\end{description}

Es ist quasi die Fortsetzung des schon in den Eröffnungssequenzen durch die Moderation feststellbare „Nationalisierung“ des Diskurses (vgl. Kapitel \vref{chap:eroeffnung}). Der bloße Hinweis auf Solidarität als Begründung erscheint beispielsweise Hermann Grohe (CDU) nicht als ausreichend, das nationale Interesse muss schon hinzutreten:

\begin{description}
	\begin{linenumbers}
		\item \#00:16:07-2\# \textbf{Grohe}: Nein, es gibt eine Absicherung weiterer Staatsfinanzierung. Es gibt darüber hinaus massive Hilfen aus den Strukturmitteln der Europäischen Union. Da wird auch beraten, dass die in Zukunft sinnvoller eingesetzt werden können. Es gibt gar kein vertun das Griechenland in einer Fülle von Bereichen auch hiervon profitiert hat, aber eben nicht genug. Und deswegen muss es darum gehen, dass das was solidarisch zur Verfügung gestellt auch sinnvoll eingesetzt wird. Deutschland haftet mit 37 Milliarden heute schon mit das was an Hilfe für Griechenland geleistet wird, das ist ein gewaltiger Solidaritätsakt. Aber der geschieht auch im eigenen Interesse, weil wir in der Tat auch den Euroraum stabil halten müssen. ((Applaus)) \#00:16:44-2\#
	\end{linenumbers}
	\captionof{transkript}{hart aber fair (18.06.2012)}
\end{description}