\chapter{Fazit}

Zum Schluss geht es nun noch darum, die in der Analyse erhaltenen Puzzleteile zu einem kohärenten Bild zusammen zu setzen. Natürlich hat auch die vorliegende Arbeit Defizite und ist damit in ihrer Aussagekraft beschränkt. Insbesondere musste aus forschungsökonomischen Gründen auf die Einbeziehung einer Rezeptionsperspektive verzichtet werden und damit auch auf eine Beurteilung der Wirkung der Sendungen. Diese aus den vorkommenden Topoi und deren Ungleichverteilung abzuleiten wäre zu einfach, da Rezeption und Wirkung nicht unabhängig von einander betrachtet werden können \parencite{bucherGrundlagenInteraktionalenRezeptionstheorie2012}. Dennoch ist es sinnvoll diese Einseitigkeiten zu konstatieren und zu kritisieren, da nur das rezipiert werden kann was überhaupt in den Sendungen vorkommt. Auch beim Inhalt der Sendungen mussten Aspekte unberücksichtigt bleiben, dies gilt insbesondere für weite Teile der visuellen Ebene, beispielsweise der Bildregie, da dies eine nicht leistbare Vervielfachung der notwendigen Arbeit bedeutet hätte. Ebenso musste die Analyse an einigen Punkten an der Oberfläche des Materials bleiben, da dieses schlicht zu umfangreich für eine tiefer gehende und umfassende Bearbeitung war. Es sind also genügend Anknüpfungspunkte für eine zielgerichtete Vertiefung der dargestellten Ergebnisse vorhanden.

Ziel der Arbeit war es die Gäste- und Themenstruktur der deutschen, politischen Talklandschaft zu vermessen und den Umgang mit politisch und gesellschaftlich relevanten Themen in den Polittalks am Beispiel der Eurokrise zu analysieren. Dazu wurden sieben politische Talkshows – \textit{Anne Will}, \textit{Beckmann}, \textit{Günther Jauch}, \textit{hart aber fair}, \textit{maybrit illner}, \textit{Menschen bei Maischberger} und \textit{log in} – ausgewählt, diese repräsentieren sowohl die reichweitenstärksten Polittalks im deutschen Fernsehen als auch ein jüngeres, innovativeres Formate.

Auf der Makroebene konnte gezeigt werden, dass die untersuchten Polittalks sich durch ihre Gäste- und Themenwahl nur bedingt von einander absetzen können, vielmehr lassen sich drei Gruppen bilden. Da wären zum einen die beiden klassischen Polittalks \textit{Günther Jauch} und \textit{maybrit illner}, die sich am stärksten am Vorbild \textit{Sabine Christiansen} orientieren. Beide Sendungen zeichnen sich durch einen hohen Anteil an Themen aus den Bereichen Politik und Wirtschaft und einer entsprechenden Gästestruktur aus, bei der Politiker und Personen aus dem Wirtschaftsleben die Mehrheit bilden.

Die zweite Kategorie umfasst \textit{Anne Will} und \textit{log in}. Ihre Themenstruktur ähnelt zwar der ersten Gruppe, hat aber eine Gästestruktur die nicht so stark auf den politisch-administrativen und wirtschaftlichen Bereich ausgerichtet ist und im Falle von \textit{log in} den höchsten Anteil zivilgesellschaftlicher Akteure unter den Gästen aufweist.

In die letzte Gruppe fallen \textit{Beckmann}, \textit{Menschen bei Maischberger} und \textit{hart aber fair}. Diese drei Sendungen behandeln mehr „bunte“ Themen und erreichen so einen vielfäl"-tiger"-en Themenmix, entfernen sich aber zugleich vom reinen Politiktalk und nähern sich dem an was hier als Personality-Show bezeichnet wird (vgl. Kapitel \vref{chap:typologien}). Die Gästestruktur stützt diesen Eindruck, haben die drei Sendungen doch den niedrigsten Anteil an Politikern unter den Gästen und gleichzeitig den höchsten Anteil an Gästen aus den Bereichen Kultur und Sport. Diese Entwicklung ist zumindest bei \textit{hart aber fair} bemerkenswert, gehörte die „harte Politik“ doch jahrelang zum Image der Sendung. Laut ARD-Programmbeirat hat sich Plasberg allerdings selbst „zu dem Profilwechsel entschlossen“ \parencite{ard-programmbeiratTalkformateImErsten2012}.

Allen Sendungen gemein ist, dass nur ein relativ kleines Spektrum politischer Themenfelder behandelt wird (vgl. Kapitel \vref{chap:themenbereiche}). Innenpolitische Themen dominieren die Diskussionssendungen, daneben kommen bloß noch sozialpolitische Themen bei \textit{Anne Will}, \textit{hart aber fair} und \textit{Menschen bei Maischberger} in nennenswertem Umfang vor. Alle anderen Themenbereiche, wie beispielsweise die Außenpolitik, werden dagegen stark vernachlässigt. Damit einhergehen die bei allen untersuchten Sendungen zu findenden Themenwiederholungen (vgl. Kapitel \vref{chap:themencluster}). Diese sind sowohl innerhalb einzelner Sendungen zu finden – etwa in den zahlreichen Ausgaben von \textit{maybrit illner} zur Eurokrise – als auch zwischen den Sendungen – beispielsweise wenn sich innerhalb von sechs Tagen gleich vier Talks mit der Wahl des Bundespräsidenten beschäftigen. Negativ kommt weiterhin hinzu, dass die Titel der einzelnen Folgen oftmals die behandelten Themen skandalisieren oder in Alarmismus abgleiten (vgl. Kapitel \vref{chap:titelgebung}).

Weitere Verstöße gegen das Gebot der Pluralität und Ausgewogenheit (vgl. Kapitel \vref{chap:anforderungen}) finden sich auch hinsichtlich der Gästestruktur. Zum einen wäre da die Bildung von Talkeliten innerhalb einzelner gesellschaftlicher Gruppen, so trat im Schnitt jeder Politiker zweimal in den Talks auf und speziell beim Thema Eurokrise lässt sich eine Elitenbildung bei Politikern, Experten und Wirtschaftsvertretern feststellen (vgl. Kapitel \vref{chap:personenspektrum}), die zudem noch in diesen Talks zu den vier größten Gästegruppen zählen (vgl. Kapitel \vref{chap:gesellschaftlicherollen}).

Innerhalb der gesellschaftlichen Gruppen setzen sich derartige Ungleichheiten fort (vgl. Kapitel \vref{chap:gesellschaftlicherollen}), so lässt sich ein starkes Übergewicht bundespolitischer Akteure feststellen. Sendungsspezifisch ist dagegen die unterschiedliche Repräsentation der politischen Parteien. Während die Werte insgesamt den Ergebnissen der letzten Bundestagswahl ähneln, konnte bei den Talkfolgen zur Eurokrise eine Überrepräsentation der Regierungsparteien festgestellt werden. Unter den Wirtschaftsvertretern gibt es in allen Sendungen eine Dominanz der arbeitgebernahen Akteure, die sich beim Thema Eurokrise zur Hegemonie auswächst und dadurch verstärkt wird, dass kaum zivilgesellschaftliche Akteure eingeladen werden.

Die Altersstruktur der Sendungen entspricht nicht der gesellschaftlichen Realität, stattdessen ist eine Überalterung in allen Sendungen festzustellen und \textit{Menschen bei Maischberger} mit dem höchsten Altersschnitt, so zu einer Art „Geronto-Talk“ wird. Dies lässt sich nur bedingt auf die Themenstruktur des Talks zurückführen, da das ähnlich ausgerichtete \textit{Beckmann} auf einen sieben Jahre jüngeren Schnitt kommt. Anders ist es beim Geschlechterverhältnis, dort lässt sich bei den beiden Sendungen mit der größten Nähe zum Personality-Talk, \textit{Beckmann} und \textit{Menschen bei Maischberger}, noch das ausgewogenste Verhältnis von männlichen zu weiblichen Gästen feststellen, wobei trotz allem über 60 \% der Gäste Männer sind (vgl. Kapitel \vref{chap:demografie}).

\textit{log in} kann sich von den anderen Sendungen nur in einzelnen Aspekten absetzten – in dem relativ hohen Anteil an zivilgesellschaftlichen und arbeitnehmernahen Gästen, einem im Vergleich niedrigen Altersschnitt und der überproportional häufigen Einladung von Politiker der Piratenpartei. Insgesamt jedoch unterscheidet sich die Sendung nicht allzu stark von den anderen hier untersuchten Polittalks.

Hoffnungen, dass in den untersuchten politischen Talkshows Akteure zu Wort kommen, die sonst in den Medien unterrepräsentiert sind \parencite[386]{richardsonSpecificDebateFormats2008}, müssen aufgrund der vorliegenden Ergebnisse jedenfalls abschlägig beschieden werden. Die Zusammensetzung der Gäste reproduziert und verstärkt stattdessen eher noch bereits bestehende gesellschaftliche Machtgefälle. Die im Laufe der Arbeit angeführten früheren Forschungsbefunde zu Themen- und Gästestruktur anderen politischer Talksendungen gelten im Großen und Ganzen auch für die aktuellen Sendungen, wenn auch die Vergleichbarkeit aufgrund unterschiedlicher Untersuchungsdesigns eingeschränkt ist.

Die Befunde der Strukturanalyse setzen sich auf der Mikroebene quasi fort. Bereits in der Gästeauswahl konnte eine Bevorzugung von arbeitgebernahen Akteuren und Politikern des Regierungslagers festgestellt werden und diese spiegeln sich auch in der Häufigkeit der in den Sendungen vorkommenden Topoi wieder. Sowohl hinsichtlich der Ursachenerklärung als auch der Lösungsvorschläge kommen jeweils die Topoi am häufigsten vor, die vornehmlich von Regierungspolitikern und wirtschaftsnahen Akteuren vertreten werden. Allerdings ist der Unterschied zu den anderen Topoi nicht so groß, wie die quantitativen Ungleichgewichte in der Gästestruktur es vermuten ließen, was darauf hindeutet, dass die gleiche gesellschaftliche Rolle nicht unbedingt die gleiche Interessenlage nach sich ziehen muss.

Die Untersuchung des Inhalts der Diskussionen mittels zweier Leitfragen – Ursache und Lösung der Krise – ließ aber auch erkennen, dass sich die vorkommenden Argumentationen zu vier bzw. drei Topoi zusammenfassen lassen, die sich in unterschiedlichem Ausmaß durch alle Sendungen hinweg wiederholen (vgl. Kapitel \vref{chap:diskurstopoi}).

Das erste Ursachentopos behandelte die Krise als Folge unverantwortlichen Handelns einzelner Nationalstaaten, die so über ihre Verhältnisse gelebt und zu hohe Staatssschulden aufgehäuft hätten. Wenig überraschend korrespondiert hiermit der erste Lösungstopos, in dem gefordert wird, dass sich die Staaten durch Sparmaßnahmen und neoliberale Reformen nach dem Vorbild der deutschen „Agenda-Politik“ wieder wettbewerbsfähig mache sollten. Diese beiden Topoi sind diejenigen, die vornehmlich von Regierungspolitikern und wirtschaftsnahen Akteuren vertreten werden.

Im zweiten Ursachentopos wird die Ursache an der Eurokrise dagegen den Banken und Finanzmärkten zugeschrieben, die sich verspekuliert und in der Finanzkrise durch die Staaten hätten gerettet werden müssen, sodass sich diese wiederum zu stark verschuldet hätten. Aus dieser Diagnose folgt wiederum das zweite Lösungstopos – der Ruf nach Regulierung und staatlichem Eingreifen. Der Finanzsektor müsste durch Verbote in seine Schranken verwiesen werden und durch höhere Steuern auf große Vermögen, sollte eine gesellschaftliche Umverteilung initiiert und Investitionsprogramme finanziert werden. Beide Topoi sind die am zweithäufigsten artikulierten und werden primär durch die eingeladenen Oppositionspolitiker sowie einige Journalisten und Experten vertreten, aber auch einzelne Vertreter des ersten Lösungstopos können sich mit einzelnen Argumenten des Topos anfreunden.

Als dritten Ursachentopos lässt sich die Schuldzuweisung an das gemeinsame Währungs"-system an sich ausmachen. Diese Argumentationslinie sucht und findet den Grund der Krise im Euro, der auf die eine oder andere Weise eine Fehlkonstruktion sei. Die Gruppe die diesen Topos vertritt ist relativ heterogen und reicht vom konservativen Euroskeptiker bis zum linken Globalisierungskritiker. Entsprechend wird der zugeordnete Lösungstopos – „Raus aus dem Euro“ – nicht von allen Mitgliedern dieser Gruppen geteilt, dennoch findet er sich in immerhin sechs der vierzehn untersuchten Folgen.

Das letzte Ursachentopos, bei dem die Krisenursache im herrschenden Wirtschaftssystem, dem Kapitalismus, ausgemacht wird, ist zugleich dasjenige mit der marginalsten Rolle in den Sendungen. Vertreten nur durch wenige Einzelpersonen, die zudem meist nicht die ganze Sendung über anwesend sind, kommt es in den Diskussionen so gut wie gar nicht vor. Es kann folglich auch nicht verwundern, dass sich kein zugehöriges Lösungstopos findet – die Forderung nach der Überwindung des Kapitalismus ist dann wohl doch zu anrüchig, als dass sie sich für die politischen Talkshows des öffentlich-rechtlichen Fernsehens qualifizieren würde.

Bei der Behandlung der Eurokrise in den Talkshows zeigt sich nicht nur eine Reduktion der Argumentationslinien auf bloß drei, vier wiederkehrende Topoi, sondern auch auf bestimmte Schlüsselbegriffe die unterschiedlichste Akteure aufgreifen und versuchen in ihrem Sinne zu besetzen (vgl. Kapitel \vref{chap:schluesselbegriffe}). Besonders zentral sind dabei die beiden Begriffe „Europa“ und „Nationales Interesse“. Das Bekenntnis zu beiden ist quasi die Eintrittskarte um seine eigene Forderungen als legitime Position im Diskurs vertreten zu können.

Die aufgezeigten Topoi bilden dabei „den Raum des legitimen Diskurses“ \parencite[142]{doernerPolitainmentPolitikMedialen2001} der auch außerhalb der Talkshows existiert, aber doch gleichzeitig auch durch diese geformt wird. Damit sagen die vorkommenden – und nicht vorkommenden – Topoi und deren Ungleichverteilung auch etwas über die ungleichen Machtverhältnisse der hinter ihnen stehenden Interessen in der Gesellschaft aus.

Abseits der Topoi wurde zudem ein Blick auf einzelne, zentrale Sendungselemente der Talks geworfen (vgl. Kapitel \vref{chap:sendungselemente}). Die Eröffnung der Gesprächsrunde durch die Moderatoren unterscheidet sich wenig von einander. In der Mehrzahl versuchen die Moderatoren die Position des Anwalts der Zuschauer zu übernehmen und sorgen damit unwillkürlich für ein nationales Framing der behandelten Probleme. Zusammen mit dem häufigen Bezug der Gäste auf das „nationale Interesse“ ist dies zudem ein Hinweis auf die von Carsten Brosda festgestellte „Re-Nationalisierung des Blickes auf Europa auch in der seriösen Berichterstattung der Qualitätsmedien“ \parencite{siniawskiSchuldenbergeHilfspaketeUnd2012}. Die Schlussrunden hingegen dienen fast nie dem Ziehen eines Fazits oder der Suche nach einem Kompromiss, was durchaus positiv bewertet werden kann. Stattdessen enden die Runden entweder unvermittelt, indem einfach jeder Gast nochmal etwas sagen darf oder durch einen kurzen Ausflug auf die Ebene des Persönlichen.

Bei der Gesprächsorganisation und dem Umgang mit verschiedenen Akteursgruppen geben sich fünf von sieben Moderatoren keine grobe Blöße. Sandra Maischberger fällt dagegen durch eine mangelnde Durchsetzungsfähigkeit auf, die zu einer thematischen Inkohärenz führt. Frank Plasberg hat zwar seine Gäste im Griff, bevorzugt dabei aber sowohl durch Moderation als auch durch Einspieler einseitig bestimmte Positionen und verstärkt damit die bereits erwähnte Dominanz bestimmter Diskurstopoi. Fairerweise muss man allerdings hinzufügen, dass sich fragwürdige Einspieler auch bei anderen Sendungen finden und es keinem Moderator gelingt strukturelle Defizite bei der Zusammensetzung der Gäste durch seine Gesprächsführung auszugleichen.

Ebenfalls keine gute Wahl scheint die Einladung von Prominenten zu sein, wie sich ebenfalls bei \textit{hart aber fair} zeigen ließ, trug beispielsweise die Einladung Costa Cordalis mehr zur Belustigung des Publikums bei, denn zur Klärung politischer Fragen. Positiv an \textit{hart aber fair} ist hingegen die Integration der „Betroffenen“ in die Runde, auf diese Weise konnte sie gleichberechtigt an der Diskussion teilnehmen. Gleiches wäre auch für die zivilgesellschaftlichen Akteure in den anderen Sendungen wünschenswert gewesen, die leider in fast allen untersuchten Folgen bloße Stichwortgeber blieben.

\textit{log in} kann sich auch auf dieser Ebene nur bedingt von den etablierten Talks absetzen. Zwar führt die Konzentration auf zwei gegensätzliche Standpunkte zu einer klareren Konturierung der Debatte und die starke Einbindung der Zuschauer macht die anwaltschaftliche Rolle am deutlichsten. Inhaltlich aber kamen bei log in nicht mehr oder weniger Positionen zu Wort als in den anderen Sendungen. Die Sendung hat also zwar auf der gestalterischen und dramaturgischen Ebene eine Sonderstellung inne, eifert aber den etablierten Polittalks inhaltlich (noch) zu sehr nach.

Insgesamt betrachtet muss also festgestellt werden, dass es bei den  untersuchten politischen Talkshows tatsächlich zu einer Themen- als auch Gästeinflation, zumindest bei bestimmten Gästegruppen, kommt. Dies wirkt sich logischerweise auch auf die Relevanz der einzelnen Sendungen aus, denn wenn alle ein Thema gleichzeitig mit den gleichen Gästen diskutieren, verliert die einzelne Folge an Relevanz, wurden doch alles wahrscheinlich bereits in einer anderen Sendung gesagt. Gleichzeitig verstoßen Ungleichheiten in der Gästezusammensetzung und -demografie gegen die eigentlich gebotene Pluralität und Ausgewogenheit. Die Bearbeitung der Eurokrise ist dabei ähnlich ernüchternd, statt neue Standpunkte in den Diskurs einzubringen, kommt es auch hier primär zu einer Wiederholung bereits bekannter Argumente und der Reproduktion und Schaffung von Ungleichgewichten zugunsten bestimmter Positionen. In diesem Sinne erfüllen die Talks zwar ihre Aufgabe, verschiedene politische Konfliktlinien sichtbar zu machen, sind dabei aber dennoch zu selektiv.

Aufgrund der umfangreichen Materialbasis dürften die Ergebnisse in gewissem Maße auch auf die anderen im deutschen Fernsehen ausgestrahlten Polittalks übertragbar sein.

Zum Schluss sei noch eine Antwort auf die Fragen zu geben, ob die ARD-Talkschiene nicht zu viel des Guten ist: Ja ist sie. Wie gezeigt werden konnte wiederholen sich sowohl Gäste als auch Themen in den sieben untersuchten Talksendungen. Hinzu kommt, dass auch innerhalb der Sendungen, wie am Beispiel der Eurokrise illustriert wurde, bloß einige wenige Topoi wiederholt werden. Dieses Urteil wird verschärft durch den Umstand, dass sich sowohl auf der Makroebene der Gästezusammensetzung als auch in der Mikrostruktur einzelner Folgen Ungleichgewichte zugunsten einzelner Akteurs- und Interessengruppen zeigen lassen. Im konkreten Fall der Eurokrise schlägt sich dies nieder in einer Bevorzugung der von der Regierung vertretenen Ursachenzuschreibungen und Lösungsvorschläge. Allerdings lassen sich die aufgeführten Verstöße gegen Pluralismus, Repräsentativität und journalistische Professionalität nicht nur bei den ARD-Talks sondern auch bei den beiden ZDF-Formaten finden.

Diese Defizite sollten dennoch nicht voreilig zu der Annahme verleiten, dass Talkformate vollkommen ungeeignete für die Behandlung politischer Fragestellungen sind. Es sollte aber sowohl die Gäste und Themenzusammensetzung, welche leicht durch die Einladung von Landes- oder Kommunalpolitiker erweitert werden könnte, als auch der Umgang der Moderatoren mit ihren Gästen überdacht werden. Hierbei kann es helfen die Zahl der politischen Talkshows (deutlich) zu reduzieren. Anbieten würde sich die Einstellung von \textit{Menschen bei Maischberger}, deren beiden untersuchten Folgen zur Eurokrise eine Art worst case darstellen. Der positiven ARD-internen Beurteilung ist hier unbedingt zu widersprechen \parencite[7]{ard-programmbeiratTalkformateImErsten2012}. Aber auch das oft hochgelobte \textit{hart aber fair} \parencites[7]{ard-programmbeiratTalkformateImErsten2012}[102ff.; vgl. Kapitel \vref{chap:hartaberfair}]{eisentrautPolitTalkAlsForm2007}, scheint seinen Zenit inzwischen überschritten zu haben. Im Versuch sich von seinen Anstaltskollegen abzusetzen – ein Problem vor dem auch die anderen ARD-Talks stehen –, entwickelt sich das Gäste- und Themenprofil immer mehr hin zur Personality-Show und auch der Umgang des Moderators mit Positionen, die ihm nicht liegen, erscheint fragwürdig. Für eine Reduzierung spricht zudem, dass keine der vierzehn untersuchten Folgen einen exklusiven Einblick in den Politikbetrieb und geplante Vorhaben erhaschen konnte.

Skeptisch könnte einen ob solcher Reform Versuche allerdings machen, dass sie bereits regelmäßig angemahnt wurden, ohne das sich viel geändert hätte\footnote{Vergleiche für die USA bspw. \textcite{nixMeetPressGame1974} oder für Deutschland \textcites[17f.]{muellerSchaubuehneFuerEinflussreichen2006}[117]{gaeblerUndUnserenTaeglichen2011}.}. Vielleicht ist es auch an der Zeit sich auf alte Tugenden zu besinnen. Bereits 1975 hieß es über III nach 9: „Dem Ideal der Rede-Freiheit scheint man mit einem gewissen Maß an 'Unordnung' bereits näherkommen zu können.“ Einem Maß an Unordnung, das man heute nur noch bei Talkshows wie \textit{Roche \& Böhmermann} beim Nischenkanal \textit{zdf.kultur} findet und die zumindest für Spiegel Online „eine der innovativsten Sendungen des Jahres“ \parencite{gitschierStudentenAlsProduzenten2012} ist. Vielleicht muss sich aber auch erst die Gesellschaft ändern, in der die Sendungen entstehen.