\chapter{Methodik und Vorgehen}

Die zweiteilige Untersuchung macht die Triangulation mehrerer Methoden notwendig, sodass die quantitative Inhaltsanalyse von Strukturmerkmalen der untersuchten Sendungen mit einer qualitativen Analyse einzelner Ausgaben der Sendungen gekoppelt wird. Ziel ist es die Makroebene der Sendungen, ihrer Themen- und Gästestruktur, mit der Mikroebene, dem Verlauf einzelner Ausgaben der Sendungen, zusammenzubringen.

\section{Quantitative Inhaltsanalyse}

Im ersten Teil werden folglich mittels einer quantitative Inhaltsanalyse die Gäste- und Themenstruktur der Sendungen genauer analysiert. Auf diese Weise können zum einen Gemeinsamkeiten und Unterschiede zwischen den sieben Sendungen herauspräpariert werden, zum anderen können so auch Einseitigkeiten und Doppelungen innerhalb der deutschen Talklandschaft erkannt werden.

Der codierte Korpus umfasst alle Ausgaben der fünf politischen Talkshows \textit{Anne Will} (ARD), \textit{Beckmann} (ARD),\textit{ Günther Jauch} (ARD), \textit{hart aber fair} (ARD), \textit{Menschen bei Maischberger}\footnote{Im Folgenden wird statt des vollen Titels in Tabellen aus Platzgründen teilweise die Bezeichnung \textit{Maischberger} benutzt.} (ARD), \textit{log in} (ZDF) und \textit{maybrit illner} (ZDF) Untersuchungszeitraum vom Ende der Sommerpause Anfang September 2011 bis Anfang September 2012\footnote{Eine ausführlichere Vorstellung der Sendungen findet sich in Kapitel \vref{chap:polittalks}.}. \textit{log in} bildet dabei mit der Sendung vom 24. August 2011 den Auftakt, den Abschluss bildet \textit{maybrit illner} vom 13. September 2012. Insgesamt wurden in diesem Zeitraum 284 Folgen ausgestrahlt. Da es sich bei einer Folge um eine „Best of“ Ausgabe handelte, wurden letztlich 283 codiert. Wie Tabelle \vref{tab:quandaten} zu entnehmen ist, sind die Sendungen zu ungefähr gleichen Teilen im Korpus vertreten.

\begin{table}[htb]
\caption{Korpus im quantitativen Teil}
\centering
\resizebox{\textwidth}{!}{%
	\begin{tabular}{lcc}
		\toprule
		\textbf{Sendung} & \textbf{Anzahl} & \textbf{Prozentualer Anteil }\\
		\bottomrule
		Anne Will & 37 & 13,1 \% \\
		\hline
		Beckmann & 43 & 15,2 \% \\
		\hline
		Günther Jauch & 40 & 14,1 \% \\
		\hline
		hart aber fair & 36 & 12,7 \% \\
		\hline
		log in & 45 & 15,9 \% \\
		\hline
		maybrit illner & 41 & 14,5 \% \\
		\hline
		Menschen bei Maischberger & 41 & 14,5 \% \\
		\toprule
		Alle Sendungen & 283 & 100 \% \\
		\bottomrule
	\end{tabular}
}
\label{tab:quandaten}
\end{table}

Die Themenstruktur wurde mittels zweier Variablen erfasst. Zuerst wurde der grobe Themenbereich mittels eines festen Kategoriensystems codiert1. Ergänzend zu diesem relativ groben Schema wurde das übergeordnete Thema durch ein Schlagwort genauer charakterisiert, um auf diese Weise Themencluster – bspw. Eurokrise oder „Wulff-Affäre“ – herauspräparieren zu können.

Pro Gast wurden fünf Variablen vercodet – Name, die Rolle, in der dieser in die Sendung eingeladen war, die Parteizugehörigkeit, das Alter und das Geschlecht. Die Daten wurden den jeweiligen Angaben auf den Sendungswebseiten oder, falls diese nicht vorlagen, den Folgen selbst entnommen. Das verwendete Codebuch ist im Anhang dokumentiert (vgl. S. \pageref{anhang:codebuch}).

Die Codierungen wurden vom Autor selbst vorgenommen, weswegen sich eine Überprüf"-ung der Intercoder-Reliabilität erübrigt. Zur Überprüfung der Intracoder-Reliabilität \parencite[183-193]{roesslerInhaltsanalyse2005} wurden zehn Prozent der Fälle pro Sendung ungefähr vier Wochen nach der ersten Codierung erneut codiert. Die Werte des Reliabilitätskoeffizienten lagen zwischen .88 und .75.

\section{Qualitative Analyse}\label{chap:qualitativeanalyse}

Ausgehend zum einen davon, dass die „Dichotomie von qualitativ versus quantitativ [\ldots] nicht nur forschungslogisch problematisch“, sondern auch „forschungspraktisch überholt“ \parencite[14]{bucherGrundlagenInteraktionalenRezeptionstheorie2012} ist und zum anderen, dass im Sinne der grounded theory die Analyse quantitativer Daten die Formulierung von Hypothesen erlaubt, die dann mittels einer qualitativen Analyse am Material genauer untersucht werden können, wurden im zweiten Teil einige Ausgaben der Polittalks einer qualitativen Einzelanalyse unterzogen.

Dazu wurden zunächst alle Sendungen ausgewählt, die sich thematisch mit der Eurokrise beschäftigen. Von diesen immerhin 47 Folgen\footnote{Eine Übersicht über diese Folgen findet sich im Anhang in Tabelle \vref{tab:anhang_folgen_eurokrise}.} – 23 davon im Jahr 2011 und 24 im Jahr 2012 – wurden dann pro Sendung jeweils zwei ausgewählt. Dabei wurde zum einen darauf geachtet, dass innerhalb der beiden Samples jeweils alle sieben Sendungen zeitlich möglichst nahe beieinander lagen und zum anderen je eine Stichprobe im ersten Halbjahr, also Ende 2011, und im zweiten Halbjahr des Erfassungszeitraums, also Mitte 2012, lagen.

\begin{table}[htb]\label{tab:qualdaten1}
	\caption{Sample 1 für die Einzelanalyse}
	\centering
	\resizebox{\textwidth}{!}{%
		\begin{tabular}{cll}
			\toprule
			\textbf{Datum} & \textbf{Sendung} & \textbf{Titel}\\
			\bottomrule
			28.09.11 & Anne Will & Die Euro-Abstimmung – Riskieren wir morgen alles? \\
			\hline
			13.10.11 & maybrit illner & Griechen pleite, Banken in Not – wer rettet den Steuerzahler? \\
			\hline
			18.10.11 & Maischberger & Die Angst wächst – Eurokalypse now? \\
			\hline
			23.10.11 & Günther Jauch & Klartext in der Krise – Helmut Schmidt und Peer Steinbrück zu Gast bei Günther Jauch \\
			\hline
			24.10.11 & hart aber fair & Bürger gegen Banken – Wut und Angst im Euroland \\
			\hline
			26.10.11 & log in & Der Euro-Poker – Retten wir die Richtigen? \\
			\hline
			27.10.11 & Beckmann & Europa vor dem Abgrund – wie sicher ist unser Geld? \\
			\bottomrule
		\end{tabular}
	}
\end{table}

\begin{table}[htb]\label{tab:qualdaten2}
	\caption{Sample 2 für die Einzelanalyse}
	\centering
	\resizebox{\textwidth}{!}{%
		\begin{tabular}{cll}
			\toprule
			\textbf{Datum} & \textbf{Sendung} & \textbf{Titel}\\
			\bottomrule
			08.05.12 & Maischberger & Alle gegen Merkel – Europa in Gefahr? \\
			\hline
			18.06.12 & maybrit illner & Die Griechenwahl – statt Ende mit Schrecken jetzt Schecks ohne Ende? \\
			\hline
			06.09.12 & Beckmann & Regieren Banken die Politik? \\
			\hline
			09.09.12 & Günther Jauch & Im Namen des Volkes! Müssen wir die Euro-Rettung stoppen? \\
			\hline
			12.09.12 & Anne Will & Das Euro-Urteil – ein guter Tag für Deutschland? \\
			\hline
			12.09.12 & log in & Zahlen, bis es kracht – Zum Euro verurteilt? \\
			\hline
			13.09.12 & maybrit illner & Zur Rettung verurteilt - Was ist uns Europa wert? \\
			\bottomrule
		\end{tabular}
	}
\end{table}

Methodisch wurde sich an der Dialog- und Diskursanalyse orientiert. Ein Feld das sich  durch zahlreiche verschiedene und teils miteinander konkurrierende Ansätze, Methoden und Forschungsziele auszeichnet \parencites[für einen Überblick siehe][]{geeRoutledgeHandbookDiscourse2012}[bzw. für die Dialoganalyse:][]{biereVerstehenUndBeschreiben1994}{hundsnurscherDialoganalyseReferateArbeitstagung1986}{hundsnurscherDialoganalyseIIReferate1989}{statiDialoganalyseIIIReferate1991} und deshalb hier primär als eine Art Werkzeugkasten genutzt wurde.

Zur Erschließung der vierzehn Talkfolgen wurde auf eine zweigleisige Strategie zurück"-ge"-griffen. Zuerst wurden bei jeder Folge die Eröffnungssequenzen transkribiert, dies umfasst die Anmoderation, die Vorstellung der Talkgäste sowie den weiteren Verlauf bis circa zur fünften Minute. Ebenfalls wurde bei allen Sendungen die Abmoderation oder, falls vorhanden, die Schlussrunde transkribiert. Am Anfang der Talks wird der thematische Rahmen vorgeben und in die jeweilige Thematik eingeführt. Das hier stattfindende Framing steckt zu einem bestimmten Grad von vornherein die folgenden Diskussionen ab und dürfte auch deren Interpretation durch die Zuschauer beeinflussen. Der Schlussabschnitt der Talks hingegen bietet den Moderatoren die Möglichkeit die zurückliegende Diskussion zusammenzufassen, ein Fazit zu ziehen und dadurch den Eindruck, den der Zuschauer hiervon hat zu prägen. Somit handelt es sich bei Eröffnung und Schluss um zwei zentrale Teile der Sendungen.

Zweitens dienten zur Erschließung der in den untersuchten Sendungen vorkommenden zentralen Diskurstopoi zwei inhaltliche Leitfragen\footnote{Ähnlich geht \textcite{wengelerNochNieZuvor2010} in seiner diachronen Analyse von Krisendarstellungen im Spiegel vor. Entsprechend wurde sich hier an den dort verwendeten fünf Leitfragen und der Untersuchungsmethode orientiert \parencite[143]{wengelerNochNieZuvor2010}.}. Diese versuchen zwei zentrale und zusammenhängende Aspekte der Diskussionen über die Eurokrise zu erfassen, nämlich die Frage nach ihren Ursachen und nach dem Weg zu ihrer Lösung:

\begin{enumerate}
	\item Was hat zur Eurokrise geführt?
	\item Welche Lösungen für die Krise gibt es?
\end{enumerate}

Prinzipiell wären zusätzliche Fragen zur Abdeckung weiterer, weniger zentraler Aspekte -- etwa der geschichtlichen Einordnung -- denkbar gewesen, aus forschungsökonomischen Gründen musste hierauf aber verzichtet werden, zumal die beiden Leitfragen ausreichend sind, um wiederkehrende Topoi zu identifizieren.

Aus Zeitgründen war in diesem Teil keine vollständige Transkription aller vierzehn Sendungen möglich, sodass bloß ein Rohtranskript\footnote{Hier wird bewusst nicht das Wort „Grobtrankript“ verwendet, da es nicht dessen Anforderungen erfüllt \parencite[83ff.]{dittmarTranskriptionLeitfadenMit2004}.} angefertigt wurde, welches die zentralen Aussagen und Vorgänge festhält. An dieses Rohtranskript wurden dann die beiden vorgestellten Leitfragen herangetragen. Zudem konnten mit seiner Hilfe Auffälligkeiten und Besonderheiten einzelner Sendungen festgehalten werden. Vollständig transkribiert wurden dann nur noch die jeweils für eine bestimmte Sache exemplarischen Ausschnitte\footnote{Da selbst diese Transkripte zusammen genommen einen Umfang von mehreren dutzend Seiten haben, wurde auf einen Abdruck im Anhang verzichtet. Dort finden sich nur diejenigen Passagen auf die in der Arbeit verwiesen wird, die vollständigen Transkripte finden sich unter \url{https://github.com/cmlnet/polittalks}.}. Dieses Vorgehen ist in der Tat nicht vollkommen ideal, war aber der einzig gangbare Kompromiss angesichts der Materialfülle und -komplexität und dem eingeschränkten Zeitbudget.

Mit Hilfe dieses Vorgehens können zum einen sendungsbezogene Eigenschaften – Umgang mit Gästen, Einspieler, Sendungseröffnung und -abschluss, etc. – als auch die inhaltliche Präsentation und Aufbereitung des Themas Eurokrise erfasst und analysiert werden.

Entsprechend der Maxime, dass „der Umfang des Textkorpus ein möglichst praktikables Transkriptionsverfahren mit einer optischen Einprägsamkeit [erfordert], das die für  die Untersuchung irrelevanten Phänomene der gesprochenen Sprache konsequent ausklammert“ \parencites[81]{hoffmannPolitischeFernsehinterviewsEmpirische1982}[vgl. auch][169-174]{biereVerstehenUndBeschreiben1994}, wurde auf Basis des von \textcite[18-23]{dresingPraxisbuchTranskriptionRegelsysteme2011} vorgeschlagenen einfachen Transkriptionssystem transkribiert. Dieses System reicht aus die zur Beantwortung der Fragestellungen notwendigen Informationen zu erhalten und zugleich die Lesbarkeit und den Transkriptionsaufwand in einem angemessenen Rahmen zu halten. Eine Dokumentation der in den Transkripten benutzen Zeichen findet sich im Anhang auf Seite \pageref{chap:transkriptionsregeln}.

\section{Definition und Typologie der Talkshow}

\subsection{Die Talkshow im Allgemeinen}

Auf den ersten Blick mag es einfach erscheinen zu bestimmen welche Sendung eine Talkshow ist und welche nicht. Bei genauerer Betrachtung ist dies allerdings schwieriger als gedacht. So lassen sich zahlreiche Beispiele finden, bei denen entweder Talkelemente in eine Sendung integriert sind oder bei denen von den typischen Merkmalen einer Talkshow in einzelnen Punkten abgewichen wird \parencite[14]{kellerGeschichteTalkshowDeutschland2009}. Hinzu kommt, dass teilweise Sendungen als Talkshows deklariert werden, die – je nach Definition –  keine sind. Folge dieser Schwierigkeiten sind eine Vielzahl von Definitionen und Typologien.

Die sicherlich älteste Definition im deutschsprachigen Raum findet sich bei \textcite{barloewenGespraechMitGaesten1975}. Nach ihnen sind die drei konstitutiven Faktoren einer Talkshow (a) der Seriencharakter der Sendung, (b) die zentrale Rolle des Gastgebers und (c) das personen- und nicht themenbezogene Gespräch \parencite[18]{barloewenGespraechMitGaesten1975}. Das letzte Merkmal sei dabei das entscheidende Unterscheidungskriterium zu traditionellen Gesprächssendungen. Ähnlich nimmt \textcite[16]{hoeferTalkMenschlich1975} die Abgrenzung vor:

\begin{quote}
	„In einer Talk Show kann und soll jeder über alles reden, ohne dass alle alles und jeden verstehen müssen. Bei einer Diskussion sollte nur der mitreden, der über das Thema mehr weiß als andere, Diskussionsleiter einbegriffen.“
\end{quote}

Würde man diesen älteren Definitionen heute folgen, so könnte man die komplette Talkschiene der ARD, mit Ausnahme vielleicht von Beckmann und Menschen bei Maischberger, nicht als Talkshow bezeichnen. Damals musste die Talkshow von älteren, bereits etablierten Gesprächsformaten als neuartig abgegrenzt werden – und suchte sich auch selbst abzugrenzen. Eine aktuelle Definition hingegen hat die Veränderungen, die das Format Talkshow seit Aufkommen des Begriffs in den 1970ern bis heute hinter sich hat, zu berücksichtigen \parencites[101]{kalverkaemperKommentierteBibliographieZur1980}[13]{kellerGeschichteTalkshowDeutschland2009}.

Von der Wortbedeutung ausgehend, kommt man zu einer recht weiten Definition. Wört"-lich ge"-nommen sind Talkshows erst einmal nichts anderes als „televised broadcasts of conversation“ \parencite[329]{roseTalkShow1985}. Der Duden bestimmt sie allgemein als „Fernsehsendung, in der ein Moderator od. eine Moderation u. geladene Gäste miteinander [über ein Thema] sprechen“ \parencite[1047]{schochDudenDeutscheRechtschreibung2009}. Von einer fehlenden Themenfestlegung ist hier keine Rede mehr.

Ähnlich bei \textcite[134]{doernerPolitainmentPolitikMedialen2001}, welcher ebenfalls ganz grundlegend in der Talkshow „zunächst einmal ein zum Zweck der massenmedialen Verbreitung inszeniertes Gespräch, dessen primäre Funktion in der Unterhaltung des Publikums besteht“ sieht. Er weist damit auf drei zentrale Aspekte hin – die Verbreitung über Massenmedien, die Inszenierung und die Unterhaltungsfunktion.

Obwohl heute solche relativ breiten Definitionen vorherrschen\footnote{Weitere Definitionen finden sich etwa bei \textcite[10f.]{schichaTalkAufAllen2002}, \textcite[600f.]{eimerenTalkshowsFormateUnd1998} und \textcite[609]{kruegerThementrendsTalkshows90er1998}.}, ist nicht jeder Wortwechsel im Fernsehen eine Talkshow. Zu unterscheiden ist sie primär vom Interview. Bei Interviews ist die Rollenverteilung relativ streng festgelegt – ein Interviewer trifft auf einen Interviewten, der Ablauf besteht aus Frage-Antwort-Sequenzen \parencite[98f.]{heritageAnalyzingNewsInterviews1989}. Talkrunden sind dagegen im Grunde „Gruppengespräche“. Die Rollenverteilung zwischen Fragendem und Befragten – ausgenommen der Moderator als Diskussionsleiter – können ad hoc wechseln. Die Handlungsmuster sind vielfältiger und bestehen nicht nur aus der Abfolge Frage-Antwort, sondern vielmehr aus Rede-Gegenrede-Sequenzen\footnote{Hieran ändert auch der Umstand nichts, dass politische Gesprächsrunden sich in ihrem Verlauf durchaus zu einer Art Gruppeninterview entwickeln können \parencite[78ff.]{hollyPolitischeFernsehdiskussionenZur1986}.}. Ähnlich hält bereits \textcite[101]{mahloZurDiskussionUm1956} Interview und Diskussionssendung auseinander:

\begin{quote}
	„Befragt der eine den anderen, worauf der andere mehr oder weniger bestimmt antwortet, so handelt es sich um ein Interview. Auch hierbei geht es nicht um Auseinandersetzung, sondern lediglich um die Festlegung des Standpunktes, den der Befragte hat. Sprechen mehrere Menschen durcheinander, trotzdem aber miteinander, dann kann man von einer Unterhaltung oder von der Diskussion sprechen.“
\end{quote}

\subsection{Typologien}\label{chap:typologien}

Ähnlich zahlreich wie die Definitionen sind auch die Typologien die das Genre der Talkshows und seiner Subgenres weiter zu ordnen versuchen\footnote{Typologievorschläge finden sich z.B. bei \textcite[387]{richardsonSpecificDebateFormats2008}, \textcite{timbergTaxonomyTelevisionTalk2002}, \textcite[20]{kellerGeschichteTalkshowDeutschland2009}, \textcite[602]{eimerenTalkshowsFormateUnd1998} oder \textcite[56]{tenscherTalkshowisierungAlsElement2002}.}. Hier findet die Einteilung von \parencite[32f.]{plakeTalkshowsIndustrialisierungKommunikation1999} in \textit{Debattenshow / Forum}, \textit{Personality-Show} und \textit{Bekenntnisshow} Verwendung. Bei der \textit{Debattenshow} stehen „Politik und andere Fragen von öffentlichem Interesse“  \parencite[32]{plakeTalkshowsIndustrialisierungKommunikation1999} im Vordergrund. Man kann also auch von politischen Talkshows sprechen. Die \textit{Personality-Show} zeichnet sich hingegen dadurch aus, dass es hier um einzelne Menschen, meist Prominente, und ihre Persönlichkeit geht. Entsprechend wird hier des Öfteren auf eine Festlegung des Themas verzichtet und die Gäste selbst stehen im Vordergrund. Während die \textit{Bekenntnisshow} wohl am ehesten dem entspricht was als Daily-Talk bezeichnet wird, also der inszenierten Enthüllung intimer Gefühle und Vorgänge.

\subsection{Die politische Talkshow und ihre gesellschaftliche Rolle}
\label{sec:polittalk_rolle}

Wenn man dem berühmten Diktum Luhmanns Glauben schenken mag, dass wir das „[w]as wir über unsere Gesellschaft, ja über die Welt, in der wir leben, wissen, [\ldots] durch die Massenmedien wissen“ \parencite[9]{luhmannRealitaetMassenmedien2017}, dann ist es folgerichtig auch bei der Politikvermittlung und -darstellung den Massenmedien eine, wenn nicht die, zentrale Rolle zu zu billigen \parencite[37-45]{doernerPolitainmentPolitikMedialen2001}. Das Fernsehen nimmt dabei eine Schlüsselrolle ein, denn auch wenn die durchschnittliche Nutzungsdauer des Internets bei den 14 bis 29-jährigen bereits leicht über der des Fernsehens liegt, so bleibt dieses insgesamt betrachtet doch weiterhin das täglich am längsten genutzte Medium \parencite{ard/zdf-medienkommissionARDZDFOnlinestudie2012}. Hinzu kommt, dass auch wenn es um Informationsbeschaffung geht das Fernsehen am häufigsten genutzt wird\footnote{Dies sagt natürlich nichts über die Informationsleistung der jeweiligen Medien aus. Zurecht wird darauf hingewiesen, dass Hyptertextmedien eine „in der Regel aktivere und damit intensivere Informationsrezeption“ \parencite[17]{mendeMedienuebergreifendeInformationsnutzungUnd2012} erfordern.} \parencite[16]{mendeMedienuebergreifendeInformationsnutzungUnd2012}. Immer noch ist es also Leitmedium im Bereich der Politikdarstellung \parencite[275]{bucherMedienrealitaetPolitischenZur2004}.

Gleichzeitig sind die westlichen Gesellschaften gekennzeichnet durch eine Vielzahl heterogener, sich widersprechender Meinungen und Interessen, die zugleich in zahlreichen parallel existierenden diskursiven Arenen aufeinandertreffen und dort versuchen hegemonial zu werden \parencite[306]{nonhoffHegemonieanalyseTheorieMethode2010}. Polittalks können als eine dieser Arenen gesehen werden und versuchen als Fernsehformat ja auch „den Eindruck eines demokratischen Diskurses entstehen zu lassen, von freier Rede und Gegenrede also, vom Austausch der Argumente, bei dem die Persönlichkeit, der gesellschaftliche Rang, ja sogar der Beruf zurücktritt, um der Kraft des Wortes Platz zu machen“ \parencite[32]{plakeTalkshowsIndustrialisierungKommunikation1999}. Einige sehen in (politischen) Talkshows deshalb die Möglichkeit einer Art diskursiven Forums, in dem verschiedene, gegensätzliche Positionen aufeinanderträfen, ohne Versöhnung nebeneinander stehen blieben und so einen aktiven Rezipienten, der sich selbst zwischen den Positionen entscheiden muss, förderten \parencite[18]{tolsonTalkingTalkAcademic2001}. Zudem könnten in ihnen Akteure zu Wort kommen, die sonst in den Medien unterrepräsentiert sind \parencite[386]{richardsonSpecificDebateFormats2008}.

Dabei darf aber nicht vergessen werden, dass es sich nicht um reine Informationssendungen handelt. Auch die politische Talkshow wollte und will unterhalten und trug damit ihren Teil dazu bei, die strikte Trennung zwischen Information und Unterhaltung aufzulösen \parencite[100]{kalverkaemperKommentierteBibliographieZur1980}.

Während einige Autoren deshalb zu einer negativen Einschätzung des Formats gelangen\footnote{\textcite{postmanWirAmuesierenUns2003} sah in der Verbindung von Unterhaltung und Fernsehen noch den Untergang der westlichen Demokratie heraufziehen.}, wird hier die Position vertreten, dass Unterhaltung und Information nicht per se unvereinbare Gegensätze darstellen \parencites{bosshartInformationUndOder2007}[57-62]{doernerPolitainmentPolitikMedialen2001}. Es handelt sich vielmehr „um komplementäre Bestandteile jeglicher massenmedialer Kommunikation“ \parencite[70]{luenenborgUnterhaltungAlsJournalismus2007}. Dies ist besonders dann augenfällig, wenn man Unterhaltung als Kategorie der Rezeption betrachtet: Zuschauer werden auch beim Rezipieren von Formaten, die traditionell der reinen Informationsvermittlung zugeschlagen werden, unterhalten und diese Formate finden teils genau deshalb ein größeres Publikum \parencite[74ff.]{luenenborgUnterhaltungAlsJournalismus2007}. Im Umkehrschluss können klassische Unterhaltungsformate eben auch politische Informationen vermitteln \parencite[59f.]{doernerPolitainmentPolitikMedialen2001}. Man kann bei politischen Talkshows also von einer Art „journalistischen Unterhaltung“ sprechen. Manche Autoren sehen aus diesem Umstand heraus in politischen Talkshows das Potenzial ein sonst politisch uninteressiertes Publikum mit politischen Informationen zu konfrontieren  \parencites[240]{doernerPolitainmentPolitikMedialen2001}[51ff.]{bockPolitainmentImDeutschen2009}. Andere wiederum bestreiten dies mit mehr oder weniger guten Argumenten\footnote{Ein eher ungeeignetes Gegenargument kommt etwa von Norbert Lammert. Da die durchschnittliche Verweildauer der Zuschauer bloß sechs Minuten betrage, könne „der typische durchschnittliche Fernsehzuschauer dann gerade einen Diskutanten gesehen oder ein Argument gehört“ haben \parencite[120]{gaeblerInterviewMitDr2011}. Leider übersieht er dabei, dass der durchschnittliche Fernsehzuschauer ein Idealtyp ist, den es in der Realität nicht gibt. Tatsächlich werden einige Zuschauer einer Talkshow nur wenige Sekunden folgen, andere vielleicht bis zur Hälfte dran bleiben und einige weitere wiederum die komplette Zeit zusehen. Die durchschnittliche Verweildauer sagt folglich nichts darüber aus, inwiefern das Format geeignet ist unpolitische Personen an Politik heranzuführen. Dazu wäre eine soziodemographische Erforschung des Talkshowpublikums notwendig – solche Untersuchungen sind allerdings Mangelware. Hinzu kommt natürlich, dass in den heutigen Talks kaum ein Gast sechs Minuten ununterbrochen sein Argument darlegen kann.}.

Schon \textcite[116]{burgerDiskussionOhnRitual1989} stellt fest, „dass politische Diskussionen im Fernsehen hochgradig stereotypisiert sind, dass von 'Gespräch' im Sinne alltäglicher Dialoge – welchen Typs und welcher Intensität auch immer – kaum die Rede sein kann.“ Die Zwänge des Mediums, in dem sie stattfinden, prägen sie. Es gilt hier was für alle anderen Talkshow­arten ebenfalls gilt: „talk is produced in an institutional setting [\ldots] for, and oriented toward an 'overhearing audience'“ \parencite[28]{tolsonTalkingTalkAcademic2001}.

Mit der Figur der „overhearing audience“ ist gemeint, dass es sich bei Gesprächen und Diskussionen in den Massenmedien immer und hauptsächlich um für ein Publikum produzierte Gespräche handelt \parencite[99f., 112-116]{heritageAnalyzingNewsInterviews1989}. Dadurch entsteht ein Adressierungsdilemmata, da die Talkgäste immer zu drei Kommunikationskreisen sprechen – zu den anderen Gästen und dem Moderator, zum Studiopublikum und zu den Zuschauern daheim. Spräche der Gast offensichtlich nur zu einem der Kreise, würde dies sanktioniert werden \parencite[289ff.]{bucherMedienrealitaetPolitischenZur2004}.

Für Politiker bieten diese Formate dabei zahlreiche Vorteile – die Verknüpfung von Positionen und der eigenen Person, die weiterhin relativ hohe Reichweite, die im Vergleich relativ freie Möglichkeit zu Wort zu kommen, die folgende Anschlusskommunikation in anderen Medien, aber auch einige Nachteile – durch den Livecharakter und den anwesenden Moderator ist der Verlauf schwieriger zu kontrollieren, ein gewisses Maß an Schlagfertigkeit, Spontaneität und formatgerechter Selbstdarstellung ist nötig \parencites[326f.]{nielandTalkshowisierungWahlkampfesAnalyse2002}[287]{bucherMedienrealitaetPolitischenZur2004}.

Dabei ist mit \textcite[152]{faircloughPoliticalDiscourseMedia1998} festzuhalten, dass „programmes on radio and television [\ldots] not only manifest a political order of discourse which transcends them, but also actively contribute to its constitution and transformation“. Medien müssen also nicht nur „als Bühne einer Inszenierung der Politik herhalten“, sondern können auch „selbst politische Realitäten inszenieren“ \parencite[271]{bucherMedienrealitaetPolitischenZur2004}. In Talkshows treffen die Logik von Politik und Journalismus sowie die ökonomische Logik der Medien, die mit einer günstigen Sendeform hohe Einschaltquoten erreichen wollen, quasi direkt und unvermittelt aufeinander \parencite[16-20, 41]{bucherLogikPolitikLogik2007}.

Politische Talkshows sind folglich komplexe Kommunikationsarrangements in denen verschiedene (politische) Akteure aufeinandertreffen und ihre jeweilige Position möglichst gut „verkaufen“ wollen. Es sind inszenierte Diskussionen für das Publikum vor den Fernsehern, bei denen aber auch Redaktion und Moderation durch Themen- und Gäste"-auswahl und Gesprächsführung Einfluss auf die mediale Behandlung gesellschaftlicher Dis"-kurse nehmen kann und damit letztlich auch diese selbst beeinflusst.

\section{Anforderungen an politische Talkshows}\label{chap:anforderungen}

Entsprechend ihrer Rolle im medialen Vermittlungsprozess sollten politische Talkshows bestimmten Qualitätsanforderungen genügen, insbesondere wenn sie von öffentlich-""rechtlichen Sendern verantwortet werden\footnote{Da die privaten Sender stärker den Marktgesetzen unterworfen sind und sich an hohen Einschaltquoten orientieren müssen, können an öffentlich-""rechtliche Sender höhere Ansprüche gestellt werden.}. Diese Anforderungen lassen sich aus mehreren Quellen ableiten – aus der Rolle von Massenmedien in modernen Demokratien, aus etablierten journalistischen Qualitätsstandards oder aus den rechtlichen Vorschriften, die in der Bundesrepublik die Aufgaben des Rundfunks regeln\footnote{Den rechtlichen Rahmen für den öffentlich-rechtlichen Rundfunk bilden in Deutschland vor allem die entsprechenden Rundfunkurteile des Bundesverfassungsgerichts, der Rundfunkstaatsvertrag sowie die entsprechenden Staatsverträge zwischen den Ländern und den jeweiligen Rundfunkanstalten respektive der ZDF-Staatsvertrag. Artikel 5 des Grundgesetzes bildet dabei sozusagen den Grundpfeiler durch die Garantie der Meinungs- und Pressefreiheit.}. Problematisch an solche einer Vielfalt von Beurteilungsgrundlagen ist, dass diese, „zwischen verschiedenen gesellschaftlichen Gruppen und Subsystemen – teilweise sogar innerhalb dieser – umstritten“ sein können \parencite[691]{schatzQualitaetFernsehprogrammenKriterien1992}. Das heißt es kann nicht per se von einem existierenden Konsens über die Normen, die an mediale Produkte anzulegen sind, ausgegangen werden. \textcite{schatzQualitaetFernsehprogrammenKriterien1992} versuchen dieses Problem zu lösen, indem sie sich einzig auf rechtlich kodifizierte Vorgaben beziehen, da diese einen „verbindlichen Orientierungsrahmen“ \parencite[691]{schatzQualitaetFernsehprogrammenKriterien1992} vorgäben. Diesem Vorschlag soll hier, trotz seiner Unzulänglichkeiten\footnote{Diese Lösung ist eigentlich nur eine scheinbare, zwar sind die Gesetze tatsächlich ein „verbindlicher Orientierungsrahmen“, allerdings nur da bei einem Verstoß gegen sie Strafe droht. Sie basieren also nicht per se auf einem „gesellschaftlichen Konsens“ wie es \textcite{schatzQualitaetFernsehprogrammenKriterien1992} suggerieren, sondern letztlich auf Zwang und können innerhalb der Gesellschaft sehr wohl umstritten sein. Es ist sowieso relativ unwahrscheinlich, dass in pluralen Gesellschaften Qualitätsstandards jemals vollkommen unumstritten sein können.}, gefolgt werden.

Dazu werden die von \textcite[692f.]{schatzQualitaetFernsehprogrammenKriterien1992} in ihrer Untersuchung der Rechtstexte identifizierten fünf Qualitätsdimensionen – (1) Pluralität, (2) Relevanz, (3) Journalistische Professionalität\footnote{\textcite[702]{schatzQualitaetFernsehprogrammenKriterien1992} unterscheiden zwischen inhaltlicher und gestalterischer oder ästhetischer Professionalität. Da letztere jedoch vorrangig bei fiktionalen Sendungen eine Rolle spielen, ist hier nur von der inhaltlichen bzw. konkreter der journalistischen Qualität die Rede.}, (4) Akzeptanz, (5) Rechtmäßigkeit – als Ausgangsbasis übernommen und an die Spezifika politischer Talkshows angepasst. Die Dimensionen vier und fünf werden von vornherein verworfen, da sie eine andere Art von Untersuchung erfordert hätten\footnote{Um die Akzeptanz politischer Talkshows zu messen, wäre es beispielsweise nötig gewesen zusätzlich Zuschauerbefragungen durchzuführen, was den Rahmen dieser Arbeit gesprengt hätte. Nimmt man den unzureichenden Indikator Zuschauer- bzw. Marktanteil zur Hand, dann kann davon ausgegangen werden, dass die Sendungen – für die Zahlen vorliegen – dieses Kriterium erfüllen (\textbf{vgl. Tabelle X}). Zur Beurteilung der Rechtmäßigkeit wären dagegen entsprechende juristische Gutachten erforderlich gewesen.}.

\subsection{Pluralität}

Der Rundfunkstaatsvertrag (RStV) spricht den öffentlich-rechtlichen Rundfunkanstalten die Aufgabe zu „als Medium und Faktor des Prozesses freier individueller und öffentlicher Meinungsbildung zu wirken“ und „einen umfassenden Überblick über das internationale, europäische, nationale und regionale Geschehen in allen wesentlichen Lebensbereichen zu geben“ \parencite[§11 Abs. 1]{o.a.Rundfunkstaatsvertrag1991} und findet sich in allen Rundfunkgesetzen bzw. -staatsverträgen \parencite[693]{schatzQualitaetFernsehprogrammenKriterien1992}.

Um diese Aufgabe zu erfüllen, müssen die Programme vielfältig und repräsentativ sein. Zum einen muss eine strukturelle Pluralität gegeben sein, also verschiedene Programmsparten und -formen. Zum anderen – und dies ist im Weiteren relevant – ist auch eine inhaltliche Pluralität erforderlich. Dies bedeutet unter anderem, dass den verschiedenen gesellschaftlichen und politischen Interessen Rechnung getragen werden muss, und zwar sowohl auf der Ebene der behandelten Themen als auch der vorkommenden Akteure \parencite[693-695]{schatzQualitaetFernsehprogrammenKriterien1992}.

Gemeinhin wird dieser Anspruch zwar am Gesamtprogramm geltend gemacht, da Polittalks allerdings politischen und gesellschaftlichen Interessen eine direkte mediale Bühne bieten, ist es sinnvoll dieses Kriterium auch auf einzelne Sendungen zu beziehen. Folglich ist auch von politischen Talkshows zu verlangen, dass sie die existierende Meinungsvielfalt versuchen abzubilden – konkret also eine entsprechende vielfältige Themen- und Gästestruktur vorweisen, die die tatsächliche heterogene Gestalt moderner Gesellschaft widerspiegelt und es so vermeidet ein verzerrtes Bild der Realität zu vermitteln \parencite[53–57]{plakeTalkshowsIndustrialisierungKommunikation1999}.

Als Indikator für eine vielfältige Gästestruktur können nicht nur gesellschaftliche Funktion und Parteizugehörigkeit dienen, sondern auch Geschlecht und Alter der Talkgäste. So haben sich sowohl der NDR \parencite[126]{schulzGesetzessammlungInformationKommunikation2012}, der WDR \parencite[§7 Abs. 4]{wdrGesetzUeberWestdeutschen2011} als auch das ZDF \parencite[2]{zdfRichtlinenFuerSendungen2009} verpflichtet die Gleichstellung von Mann und Frau zu fördern.

\subsection{Relevanz}

Da die Einschätzung der Relevanz eines Themas stark vom individuellen Standpunkt des Rezipienten und damit von heterogenen Interessen, Normen und Werten abhängt \parencite[698]{schatzQualitaetFernsehprogrammenKriterien1992}, bestimmt sie sich situativ in Bezug auf bestimmte Ebenen. Das heißt in Bezug auf den individuellen Rezipienten (Mikro-), bestimmte Organisationen oder Gruppen (Meso-) oder die Gesellschaft (Makroebene) \parencite[696]{schatzQualitaetFernsehprogrammenKriterien1992}. Entsprechend gibt es eine Vielzahl möglicher Indikatoren, um die Relevanz eines Themas auf einer bestimmten Ebene zu ermitteln, wobei jeder dieser Indikatoren seine eigenen Schwächen hat \parencite[697-701]{schatzQualitaetFernsehprogrammenKriterien1992}.

In dieser Analyse soll nicht versucht werden, die Relevanz etwa über den „Nachrichtenwert“ \parencite[237, passim]{lippmannOeffentlicheMeinung1990} der behandelten Themen umfassend herauszuarbeiten\footnote{Dieses Unterfangen würde aufgrund der dafür notwendigen inter- und intramedialen Vergleiche und Analysen für eine eigenständige Arbeit ausreichen.}, stattdessen, soll eine Annäherung über das Themenprofil der Sendungen geschehen. Da alle untersuchten Talksendungen den – mehr oder weniger starken – Anspruch haben zur politischen Willensbildung beizutragen, sind für sie Themen aus den Bereichen Politik und Wirtschaft relevanter als etwa jene aus dem Bereich Personality und Prominenz\footnote{Einschränkend sei darauf hingewiesen, dass in Polittalks auch vermeintlich unpolitische Themen wie Gesundheit mit gesamtgesellschaftlichem und politischem Fokus, etwas in Hinblick auf das Gesundheitssystem, behandelt werden können \parencite[86]{plakeTalkshowsIndustrialisierungKommunikation1999}.}.

Zu berücksichtigen ist auch, dass sich Relevanz und Pluralität gegenseitig beeinträchtigen können. So kann beispielsweise ein Thema an sich zwar – zumindest in den Augen der jeweiligen Redaktion – eine hohe gesellschaftliche Relevanz haben, wenn nun aber alle Talkshows über einen längeren Zeitraum nur noch Sendungen zu diesem einen Thema produzieren, wird kaum noch dem Vielfaltsanspruch Rechnung getragen.

\subsection{Journalistische Professionalität}

„Berichterstattung und Informationssendungen haben den anerkannten journalistischen Grundsätzen [\ldots] zu entsprechen“, so fordert §10 des RStV. Der Moderator hat dabei eine zentrale Rolle und zugleich eine Doppelfunktion inne. Zum einen agiert er als Gastgeber, er organisiert das Gespräch, indem er das Rederecht verteilt, neue Themen einbringt, zwischen den Positionen vermittelt und für Kohärenz sorgt (\textit{Gastgeberfunktion}). Zum anderen soll er aber auch als kritischer Journalist handeln und quasi als Stellvertreter der Zuschauer bzw. der Öffentlichkeit die Gäste, insbesondere Politiker, mit den Fragen, den Ängsten, der Kritik und dem Informationsinteresse jener konfrontieren (\textit{anwaltschaftliche Funktion}) \parencites[452f.]{schrottElefantenUnterSich1996}[25]{bucherLogikPolitikLogik2007}.

Gleichzeitig wird von ihm Neutralität und Ausgewogenheit gefordert. Neutralität, indem er sich nicht mit einem Gesprächsteilnehmer gemein macht oder versucht „durch falschen Zungenschlag die Polarität aufzuheben und die Spannung zugunsten einer einseitigen Meinungsfestlegung zu vernichten“ \parencite[106]{mahloZurDiskussionUm1956}. Die Forderung nach Ausgewogenheit überträgt quasi das Kriterium der Pluralität auf die Handlungsebene der Sendungen:

\begin{quote}
	„Ausgewogene Darbietung heißt [$\ldots$], dass möglichst alle in der öffentlichen Diskussion des Themas vorgetragenen Argumente und Standpunkte be"-rück"-sich"-tigt werden. Das impliziert auch, dass die jeweiligen Interessengruppen und ihre Repräsentanten angemessen zu Wort kommen, dass ferner die Sicht der unmittelbar Betroffenen [\ldots] berücksichtigt wird.“ \parencite[704]{schatzQualitaetFernsehprogrammenKriterien1992}
\end{quote}

Hinsichtlich Talkshows bedeutet dies, über das bereits gesagte hinaus, dass es prinzipiell positiv zu bewerten ist, wenn „Normalbürger“, als Betroffene, zu Wort kommen, um ihre Sicht darzulegen. Allerdings gilt es zu beachten, dass sich hierfür nicht jedes Thema eignet und der Betroffene auch nicht Selbstzweck sein kann, es also Aufgabe des Moderators ist dessen Beiträge und Einsichten in die Diskussion sinnvoll zu integrieren. Dies macht skeptisch gegenüber Institutionen wie dem 'Betroffenensofa' bei \textit{Anne Will}, wird hier der anwesende „Normalbürger“ doch quasi aus der Diskussion ausgeschlossen.

An dieser Stelle wird deutlich, dass auch die bisher nicht erwähnte Redaktion Einfluss nehmen kann, unter anderem durch Einspieler. Mit ihnen kann sowohl Verständlichkeit erzeugt oder Politiker mit Fakten konfrontiert werden als auch im negativen Sinne „Meinung gemacht“ werden. Sie geben dem Moderator ein Werkzeug an die Hand, um die Diskussion wieder in Gang zu bringen oder in neue Bahnen zu lenken. Aber auch die in neueren Untersuchungen kaum beachtete Bildregie ist ein wichtiges Instrument der Redaktion. Durch geschickte Schnitte und Perspektivenwechsel kann der Talkrunde Lebendigkeit eingehaucht, als auch Teilnehmer in ein schlechtes Licht gerückt werden \parencite[107]{mahloZurDiskussionUm1956}.

\subsection{Zusammenfassung}

Aus den drei Qualitätsdimensionen – Vielfalt, Relevanz, Journalistische Professionalität – lassen sich folgende Kriterien zur Bestimmung der Qualität einer politischen Talkshow ableiten:

Die \textbf{Gäste} sollten von den Redaktionen so ausgewählt werden, dass sie die pluralen und gegensätzlichen gesellschaftlichen und politischen Interessen wiedergeben und damit visualisieren. Dies gilt auch für das Geschlechterverhältnis und die Altersstruktur der Gäste, die sich an den tatsächlichen Gegebenheiten orientieren sollten.

Die \textbf{Themen} sollten ebenfalls vielfältig sein. Themenballung oder -wiederholungen – insbesondere in zeitlicher Nähe zu anderen Polittalks – sollten vermieden werden, es sei denn sie tragen Neues zum Diskurs bei. Die abgedeckten Themenbereiche sollten für den selbstgesteckten Anspruch der Polittalks relevant sein, also hauptsächlich aus den Bereichen Politik und Wirtschaft stammen.

Die \textbf{Moderation} sollte sich mit keinem Standpunkt gemein machen, sondern durch entsprechende Kommunikationsstrategien die Talkgäste kontrollieren, ihre Aussagen kritisch hinterfragen und die zugrundeliegenden Sachverhalte dem Zuschauer verständlich machen. Dies gilt genauso für die von der jeweiligen Talkredaktion verantworteten Elemente, wie Einspieler oder Bildregie.

Dabei gilt, dass die Maßstäbe desto strenger anzulegen sind, je eher eine Sendung dem Typus der \textit{Debattenshow} entspricht.

\section{Die Eurokrise als journalistische Herausforderung}

Bereits angesprochen wurde, dass die Eurokrise quasi als Aufhänger für die qualitative Analyse dienen soll. Dies ist keine willkürlich Entscheidung, vielmehr ist die Eurokrise eine genuine journalistische Herausforderung und damit auch eine Art Lackmustest für politische Talkshows. Außerdem ist sie das am häufigsten behandelte Thema im Untersuchungszeitraum (vgl. Kapitel \vref{chap:themencluster}).

Mit „Eurokrise“ wird der Umstand bezeichnet, dass es einigen Ländern der Eurozone gar nicht oder nur mehr schwer möglich ist, ihre Staatsschulden ohne externe Hilfen zu finanzieren. Die genauen Ursachen der Krise sind politisch hoch umstritten, angeführt werden beispielsweise die vorausgegangene Finanzkrise von 2007 und die damit verbundenen Bankenrettungen 2008/09, eine unverantwortliche Haushaltspolitik der betroffenen Staaten oder ökonomische Ungleichgewichte durch die gemeinsame Währung. Deswegen wird bereits die Bezeichnung der Eurokrise als „Staatsschuldenkrise“ in Zweifel gezogen, da es sich hier um einen „Coup der Strategen der Finanzindustrie und der mit ihnen verbundenen Politik und Medien“ \parencite{mullerEuroKriseLugeSystemrelevanz2011} handele, um von der eigenen Verantwortung abzulenken.

Als Startpunkt der Krise wird allgemein der Oktober 2009 ausgemacht, als nach einem Regierungswechsel der tatsächliche Schuldenstand Griechenlands offenbar wurde und Europäische Union (EU) und Internationaler Währungsfond (IWF) das Land schließlich im Frühjahr 2010 vor der Zahlungsunfähigkeit bewahren mussten \parencites{deutschewelleStationenKrise200920112012}{kaufmannSchummelGriechenMachenUnseren2012}. Andere Eurostaaten gerieten ebenfalls in Schwierigkeiten, so dass bis dato fünf Länder zu den am stärksten betroffenen gezählt werden – Irland, Zypern, Portugal, Spanien und Slowenien. Es folgten in der Zwischenzeit zudem zahlreiche Krisengipfel der EU-Staaten und Rettungsmaßnahmen, unter anderem die Einrichtung der Europäischen Finanzstabilisierungsfazilität (EFSF) und des Europäischen Stabilisierungsmechanismus (ESM) \parencite[24–37]{kaufmannSchummelGriechenMachenUnseren2012}.

Zusammenfassend betrachtet ist die Eurokrise ein hochdynamisches, komplexes und politisch stark umstrittenes Thema mit potenziell weitreichenden Folgen für die im Euroraum lebenden Menschen und entsprechend hohem Diskussionsbedarf in der Bevölkerung. Den Massenmedien kommt eine besondere Verantwortung zu, nicht nur wegen ihrer zentralen Rolle in der heutigen Welt (vgl. Kapitel \vref{sec:polittalk_rolle}), sondern auch wegen ihres Versagens in der Berichterstattung über die globale Finanzkrise ab 2007 \parencites{kohlerImBlindflugDurch2009}{thomaSelfFulfillingProphecyDilemma2009}{feussPopularisierungsdilemma2009}{schechterJournalismusHatTeilen2009}\footnote{Umfassende Untersuchungen über die Rolle der Medien in der Finanz- und Eurokrise stehen bislang noch aus.}.