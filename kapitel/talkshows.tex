\chapter{Politische Talkshows in Deutschland}
\label{chap:polittalks}

Jedes Medium und jedes Genre hat seine Geschichte, seine Ursprünge, seine Genealogie und seine Traditionen, die sie bis heute prägen. Um die aktuellen Talkshows verstehen und analysieren zu können, ist es sinnvoll sich kurz mit der Geschichte dieser Sendungsform zu beschäftigen\footnote{Für eine umfassende Darstellung der Geschichte des gesamten Genres sei auf \textcite{kellerGeschichteTalkshowDeutschland2009} verwiesen. Für die USA kann \textcite{timbergTelevisionTalkHistory2002} zu Rate gezogen werden.}.

\section{Eine kurze Geschichte des (Polit-) Talks}

\subsection{Der Beginn: Erste Talkshows in den USA}

Erfunden wurde das Genre Talkshow in den USA, einem Land, in dem mit den „town meetings“ bereits ein tradierter Mythos der öffentlichen Debatte existierte \parencite[24-28]{kellerGeschichteTalkshowDeutschland2009} und der Öffentlichkeit von Beginn an eine große Macht zukam \parencite[38ff.]{plakeTalkshowsIndustrialisierungKommunikation1999}. Als wahrscheinlich erste (politische) Talkshow kann die Sendung \textit{Meet the Press} gelten. 1945 zuerst im Hörfunk gestartet, wurde sie zwei Jahre später für das Fernsehen adaptiert und hatte bereits damals einen Einfluss auf die politisch-mediale Agenda:

\begin{quote}
	„Im Rahmen der Sendung befragten unter der Gesprächsleitung von Martha Roundtress wechselnde Journalisten jeweils einen Gast, zumeist aus dem politischen Bereich. [\ldots] und es geschah nicht selten, dass im Zuge der Diskussionen Äußerungen fielen, die montags die Schlagzeilen der Tageszeitungen bestimmten.“ \parencite[30]{kellerGeschichteTalkshowDeutschland2009}
\end{quote}

Im Hörfunk wuchs und differenzierte sich das Genre auch zuerst aus und es entstanden eine Reihe von Sendungen mit jeweils speziellen Zielgruppen. Mit wachsender Popularität des Fernsehens wurden diese dann oftmals in selbiges verlagert \parencite[30f.]{kellerGeschichteTalkshowDeutschland2009}. Konnten diese ersten Sendungen noch unter das Rubrum \textit{public affairs shows} gefasst werden, die sich vornehmlich mit öffentlichen Angelegenheiten befassten, so entstanden in den folgenden Jahrzehnten eine Vielzahl neuer  neuen Talkshowformate – die Daytime-Talkshows (\textit{Oprah Winfrey}) oder das auf Krawall und Eskalation ausgerichtete Confrontainment (\textit{Jerry Springer}) –, die sowohl das behandelte Themen- und Gästespektrum als auch die Darbietungsformen enorm erweiterten \parencite[48-59]{kellerGeschichteTalkshowDeutschland2009}.

Obwohl die Entwicklung in den USA zeitlich jener in der BRD voraus war, so ist, laut \textcite[59]{kellerGeschichteTalkshowDeutschland2009}, „die Talkshow des deutschen Fernsehens kein reines Importprodukt, sondern verfügt über eine eigene Entstehungsgeschichte“. Allerdings wurde die Entwicklung in den USA früh vom deutschen Nachkriegsfernsehen beobachtet und adaptiert \parencite[104-110]{mahloZurDiskussionUm1956}.

\subsection{Deutschland: Erste Anfänge im Hörfunk}

Bereits Ende der 1920er Jahre wurde im Hörfunk mit dialogischen Sendungsformen experimentiert, um so durch „Zwie-“ oder „Vielgespräche“ einem möglichst breiten Publikum unterhaltend Informationen und Bildung zu vermitteln \parencite[61ff.]{kellerGeschichteTalkshowDeutschland2009}:

\begin{quote}
	„Das Mehrgespräch und der kontradiktorische Vortrag waren Programmformen, die dem Verlangen nach Pluralität auf überzeugende Weise entsprachen und bei der Politik auf Akzeptanz stießen.“ \parencite[66]{kellerGeschichteTalkshowDeutschland2009}
\end{quote}

\subsection{Nachkriegszeit: Rundfunkdiskussionen als Demokratieerziehung}

Nach der Gleichschaltung des Rundfunks in der NS-Zeit \parencite[67-69]{kellerGeschichteTalkshowDeutschland2009}, knüpfte man nach dem Zweiten Weltkrieg wieder am (politischen) Bildungsgedanken an und wollte den Rundfunk in den jeweiligen Besatzungszonen zur Demokratieerziehung nutzen \parencite[71]{kellerGeschichteTalkshowDeutschland2009}, für diesen Zweck erachtete man freie Diskussionssendung als eine qua Prinzip geeignete Sendungsform \parencite[82]{kellerGeschichteTalkshowDeutschland2009}.

\textcite[86]{kellerGeschichteTalkshowDeutschland2009} resümiert, dass in dieser frühen Nachkriegsphase „das Hörfunk"-ge"-spräch Funktionen einer politischen Pädagogik und einer allgemein informierenden Publizistik“ erfüllt habe, gleichzeitig sei es aber „auch bereits in hohem Maße dem Begehren der restituierten oder neugegründeten politischen Parteien ausgesetzt [gewesen], die ihre Standpunkte im Medium Funk in eigener Regie und tunlichst unkommentiert kundtun“ wollten – ein bis heute ein oft geäußerter Kritikpunkt auch an Fernsehtalks.

\subsection{Der Sprung ins Fernsehen}

Ähnlich wie im Radio wurden auch im Fernsehen quasi seit Beginn des Sendebetriebs  1950 mit Gesprächsformaten experimentiert \parencite[87]{kellerGeschichteTalkshowDeutschland2009}. Ebenfalls früh fing man an das Gespräch und die Diskussion im Fernsehen zu reflektieren, wofür exemplarisch Mahlos bis heute aktueller Text „Zur Diskussion um die Diskussion im Fernsehen“ \parencite{mahloZurDiskussionUm1956} stehen kann. In diesem versucht er unter anderem die verschiedenen Arten von Gesprächssendungen im Fernsehen zu definieren \parencite[100f.]{mahloZurDiskussionUm1956} und macht sich Gedanken über Themenwahl, Teilnehmerzahl und Zusammensetzung, die Rolle des Diskussionsleiters und der Regie \parencite[104-107]{mahloZurDiskussionUm1956}. Auch berücksichtigt Mahlo in seinen Überlegungen bereits die Besonderheiten des Mediums Fernsehen und weist auf die Rolle von Studioaufbau, Erscheinung, Mimik und Gestik der Diskutanten hin, welche „der Diskussion eine Plastizität [geben], die sich bei anderen Massenmedien nicht erhalten lässt“ \parencite[102]{mahloZurDiskussionUm1956} und womit der Zuschauer „in begrenztem Maß 'Mitwirkender' [werde], der dem gesprochenen Wort folgen, seine Reaktionen verfolgen und bereits Ausgesprochenes auch in schweigenden Diskussionsteilnehmern noch einmal nachträglich feststellen“ \parencite[103]{mahloZurDiskussionUm1956} könne. Zweck der Diskussionen sei „eine Mischung aus Bildung, Unterhaltung und Information“ \parencite[103]{mahloZurDiskussionUm1956}. Damit werden bereits wesentliche Merkmale heutiger Talkshows – insbesondere die Kombination von Information und Unterhaltung – vorweggenommen.

\subsection{Der deutsche Talkshowklassiker: Der Internationale Frühschoppen}

Ähnlich wie bei dem amerikanischen Polit-Talkshowklassiker \textit{Meet the Press} wurde auch das deutsche Gegenstück – \textit{Der Internationale Frühschoppen}\footnote{Es ist müßig an dieser Stelle zu diskutieren, ob \textit{Der Internationale Frühschoppen} als Talkshow gelten kann oder nicht. Im Sinne der anfangs erwähnten Definition kann er es, auch wenn der Moderator selbst seine Sendung nicht als solche verstanden wissen wollte \parencite[112]{kellerGeschichteTalkshowDeutschland2009}.} – zuerst im Radio ausgestrahlt. In der erstmals 1952 ausgestrahlten und anderthalb Jahre später auch im Fernsehen stattfindenden Sendung, diskutierten ausländische Korrespondenten aktuelle politischen Themen auf deutsch und unter Leitung von Werner Höfer. Gesendet wurde wöchentlich am Sonntagmorgen \parencite[113f.]{kellerGeschichteTalkshowDeutschland2009}.

Waren Höfer und seine Sendung anfangs bei Presse und Publikum beliebt \parencite[115ff.]{kellerGeschichteTalkshowDeutschland2009}, so kamen mit der Zeit Vorwürfe auf, die stark an die heutige Kritik an Talkshows erinnern. Zu häufig würden die gleichen Gäste oder aber solche, die wenig beizutragen hätten eingeladen, wichtige Themen würden nicht behandelt, der Moderator Höfer vermeide wirkliche Auseinandersetzungen, verfolge eine eigene politische Agenda und stelle sich selbst zu sehr in den Vordergrund \parencite[49-58]{o.a.WernerHoeferSchau1959}. Dennoch konnte sich die Sendung über 35 Jahre lang im deutschen Fernsehen behaupten \parencite[119]{kellerGeschichteTalkshowDeutschland2009}.

In den Folgejahren kam es, wiederum vergleichbar mit den USA, zu einer Ausdifferenzierung des Genres \parencite[128-139]{kellerGeschichteTalkshowDeutschland2009}. \textcite[157]{kellerGeschichteTalkshowDeutschland2009} beschreibt die Rolle der Gesprächssendungen dieser Zeit wie folgt:

\begin{quote}
	„[Sie] erfüllen publizistische Funktionen: hier werden Informationen gereicht und die für eine immer noch im Entstehen begriffene Demokratie wichtigen tagesaktuellen und gesellschaftspolitischen Fragen erörtert. Das Fernsehen eröffnet der Meinungsbildung neue Möglichkeiten, die mit Ausweitung der Seherschaft an Bedeutung gewinnen.“
\end{quote}

\subsection{1960er: Mehr Konfrontation}

In den 1960er Jahren kamen weitere Sendungen hinzu, die sich teilweise recht lange im Programm halten konnten oder aber heutige Nachfolger prägten. Den Anfang machte 1963 das ZDF mit \textit{Journalisten fragen – Politiker antworten}. Unter Leitung von Reinard Appel befragten wechselnde Journalisten Vertreter der im Bundestag vertretenen Parteien. Die Modalitäten der Sendung waren zuvor eigens per Vertrag zwischen den Parteien und dem ZDF festgelegt worden \parencite[175-178]{kellerGeschichteTalkshowDeutschland2009}. Diese Parteinähe führte dazu, „dass die Reihe zeit ihres Bestehens und ebenso die Nachfolgesendung \textit{Bonner Runde} in bruchloser Genealogie in wesentlichem Maße den Anliegen der in den Parlamenten vertretenen Parteien verpflichtet blieb“ \parencite[188]{kellerGeschichteTalkshowDeutschland2009} und andere gesellschaftliche Akteure nicht vorkamen.

Im Gegensatz zu den frühen Gesprächssendungen, zeigte sich bei \textit{Journalisten fragen – Politiker antworten} bereits ein heute wohl bekanntes Dilemma: Sowohl Sender als auch Parteien legten Wert auf hohe Einschaltquoten – erreicht wurden diese jedoch nur bei Einladung bereits wohl bekannter Politiker, sodass junge und noch unbekannte Politiker wenig Aussicht auf Einladung hatten \parencite[181]{kellerGeschichteTalkshowDeutschland2009}. Interessant ist, dass die Beliebtheit der Sendung bei den Zuschauern nicht zuletzt auf Personalisierung und dem Versprechen, eine Innenansicht auf den politischen Betrieb zu erhalten aufbaute \parencite[183]{kellerGeschichteTalkshowDeutschland2009}.

1968 schuf die ARD mit \textit{Pro + Contra} eine stark ritualisierte und reglementierte politische Debattensendung. In dieser standen sich jeweils zwei Lager gegenüber die „ein vorgegebenes Thema von aktueller Brisanz“ \parencite[169]{kellerGeschichteTalkshowDeutschland2009} verhandelten. Der Ablauf folgte einer starren Struktur, die ein freies Gespräch unmöglich machte. Am Anfang und Ende der Sendung konnte das Studiopublikum zusätzlich für eine der beiden Seiten abstimmen \parencite[169ff.]{kellerGeschichteTalkshowDeutschland2009}. Die Sendung „etablierte für diese programmgeschichtliche Phase eine Form der TV-Debatte – mit den Kernelementen Personalisierung, Polarisierung, Spannungsdramaturgie –, die später in Sendungen wie \textit{Der heiße Stuhl} (RTL) zumindest dem Ansatz nach in allerdings weniger reglementierter und damit lebhafterer Inszenierung praktiziert wurde“ \parencite[171]{kellerGeschichteTalkshowDeutschland2009}. Die Sendung spiegelte durch ihre mehr auf Konfrontation angelegte Art, die von großen Konflikten geprägte Entwicklung der bundesdeutschen Gesellschaft in den 1960er Jahren wider und unterschied sich darin von ihren Vorläufern in den 50er Jahren.

\subsection{1970er: Die erste deutsche Talkshow}

Alle bisher erwähnten Sendungen wurden zum Zeitpunkt ihres Entstehens nicht als Talkshows bezeichnet. Die erste öffentlich als Talkshow wahrgenommene und so bezeichnete deutsche Sendung war \textit{Je später der Abend}, die erst im März 1973 Premiere hatte. Vorbild war die US-amerikanische \textit{Dick Cavett Show} \parencite[226]{kellerGeschichteTalkshowDeutschland2009}.

Bereits 1974 ging \textit{III nach 9} auf Sendung, das heute noch als \textit{3 nach 9} bei Radio Bremen gesendet wird. Erstmals eingesetzt wurden dort Einspielfilme, bei den heutigen Polittalks Standard. Ebenfalls wegweisend war die Suche nach Provokation und Eskalation durch Gäste- und Themenwahl \parencite[242-246]{kellerGeschichteTalkshowDeutschland2009}. Dem standen Stimmen entgegen, die meinten, „dass eine konfrontative Gesprächsführung dem angestrebten Ziel, Politik und die Motive ihrer Vertreter transparent zu machen, eher entgegenstanden“ \parencite[241]{kellerGeschichteTalkshowDeutschland2009} und stattdessen das lockere Gespräch in den Talkshows geeigneter sei.

Auch in die 1970er Jahren hatten Gesprächssendungen, nun meist als Talkshows bezeichnet, weiterhin Konjunktur. Zurück geführt werden kann dies unter anderem auf die weiterhin schwelenden gesellschaftlichen Konflikte, die das Verlangen nach öffentlicher Diskussion beförderten.

\subsection{1980er: Krawall und Remmidemmi}

Die Zulassung privater Fernsehsender 1984 führte zu neuen politischen Talkformaten. „Innovativ“ waren vor allem zwei Sendungen – \textit{Explosiv – Der heiße Stuhl} und \textit{A.T. - Die andere Talkshow} (beide RTL, 1989). Sie brachten das Konzept des Confrontainment aus den USA nach Deutschland. Zwar hatten, wie bereits erwähnt, auch frühere Sendungen der Öffentlich-Rechtlichen ein konfrontatives Sendungskonzept. Die neuen Sendungen waren allerdings ganz bewusst auf Eskalation und nicht bloß auf kontroverse Diskussionen hin angelegt \parencite[273f.]{kellerGeschichteTalkshowDeutschland2009}.

Wenn man so will wurde die politische Diskussion einer Art Gladiatorenkampf untergeordnet \parencite[151]{faircloughPoliticalDiscourseMedia1998}.Wenig überraschend, dass Kritiker schon damals zu eher ernüchternden Einschätzungen hinsichtlich der Qualität dieser Shows gelangten. Positiv wird einzig hervorgehoben, dass das etablierte Gästekarussell aufgebrochen wurde und auch Normalbürger vermehrt in diese Sendungen eingeladen wurden \parencite[293f.]{kellerGeschichteTalkshowDeutschland2009}.

\subsection{1990er: Talk im Turm und Sabine Christiansen}

Während in den 1990er Jahren zahlreiche Dailytalks wie \textit{Hans Meiser} oder \textit{Arabella} entstanden \parencite[332-344]{kellerGeschichteTalkshowDeutschland2009}, und sich bereits über eine „Talkshow-Epidemie“ \parencite{stolleMeisenUndMystinguetten1990} beklagt wurde, wagte Sat.1 mit \textit{Talk im Turm} die Rückkehr zum ruhigeren politischen Räsonnement im Fernsehen und war damit Vorbild für den heutigen „modernen Polittalk“ \parencite{wickSchlussMitStreit2006}. Ab 1990 entschleunigte und vertiefte Moderator Erich Böhme mit seinem „gelassen-skeptischen Moderationsstil“ \parencite[301]{kellerGeschichteTalkshowDeutschland2009} die Diskussion und hatte damit bei Zuschauern und Kritikern Erfolg. Rückblickend ist die Rede davon, dass es sich um den „ersten, bald allgemein als seriös anerkannten politischen [Talk]“ handelte \parencite[13]{gaeblerUndUnserenTaeglichen2011}. Ende 1998 gab Böhme die Sendung auf. Sein Nachfolger Stefan Aust konnte nicht an den Erfolg anknüpfen, sodass \textit{Talk im Turm} kurze Zeit später eingestellt wurde\footnote{2005 kehrte Böhme mit \textit{Talk der Woche} kurz zu Sat.1 zurück, konnte aber die an \textit{Sabine Christiansen} verlorene Zuschauergunst nicht wieder erlangen \parencite[303]{kellerGeschichteTalkshowDeutschland2009}.} \parencite[300-303]{kellerGeschichteTalkshowDeutschland2009}.

Im Frühjahr desselben Jahres hatte die ARD auf die private Konkurrenz reagiert und mit \textit{Sabine Christiansen} ebenfalls einen im Wochenrhythmus abendlich stattfindenden Polittalk nach dem privaten Vorbild eingerichtet. Die Kritiken waren zwar nicht durchweg gut, dem Erfolg bei den Zuschauern und der politischen Klasse tat dies aber keinen Abbruch \parencite[301-304]{kellerGeschichteTalkshowDeutschland2009}. Legendär ist bis heute der Ausspruch des damaligen stellvertretenden Fraktionsvorsitzenden von CDU/CSU im Bundestag, Friedrich Merz, in der 250. Folge von \textit{Sabine Christiansen}:

\begin{quote}
	„Mir liegt es zunächst am Herzen – Sie haben ja heute Ihre 250. Sendung –, ich finde, wir sollten Ihnen erst einmal gratulieren zu dieser Sendung. Diese Sendung bestimmt die politische Agenda mehr als der Deutsche Bundestag.“ \parencite[zit. nach][15]{gaeblerUndUnserenTaeglichen2011}
\end{quote}

Neben solchen Lobpreisungen gab es allerdings auch scharfe Kritik. So stellt Kümmel (2011) rückblickend fest: Sabine Christiansen „galt als Gastgeberin der Großindustrie, in ihrer Sendung erschien Deutschland als ein von Streikwütigen und müden Arbeitnehmern bedrohter Wirtschaftsstandort.“ Und \textcite{wickSchlussMitStreit2006} meint anlässlich der Einstellung der Sendung 2006, „[$\ldots$] was die politische Klasse in Deutschland zu diskutieren hatte, wurde am Sonntagabend in der ARD festgelegt. [\ldots] Mit der ihr eigenen Mischung aus journalistischem Alarmismus und ritualisierter Langeweile führte Christiansen uns alle sicher durch die Aufs und Abs der letzten Jahre: Vom Börsenboom bis zum rot-grünen Neuanfang, vom Zusammenbruch der Twin Towers in New York bis zu George Bushs Kampf gegen die 'Achse des Bösen' reduzierte diese Sendung alles Ungemach auf das handhabbare Maß eines Gesprächsabends.“

Mit \textit{Sabine Christiansen} wurde der bis heute gültige Orientierungspunkt für nachfolgende Polittalks gesetzt, kaum eine Sendung, die nicht mir ihr verglichen wird. Auch der Trend den Sendungen den Namen des Moderators zu geben wurde mit der Sendung geboren – zuvor war dies nur beim Dailytalk üblich\footnote{Dabei wird der Moderator gleichsam als eine Marke etabliert, die auch außerhalb der jeweiligen Talkshow verwendet werden kann. Entsprechend soll beispielsweise Maybrit Illner vom ZDF nicht ohne Grund „im Wahljahr stärker für Sonderformate eingesetzt werden“ \parencite{o.a.SieversStattIllner2012}.}.

Dieser kurze Überblick über die Geschichte der politischen Diskussionssendungen bzw. politischen Talkshows im Fernsehen, sollte deutlich gemacht habe, dass das Genre das Fernsehen quasi seit Anbeginn begleitete und dabei überaus erfolgreich war. Trotz Vor"-läufer"-sendungen im Hörfunk, handelt es sich nicht einfach um ein gefilmtes Hör"-funk"-ge"-spräch. Vielmehr wurde bereits in den 1950ern erkannt, dass das Hinzufügen einer visuellen Komponente, ein eigenständiges Genre schuf mit eigenen Regeln und Methoden. Vieles was heute als innovativ und neu verkauft wird -- bspw. Einspielfilme -- schon in früheren Sendungen, wenn auch eventuell in abgewandelter Form, zu finden gewesen\footnote{Mit \textit{Roche \& Böhmermann} existiert inzwischen sogar eine Talkshow die explizit den „Stil des frühen Fernsehens“ (ironisch) imitieren möchte \parencite{o.a.RocheBohmermanno.J.}.}.

\subsection{Die heutige Situation}

Bei einer Momentaufnahme im August 2012 fanden sich im deutschen Fernsehen circa 46 Talkshow (vgl. Tabelle \vref{tab:anhang_talkshows}), noch 1998 wurden 65 Sendungen gezählt \parencite[600]{eimerenTalkshowsFormateUnd1998}. Thematisch ist das Spektrum weit gespannt, von Sport, über Bürgerforen, politischen und philosophischen Talkrunden bis hin zu Prominententalks. Auffällig ist, dass die privaten Sender im Vergleich nur eine relativ geringe Anzahl derartiger Sendungen anbieten. Nicht nur bei den Polittalks scheint also weiterhin zu gelten, dass diese eine Domäne der öffentlich-rechtlichen Sender sind \parencite[137]{doernerPolitainmentPolitikMedialen2001}.

Unter den zahlreichen Sendungen haben die Polittalks, welche zur besten Sendezeit in den Hauptprogrammen der Öffentlich-Rechtlichen Fernsehanstalten laufen, einen besonderen Stellenwert. Von den Sendern werden sie gerne als Aushängeschild genutzt, von führenden Politikern gerne besucht und von der Öffentlichkeit aufmerksam beobachtet und gerne auch kritisiert. Zu diesen Aushängeschildern zählen hauptsächlich \textit{Günther Jauch} (ARD) und \textit{maybrit illner} (ZDF), aber auch die weiteren hier untersuchten Sendungen der ARD-Talkschiene spielen in der deutschen Talklandschaft eine hervorgehobene Rolle. Einzig \textit{log in} (ZDF) ist eine recht kleine, sowohl vom Publikum als auch von der Presse weitgehend unbeachtete, Talksendung.

\section{maybrit illner}

Die älteste, weiterhin im Programm befindliche Polittalkshow wird von Maybrit Illner moderiert. Begonnen hatte Illner ihre Fernsehkarriere im DDR-Fernsehen, 1992 wurde sie Moderatorin des ZDF \textit{Morgenmagazins}, bevor sie schließlich im Oktober 1999 den Polittalk \textit{Berlin Mitte} (ebenfalls ZDF) übernahm, der im März 2007 ihren Namen bekam. 2002, 2005 und 2009 war sie zudem an der Moderation der Kandidatenduelle im Vorfeld der jeweiligen Bundestagswahlen beteiligt \parencite{maybritillnerMaybritIllnerGesicht2007}.

Am 17. November 2011 moderierte sie die 500. Folge von \textit{maybrit illner}. Die Sendung  genießt mit einem Marktanteil von durchschnittlich zwölf Prozent \parencite{keilBerlinerSalon2011} einen weiterhin relativ großen Anklang beim Publikum. Zudem bekam Illner für ihre Sendung bereits diverse Medienpreise \parencite[390]{nielandTalkshowisierungWahlkampfesAnalyse2002}.

Laut Selbstbeschreibung des ZDF hat die Sendung den Anspruch zwischen Politik und Bevölkerung zu vermitteln, hierbei zeichne sich Illner durch Kompetenz, Hartnäckigkeit, Charme, Schlagfertigkeit und Humor aus \parencite{maybritillnerMaybritIllnerGesicht2007}. \textcite{keilBerlinerSalon2011} nennt sie den „gesamtdeutsche[n] Gegenentwurf zu Sabine Christiansen: charmant, nahezu angstfrei, keck und nur ein bisschen rechthaberisch“. Illner selbst beschreibt das Ziel ihrer Show wie folgt:

\begin{quote}
	„Es soll um Positionen und Haltungen gestritten werden. Das heißt, wir wollen keine Porträts von Menschen liefern wie in den Personality-Talks und keine „Sozialpornos“ wie in kommerziellen Sendern. Wir versuchen zu verstehen, worin die Interessenkonflikte in diesem Land bestehen. [\ldots] eine Talkshow muss nicht investigativ sein. Ihre Aufgabe ist es, einen guten Seismographen dafür zu haben, was Menschen politisch bewegt.“ \parencite{keilIchBinNicht2011}
\end{quote}

Ihre Sendung solle Orientierung vermitteln in einer Zeit wachsender Ratlosigkeit auf Seiten der Politiker angesichts einer komplexen und sich schnell verändernden Welt \parencite{keilIchBinNicht2011}. Außerdem hält sie sich zugute, auch Gäste jenseits des „Polit- und Funktionärsmillieus“ einzuladen \parencite{brostMaybritIllnerWir2011}.

Das Anfang 2011 umgebaute Studio spiegelt diesen seriösen Anspruch wider. Farblich dominieren dunkle Blautöne und weiß. Die Sitzordnung entspricht dem klassischen Gespräch am runden Tisch. Die Gäste sitzen relativ eng beieinander mit der Moderatorin an einem Tisch, während das Saalpublikum im Halbkreis um das etwas erhöhte Podium verteilt ist. Einzelgespräche werden an einem separaten Stehtisch geführt.

Produziert wird die Sendung vom ZDF selbst in einer zwölfköpfigen Redaktion in Zusammenarbeit mit der doc.station Medienproduktion GmbH \parencite[101]{zweitesdeutschesfernsehenZDFJahrbuch20112012}. Gesendet wird wöchentlich mittwochs, jeweils 60 Minuten ab 22.15 Uhr aus dem ZDF-Hauptstadtstudio in Berlin. Zu Gast sind in der Sendung meist sechs Personen aus Politik und Gesellschaft.

\section{Günther Jauch}

\textit{Günther Jauch} ist der jüngste Spross der ARD-Talkfamilie und zugleich ihr Flaggschiff. Ursprünglich sollte Jauch bereits 2006 als Nachfolger von Sabine Christiansen engagiert werden, dies scheiterte aber. Erst im zweiten Anlauf konnte Jauch gewonnen werden, sodass seit dem 11. September 2011 am Sonntagabend nach dem Tatort um 21.45 Uhr statt \textit{Anne Will} nun \textit{Günther Jauch} in der ARD zu sehen ist \parencite[19]{gaeblerUndUnserenTaeglichen2011}.

Günther Jauch kann auf eine langjährige Fernseherfahrung zurückblicken, am bekanntesten dürfte \textit{Wer wird Millionär?} (RTL) sein, wo er seit 1999 auftritt \parencite{ardUberGuntherJaucho.J.}.

Von allen hier untersuchten Sendungen besitzt \textit{Günther Jauch} sicherlich das spektakulärste Studio und setzt damit die Tradition von \textit{Sabine Christiansen} fort \parencite[355]{nielandTalkshowisierungWahlkampfesAnalyse2002}. In einem alten Gasometer in Berlin-Schöneberg befindet sich die ehemalige „Bundestagsarena“\footnote{Bei der „Bundestagsarena“ handelt es sich um einen Bau, der Berlinbesucher während der FIFA Fußballweltmeisterschaft der Herren 2006 über den Bundestag unterrichten sollte. Die Kuppel des Gebäudes ist der des Reichstags nachempfunden \parencite{deutscherbundestagBundestagsarenaBesucherUndo.J.}.}. Durch den an die Reichstagskuppel erinnernden Aufbau der Arena wird deutlich gemacht, dass es bei \textit{Günther Jauch} um die große Politik gehen soll: Hier werden die wichtigen politischen Fragen der Berliner Republik verhandelt. Die Industrieoptik suggeriert zusätzlich, dass es hart und ernsthaft zur Sache geht.

Entsprechend der Aufmachung nehmen sich auch die Gäste aus. Jauch war der einzige Moderator, dem im untersuchten Zeitraum gelang Bundeskanzlerin Merkel als Gast zu gewinnen. Im Schnitt bestreitet Jauch die Sendung mit fünf Personen.

Ansonsten folgt der Studioaufbau klassischen Regeln. Die Gäste sitzen leicht erhöht auf Sesseln im Halbkreis, den Moderator in der Mitte zwischen ihnen. Das Studiopublikum ist wiederum im Halbkreis um die Bühne herum positioniert. Etwas abgesetzt bietet ein zweites Sesselpaar die Möglichkeit abseits von der Hauptrunde Gespräche zu führen.

Produziert wird die Sendung im Verantwortungsbereich des NDR von der i\&u Information und Unterhaltung TV Produktion GmbH \& Co. KG, deren Alleingesellschafter Günther Jauch ist \parencite{iutvProfilo.J.}. Ein Modell, das sich auch bei den anderen ARD-Talks findet und es den Moderatoren u.a. erlaubt über Tarif zu verdienen.

\section{Anne Will}

\textit{Anne Will} ging ursprünglich 2007 als Nachfolge von \textit{Sabine Christiansen} in der ARD on air \parencite[19]{gaeblerUndUnserenTaeglichen2011}. 2011 wurde die Sendung im Zuge des Starts von \textit{Günther Jauch} vom Sonntagabend auf den Mittwochabend verlegt und auf 75 Minuten verlängert.

Anne Will erlangte 1999 als Moderatorin der \textit{Sportschau} Bekanntheit und moderierte von 2001 bis 2007 die \textit{Tagesthemen}. Laut Selbstbeschreibung führt sie in ihrer Sendung „intensive Gespräche über aktuelle gesellschaftspolitische Fragen“ \parencite{ardAnneWillPersoenlicho.J.}.

Um sich von ihrer Vorgängerin abzuheben, wurde das sogenannte „Betroffenen-Sofa“ eingeführt, dieses „sollte besondere Nähe zu den Problemen der einfachen Menschen suggerieren“ \parencite[20]{gaeblerUndUnserenTaeglichen2011} und so den Verdacht der Komplizenschaft mit dem Berliner Politestablishment zerstreuen. Zu Gast sind im Schnitt sechs Personen.

Das Studio ist in dunklen Brauntönen gehalten, die Gäste sitzen in Sesseln in einem angedeuteten Rechteck um einen flachen Tisch herum, die Moderation ist genau in der Mitte platziert. Das Publikum im Saal sitzt frontal zum leicht erhöhten Podium.

Verantwortet wird \textit{Anne Will} vom NDR, produziert und redaktionell betreut allerdings von Will Media, Anne Wills eigener Produktionsfirma, in Berlin. Ausgestrahlt wird die Sendung wöchentlich mittwochs ab 22.45 Uhr.

\section{hart aber fair}\label{chap:hartaberfair}

Bereits 1953 wurde in der Zeitschrift „Fernsehen“ angeregt, anlässlich anstehender Wahlen eine Diskussionssendung einzurichten in der die Diskussionen zwischen den Parteivertretern „ebenso hart wie fair“ \parencites[353]{o.a.LebendigeWahlsendungen1953}[zit. nach][124]{kellerGeschichteTalkshowDeutschland2009}  sein sollten. Eine Talkshow die sich demonstrativ dieses Motto als Titel gab, kam allerdings erst im Januar 2001 ins deutsche Fernsehen.

\textit{hart aber fair} mit dem Moderator Frank Plasberg wurde zuerst im Dritten Programm des WDR ausgestrahlt, auf Grund des großen Erfolgs wurde die Sendung dann aber im Jahr 2007 ins Hauptprogramm der ARD geholt.

Dieser Erfolg dürfte sich zu guten Teilen mit der – zumindest unterstellten – Neuartigkeit der Sendung erklären. Plasberg wurde eine „erfrischende Andersartigkeit“ \parencite[20]{gaeblerUndUnserenTaeglichen2011} zugeschrieben, er sorge für einen dynamischen Sendungsverlauf und befrage die Teilnehmer mit der nötigen journalistischen Härte. Auch die Nutzung von Einspielern, um die Diskussion weiter zu bringen, die live eingebrachten Zuschauerkommentare („Zuschaueranwältin“) und der „Faktencheck“, die nachträgliche Überprüfung von Behauptungen der Gäste im Internet, fanden Anklang. Insgesamt sei es durch \textit{hart aber fair} zu einer „Reanimation eines gesunden, skeptischen politischen Journalismus“ \parencite[20]{gaeblerUndUnserenTaeglichen2011} gekommen.

Laut Selbstbeschreibung möchte die Sendung verständlich, umfassend und informativ sein. Die Einspieler sollen Hintergründe, harte Fakten und verschiedene Perspektiven auf das Thema liefern \parencite{ardHartAberFairo.J.}. Plasberg beschreibt seinen Auftrag selbst wie folgt: „Jeder wird so lange Auskunft geben müssen, bis die Frage wirklich beantwortet ist“ \parencite{ardHartAberFairo.J.a}.

Plasberg selbst ist ausgebildeter Journalist, der von der Zeitung über den Hörfunk zum Fernsehen kam. Neben seiner Talkshow moderiert er zahlreiche Einzelsendungen, vertrat 2009 die ARD beim Kanzlerduell und hat seit 2008 eine zweite Talkshow mit dem Titel \textit{plasberg persönlich} im WDR-Fernsehen \parencite{ardHartAberFairo.J.a}.

Gesendet wird das vom WDR verantwortete 75-minütige \textit{hart aber fair} aus Studios in Köln und Berlin. Produziert wird die Sendung von „Ansager und Schnipselmann“, Plasbergs eigener Firma \parencite{o.a.AnsagerUndSchnipselmann2005}. Ausgestrahlt wird jede Woche Montag ab 21 Uhr, also früher als die anderen Sendungen der ARD-Talkschiene.

Der Aufbau der Studios bildet den konfrontativen Anspruch der Sendung ab. Dominierende Farbe ist rot, die vier bis fünf Gäste sitzen erhöht an einem leicht gebogenen Podest. Der Moderator steht neben dem Podest und ist damit demonstrativ nicht Teil der Talkrunde. Das Studiopublikum sitzt auf zwei, ebenfalls leicht gebogenen „Tribünen“ den Gästen gegenüber, die Tribünen besitzen eine geschlossene Front. Zwischen beiden  ist das Pult der „Zuschaueranwältin“ Brigitte Büscher angesiedelt, zusätzlich gibt es einen Stehtisch für Einzelgespräche. Insgesamt erinnert das Studio so mehr an eine Zirkusmanege oder Arena als an das in den meisten übrigen Sendungen übliche Agora-Modell.

\section{Menschen bei Maischberger}

\textit{Menschen bei Maischberger} ist seit September 2003 im Ersten zu sehen. Die Show trat die Nachfolge von \textit{Boulevard Bio} an, einer eher als Personality-Talkshow zu charakterisierenden Sendung, bei der es um die Darstellung der (wahren) Persönlichkeit eines Gastes und nicht um harte Fakten oder politische Interessenkonflikte geht \parencite{gehringerBoulevardBioNur2001}.

Sandra Maischberger, die namensgebende Moderatorin, sammelte als Co-Moderatorin von Erich Böhme bei \textit{Talk im Turm} und später bei \textit{0137} erste Erfahrungen im Bereich Polittalk.

Obwohl die Sendung schon vom Titel her weniger den Eindruck eines Polittalks als eines gediegenen Gesprächsabends über dieses und jenes erweckt, stellt die ARD in der Selbstbeschreibung doch die journalistischen Qualitäten von \textit{Menschen bei Maischberger} heraus. Bei Maischberger seien „Neugier und Intelligenz gepaart mit Unnachgiebigkeit und journalistischem Instinkt“, denn Journalismus sei ihre „Berufung“ \parencite{ardSandraMaischbergero.J.}. Dennoch handelt es sich zusammen mit \textit{Beckmann} und im Vergleich zu den anderen Sendungen eher um eine „weiche“ Polittalkshow, wie die spätere Analyse noch zeigen wird.

Entsprechend gestaltet ist das Studio. Die Gäste sitzen im Halbrund in bequemen Sesseln bzw. auf Sofas, die Moderation ungefähr in der Mitte zwischen ihnen. Warme Farben dominieren die Dekoration und die Wirkung ist insgesamt freundlich. Auf Studiopublikum wird verzichtet.

Produziert wird die 75-minütigen Sendung extern von der Vincent TV GmbH für den WDR und jede Woche dienstags um 22.45 Uhr ausgestrahlt. Die Gästezahl beträgt im Schnitt sechs.

\section{Beckmann}

Mehr noch als bei \textit{Menschen bei Maischberger} ist \textit{Beckmann} im Spannungsfeld zwischen (harter) Polit- und (weicher) Personality-Talkshow angesiedelt. Entsprechend ist es umstritten, ob die Sendung überhaupt als politische Talkshow zählen kann. Da sie allerdings Teil der ARD-Talkschiene ist, explizit als Polittalk beworben wird und auch – zumindest teilweise – ein entsprechendes Gäste- und Themenspektrum aufweist, wird sie hier als solche behandelt.

Bekannt wurde Beckmann als Sportjournalist, insbesondere durch die Fußballsendung \textit{ran} (Sat.1). Im Januar 1999 startet in der ARD seine Talkshow, zudem ist er weiterhin  regelmäßig als Kommentator bei Fußballspielen tätig. Beckmanns Rolle als Moderator beschreibt die ARD folgendermaßen:

\begin{quote}
	„Mit 'Beckmann' stellt er seine Fähigkeit im persönlichen, konzentrierten Gespräch mit prominenten und ungewöhnlichen Gästen unter Beweis: sensibel und kritisch, humorvoll und schlagfertig, intelligent und professionell.“ \parencite{ardBeckmannModeratoro.J.}
\end{quote}

Das Studio befindet sich in der Hamburger Speicherstadt und war eigenen Angaben zufolge einem Loft in der Speicherstadt nachempfunden \parencite{ardBeckmannStudioo.J.}. Die Gäste saßen mit dem Moderator zusammen an einem rechteckigen Tisch. Je nach Gästezahl saß Beckmann diesen gegenüber oder war Teil der Runde. Durch diese Konstellation saßen die Teilnehmer sehr nah beieinander, wodurch eher der Eindruck eines Stammtischs denn einer Diskussionsrunde entstand. Verstärkt wurde dieser Eindruck noch durch den Verzicht auf Studiopublikum. Seit dem 1. November 2012 verfügt die Sendung über ein neues Studiodesign, welches heller und moderner wirkt und dessen Sitzarrangement eher an \textit{maybrit illner} erinnert \parencite{ardNeuesBeckmannStudioo.J.}.

Produziert wird \textit{Beckmann} von Cinecentrum in Hamburg im Auftrag von Beckground TV, an der, wie sich unschwer erahnen lässt, Reinhold Beckmann beteiligt ist. Verantwortet wird die Sendung vom NDR. Gesendet wird wöchentlich am Donnerstagabend 75 Minuten lang ab 22.45 Uhr.

\section{log in}

Stark auf Einbindung der Zuschauer setzt \textit{log in}. Die recht junge Sendung des ZDF wird seit November 2010 im Digitalkanal ZDFinfo gesendet \parencite{kluczniokLogZDFUnd2010}.

The European beschreibt die „interaktive Talkshow“ als „ein neuartiges Format im deutschen Fernsehen“ \parencite{o.a.WolfChristianUlrich2011}. Ob die Sendung wirklich so neuartig ist, darüber lässt sich trefflich streiten, liest sich doch beispielsweise eine Beschreibung der Sendung \textit{Giga real} (2004-2006, NBC Europe / GIGA TV) wie eine Blaupause zu \textit{log in}:

\begin{quote}
	„In 'Giga real' verhandelten zwei Moderatoren, jugendliche Gäste und meist ein Gesprächspartner aus der Politik Themen wie '60 Jahre Kriegsende, Studiengebühren, Sterbehilfe, Parlamentswahlen in England oder Wahlen in NRW'. [\ldots] Den Fernsehzuschauern bot sich Gelegenheit zur Teilnahme via Mail und als Diskussionspartner im Internet-Forum [\ldots].“ \parencite[222]{kellerGeschichteTalkshowDeutschland2009}
\end{quote}

Auch \textit{log in} hat zwei Moderatoren – aktuell Jeannine Michaelsen und Wolf-Christian Ulrich –, gibt sich betont jugendnah und die Zuschauer können im Vorfeld über eine spezielle Internetseite Fragen stellen. Zudem werden Tweets von Zuschauern in den Sendungsverlauf eingebunden und diese können im Lauf der Sendung über zwei Statements der beiden Hauptgäste abstimmen. Die Aufgabe das Netz zu beobachten kommt Michaelsen zu, die damit eine ähnliche Rolle hat wie die „Zuschaueranwältin“ bei hart aber fair, aber ständig anwesend und wesentlich stärker in den Sendungsverlauf eingebunden ist. Entsprechend liest sich die Selbstbeschreibung:

\begin{quote}
	„Der interaktive Polit-Talk »ZDF log in« diskutiert wöchentlich Themen, die in der politischen und gesellschaftspolitischen Auseinandersetzung – auch im Internet – eine wichtige Rolle spielen. Die Gäste der Sendung werden mit Fragen und Reaktionen aus dem Netz konfrontiert.“ \parencite{zweitesdeutschesfernsehenZDFinfoErlauterung2012}
\end{quote}

Gesendet wird \textit{log in} seit August 2012 jeden Mittwoch ab 22.20 Uhr für eine Stunde, zuvor lag der Zeitpunkt zwei Stunden früher auf 21 Uhr. Zu Gast sind im Schnitt bloß drei Gäste.

Produziert wird im ZDF-Hauptstadtstudio. Der Studioaufbau ist lockerer als bei den anderen Sendungen. In der Mitte stehen zwei Pulte, an denen sich die meist zwei Hauptgäste gegenüberstehen. Die Konfrontation von Standpunkten wird hier also schon visuell stärker verdeutlicht als bei den anderen Talks. Die Onlineredakteurin Michelsen hat ein eigenes Pult mit eingelassenem Touchscreen etwas abseits hiervon. Das Studiopublikum ist um diese Anordnung herum auf mehrere Tribünen verteilt. Farblich wirkt das Studio relativ kalt und modern.

\section{Unterschiede \& Gemeinsamkeiten}

Die sieben Sendungen im Vergleich offenbaren einige Unterschiede und Gemeinsamkeiten (vgl. auch Tabelle \vref{table:vergleich_sendungen} für einen Überblick) Allen Sendungen gemein ist ihr wöchentlicher Senderhythmus. Hinsichtlich der Sendezeit zeigen sich allerdings bereits Unterschiede. \textit{hart aber fair} wird am frühsten – um 21 Uhr – ausgestrahlt. Danach folgt 21.45 Uhr bei \textit{Günther Jauch}, der von dem davor ausgestrahlten Tatort profitiert. Die anderen sieben Sendungen beginnen alle recht spät zwischen 22.15 und 22.45 Uhr. Parallel gesendet werden \textit{Menschen bei Maischberger} und \textit{log in} am Mittwoch sowie \textit{Beckmann} und \textit{maybrit illner} am Donnerstagabend.

Die Länge der Sendungen variiert zwischen 60 und 75 Minuten und die durchschnittliche Zahl an Studiogästen zwischen fünf und sechs. \textit{log in} ist die große Ausnahme, sind doch hier im Schnitt nur drei Gäste zugegen, von denen zumal nur zwei die Hauptgäste sind und alle anderen erst im weiteren Sendungsverlauf dazu stoßen.

Unterschiede finden sich auch in der Studiogestaltung. Die reicht vom klassischen Aufbau nach Art eines Forums bei \textit{Anne Will} und \textit{Günther Jauch}, über konfrontativere Arenen bei \textit{hart aber fair} und \textit{log in}, bis hin zu einem runden Tisch bei \textit{maybrit illner} und dem eher gemütlichen Stammtisch bzw. Kamingespräch bei \textit{Beckmann}. In diesen verschiedenen Studiotypen spiegeln sich die verschiedenen Selbstbilder der Sendungen. Insbesondere bei \textit{Beckmann} und \textit{Menschen bei Maischberger} entspricht das „entspannte“, gemütliche Design und der Verzicht auf Studiopublikum der Nähe zur Personality-Show. \textit{Günther Jauch}, \textit{log in}, \textit{hart aber fair} und \textit{maybrit illner} hingegen versuchen durch das Design ihre Seriosität, Ernsthaftigkeit und Relevanz auszudrücken. Dabei hat die Gestaltung des Studios auch Einfluss auf das Kommunikationssetting und damit auf den Ablauf der Diskussionen.

Die Marktanteile der Sendungen unterscheiden sich ebenfalls. \textit{Günther Jauch} ist erwartungsgemäß der Marktführer mit durchschnittlich 4,5 Millionen Zuschauern. Dahinter folgen \textit{Anne Will}, \textit{Menschen bei Maischberger}, \textit{maybrit illner} und \textit{hart aber fair} mit dreizehn bis elf Prozent Marktanteil. Abgeschlagen folgt \textit{Beckmann} mit bloß 7 \%, hier schlägt sich die direkte Konkurrenz zu \textit{maybrit illner} nieder. Zu \textit{log in} liegen leider keine Daten vor. Der Marktanteil dürfte aber wesentlich geringer sein, so betrug er beispielsweise bei der Sendung vom 12. Dezember 2012 gerade einmal 0,5 \%.

Keiner der Polittalks hat die Reichweite der alten Flaggschiffe. \textit{Sabine Christiansen} erreichte beispielsweise 2002 im Schnitt noch einen Marktanteil von 18,4 \% (4,9 Millionen Zuschauer) und auch Maybrit Illners alte Sendung \textit{Berlin direkt} lag damals noch bei einem Schnitt von 13,3 \% (2,6 Millionen Zuschauer) \parencite[298f.]{schultzModerationPolitischerGesprachsrunden2004}.

\clearpage
%%% -----------------------------------------------
\KOMAoptions{pagesize,paper=landscape,DIV=13}
%%% -----------------------------------------------


\begin{table}
    \centering
    \caption{Untersuchte Sendungen im Vergleich}
    \resizebox{\textwidth}{!}{%
    \begin{tabular}{llllllll}
    \hline
        & \textbf{Anne Will} & \textbf{Beckmann} & \textbf{Günther Jauch} & \textbf{hart aber fair} & \textbf{Maischberger} & \textbf{log in} & \textbf{maybrit illner} \\ \toprule
        Moderator & Anne Will & Reinhold Beckmann & Günther Jauch & Frank Plasberg & Sandra Maischberger & \makecell[l]{Jeannine Michaelsen \\ Wolf-Christian Ulrich} & Maybrit Illner \\ \hline
        \makecell[l]{Typ \\ (nach \textcite[32f.]{plakeTalkshowsIndustrialisierungKommunikation1999})} & Forum & Forum / Personality-Show & Forum & Forum & Forum / Personality-Show & Forum & Forum \\ \hline
        Produzent & Will Media & Beckground TV & i\&u TV & Ansager \& Schnipselmann & Vincent TV & ZDF & ZDF \\ \hline
        Verantw. Sender & NDR & NDR & NDR & WDR & WDR & ZDF & ZDF \\ \hline
        Erstausstrahlung & 2007 & 1999 & 2011 & 2001 & 2003 & 2010 & 1999 \\ \hline
        Senderhythmus & Wöchentlich & Wöchentlich & Wöchentlich & Wöchentlich & Wöchentlich & Wöchentlich & Wöchentlich \\ \hline
        Sendezeitpunkt & Dienstag & Donnerstag & Sonntag & Montag & Mittwoch & Mittwoch & Donnerstag \\ \hline
        Uhrzeit & 22.45 & 22.45 & 21.45 & 21.00 & 22.45 & 22.20 & 22.15 \\ \hline
        \makecell[l]{Dauer \\ (Minuten)} & 75 & 75 & 60 & 75 & 75 & 60 & 60 \\ \hline
        \makecell[l]{Gäste \\ (Median | Min | Max)} & 6(5 | 7) & 5(1 | 7) & 5(1 | 7) & 5(5 | 6) & 6(1 | 7) & 3(2 | 6) & 6(4 | 7) \\ \hline
        Studiopublikum & Ja & Nein\footnote{Normalerweise gibt es bei \textit{Beckmann} kein Studiopublikum. Ausnahmsweise kann dies aber dennoch der Fall sein. So zum Beispiel bei der Folge vom 27. Oktober 2011.} & Ja & Ja & Nein & Ja & Ja \\ \hline
        Studiotyp & Forum & Stammtisch & Forum & Arena & Kamingespräch & Arena & Runder Tisch \\ \hline
        \makecell[l]{Marktanteil \\ (Zuschauer)}\footnote{Angaben nach \textcite[128]{zubayrTendenzenImZuschauerverhalten2012}, die Zahlen beziehen sich auf 2011. Bei \textit{Maischberger} und \textit{Beckmann} lagen aktuellere Zahlen vor \parencite{o.a.MitVerwunderung2012}.} & 13,1 \%(3,04 Mio) & 7 \% & 15,6 \%(4,52 Mio) & 11,3 \%(3,09 Mio) & 12 \% & k.A. & 12 \%(2,47 Mio) \\ \hline
    \end{tabular}
    }
\label{table:vergleich_sendungen}
\end{table}

\clearpage
%%% -----------------------------------------------
\KOMAoptions{pagesize,paper=portrait,DIV=10}
%%% -----------------------------------------------
